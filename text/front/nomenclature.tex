\renewcommand\nomgroup[1]{%
    \item[\bfseries
    \ifstrequal{#1}{A}{General Mathematics}{%
    \ifstrequal{#1}{B}{Complex Numbers}{%
    \ifstrequal{#1}{C}{Number sets}{%
    \ifstrequal{#1}{D}{Probability}{%
    \ifstrequal{#1}{E}{Vectors, Matrices, and Arrays}{%
    \ifstrequal{#1}{F}{Calculus}{%
    \ifstrequal{#1}{O}{Other symbols}{}}}}}}}%
]}

%%% General maths
\nomenclature[A,01]{$\mathcal{A} \coloneq \mathcal{B}$}{$\mathcal{A}$ is defined to be equal to $\mathcal{B}$}
\nomenclature[A,02]{$\lfloor a \rfloor$}{The nearest integer to $a$, such that $\lfloor a \rfloor \leq a$}
\nomenclature[A,03]{$\lceil a \rceil$}{The nearest integer to $a$, such that $\lceil a \rceil \geq a$}
\nomenclature[A,04]{$\lfloor a \rceil$}{The nearest integer to $a$}
\nomenclature[A,05]{$a \bmod b$}{$a$ modulo $b$, given by \[a \bmod b = a - \left\lfloor \frac{a}{b} \right\rfloor\]}
%%% Complex numbers
\nomenclature[B,01]{$\iu$}{The imaginary unit, $\iu \coloneq \sqrt{-1}$}
\nomenclature[B,02]{$\Re(z)$}{The real component of $z$}%
\nomenclature[B,03]{$\Im(z)$}{The imaginary component of $z$}%
\nomenclature[B,04]{$\left\lvert z \right\rvert$}{
    The absolute value of $z$, given by
    \[
        \left\lvert z \right\rvert =
            \sqrt{
                \Re(z)^2 + \Im(z)^2
            }
    \]
}
%%% Arrays
\nomenclature[E,01]{$\cdot^{\mathrm{T}}$}{Transpose}%
\nomenclature[E,02]{$\cdot^{\dagger}$}{Conjugate transpose}
\nomenclature[E,03]{$\cdot^{-1}$}{Inverse}
\nomenclature[E,04]{$\cdot^+$}{Moore-Penrose pseudo-inverse.}
\nomenclature[E,05]{$\cdot^{\circlearrowright (d)}$}{Right circular rotation by along axis $d$ of the array by one element.
    For example, for a 2D matrix $\symbf{X} \in \mathbb{F}^{M \times N}$, $\symbf{X}^{\circlearrowright (1)}$ is given by
    \[
        \begin{bmatrix}
            x_{M, 1} &
            x_{M, 2} &
            \cdots &
            x_{M, N} \\
            x_{1, 1} &
            x_{1, 2} &
            \cdots &
            x_{1, N} \\
            \vdots &
            \vdots &
            \ddots &
            \vdots \\
            x_{M - 1, 1} &
            x_{M - 1, 2} &
            \cdots &
            x_{M - 1, N} \\
        \end{bmatrix},
    \]
    while $\symbf{X}^{\circlearrowright (2)}$ is
    \[
        \begin{bmatrix}
            x_{1, N} &
            x_{1, 1} &
            \cdots &
            x_{1, N - 1} \\
            x_{2, N} &
            x_{2, 1} &
            \cdots &
            x_{2, N - 1} \\
            \vdots &
            \vdots &
            \ddots &
            \vdots \\
            x_{M, N} &
            x_{M, 1} &
            \cdots &
            x_{M, N - 1}
        \end{bmatrix}.
    \]
}
\nomenclature[E,06]{$\cdot^{\leftrightsquigarrow (d)}$}{
    Reversal of elements along axis $d$ of an array.
    For example, for a 2D matrix $\symbf{X} \in \mathbb{F}^{M \times N}$,
    $\symbf{X}^{\leftrightsquigarrow (1)}$ is given by
    \[
        \begin{bmatrix}
            x_{M, 1} &
            x_{M, 2} &
            \cdots &
            x_{M, N} \\
            x_{M - 1, 1} &
            x_{M - 1, 2} &
            \cdots &
            x_{M - 1, N} \\
            \vdots &
            \vdots &
            \ddots &
            \vdots \\
            x_{2, 1} &
            x_{2, 2} &
            \cdots &
            x_{2, N} \\
            x_{1, 1} &
            x_{1, 2} &
            \cdots &
            x_{1, N} \\
        \end{bmatrix},
    \]
}
\nomenclature[E,07]{$\diag(\cdot)$}{
    Produces a diagonal matrix from a vector. Given a vector $\symbf{x} \in
    \mathbb{F}^N$, $\diag(\symbf{x})$ is a matrix $\in \mathbb{F}^{N \times N}$
    of the form
    \[
        \begin{bmatrix}
            x_{1} & 0 & \cdots & 0 \\
            0 & x_{2} & \cdots & 0 \\
            \vdots & \vdots & \ddots & \vdots\\
            0 & 0 & \cdots & x_{N}
        \end{bmatrix}
    \]
}
\nomenclature[E,08]{$\langle \cdot, \cdot \rangle$}{
    Inner product of two arrays with identical shapes. Given $\symbf{X}$ and
    $\symbf{Y}$, two $D$-dimensional arrays $\in
    \mathbb{C}^{N_1 \times \cdots \times N_D}$:
\[
    \langle \symbf{X}, \symbf{Y}\rangle =
    \sum_{n_1 = 1}^{N_1} \cdots
    \sum_{n_D = 1}^{N_D}
        x_{n_1, \cdots, n_D}^*
        y_{n_1, \cdots, n_D}
\]
}
\nomenclature[E,09]{$\lVert \cdot \rVert$}{Norm, equivalent to $\sqrt{\langle \cdot, \cdot \rangle}$.}
\nomenclature[E,10]{$\symbf{x} \otimes \symbf{y}$}{
    Outer product of $\symbf{x} \in \mathbb{F}^M$ and $\symbf{y} \in
    \mathbb{F}^N$, which generates a matrix $\symbf{A} \in \mathbb{F}^{M \times
    N}$, such that
    \[
        a_{m,n} = x_m y_n
    \]
}

\nomenclature[F,01]{$\nabla f(\symbf{x})$}{
    Gradient vector of a differentiable, scalar value function $f(\symbf{x}):
    \mathbb{R}^{N} \rightarrow \mathbb{R}$:
    \[
        \nabla f(\symbf{x}) =
        \begin{bmatrix}
            \frac{\partial f}{\partial x_1} &
            \frac{\partial f}{\partial x_2} &
            \cdots &
            \frac{\partial f}{\partial x_{N}}
        \end{bmatrix}\T
    \]
}
\nomenclature[F,02]{$\nabla^2 f(\symbf{x})$}{
    Hessian matrix of a twice-differentiable, scalar value function
    $f(\symbf{x}): \mathbb{R}^{N} \rightarrow \mathbb{R}$:
    \[
        \nabla^2 f(\symbf{x}) =
        \def\arraystretch{1.4}
        \begin{bmatrix}
            \frac{\partial^2 f}{\partial x_{1}^2} &
            \frac{\partial^2 f}
            {
                \partial x_1^{\vphantom{2}}
                \partial x_2^{\vphantom{2}}
            } &
            \cdots &
            \frac{\partial^2 f}
            {
                \partial x_1^{\vphantom{2}}
                \partial x_N^{\vphantom{2}}
            } \\
            \frac{\partial^2 f}
            {
                \partial x_2^{\vphantom{2}}
                \partial x_1^{\vphantom{2}}
            } &
            \frac{\partial^2 f}{\partial x_{2}^2} &
            \cdots &
            \frac{\partial^2 f}
            {
                \partial x_2^{\vphantom{2}}
                \partial x_N^{\vphantom{2}}
            } \\
            \vdots & \vdots & \ddots & \vdots \\
            \frac{\partial^2 f}
            {
                \partial x_N^{\vphantom{2}}
                \partial x_1^{\vphantom{2}}
            } &
            \frac{\partial^2 f}
            {
                \partial x_N^{\vphantom{2}}
                \partial x_2^{\vphantom{2}}
            } &
            \cdots &
            \frac{\partial^2 f}{\partial x_{N}^2}
        \end{bmatrix}
    \]
}
\nomenclature[D,01]{$x \sim \mathcal{X}$}{$x$ behaves according to distribution $\mathcal{X}$}
\nomenclature[D,02]{$A \upmodels B$}{$A$ and $B$ are conditionally independent}
\nomenclature[D,04]{$\mathcal{U}\left(l, r\right)$}{
    Uniform distribution with bounds $l$ and  $r$:
    \[
        \mathcal{U}(x \hspace*{2pt} \vert \hspace*{2pt} l, r) =
            \begin{cases}
                \frac{1}{r - l} & l \leq x \leq r \\
                0 & \text{otherwise} \\
            \end{cases}
    \]
}
\nomenclature[D,03]{$\mathcal{N}\left(\mu, \sigma^2\right)$}{
    Normal (Gaussian) distribution with mean $\mu$ and variance $\sigma^2$:
    \[
        \mathcal{N}\left(x \hspace*{2pt} \vert \hspace*{2pt}\mu, \sigma^2\right) =
            \frac{1}{\sqrt{2\pi \sigma^2}}
            \exp\left( -\frac{(x - \mu)^2}{2\sigma^2}\right)
    \]
}
\nomenclature[D,04]{$\mathcal{N_C}\left(\mu, \sigma^2\right)$}{Complex normal distribution with mean $\mu$ and variance $\nicefrac{\sigma^2}{2}$}%
% %%% Other symbols
% \nomenclature[O,01]{$\cdot^{(d)}$}{Data dimension index}
% \nomenclature[O,02]{$\cdot_{m}$}{Oscillator index}
% \nomenclature[O,03]{$\foff$}{Transmitter offset}
% \nomenclature[O,04]{$\fsw$}{Sweep width (spectral window)}
% \nomenclature[O,05]{$\Dt$}{Sampling rate}
%%% Sets
\nomenclature[C,01]{$\mathbb{C}$}{The set of complex numbers}%
\nomenclature[C,02]{$\mathbb{N}$}{The set of natural numbers with zero excluded}%
\nomenclature[C,03]{$\mathbb{N}_0$}{The set of natural numbers with zero included}%
\nomenclature[C,04]{$\mathbb{R}$}{The set of real numbers}%
\nomenclature[C,05]{$\mathbb{R}_{>0}$}{The set of positive real numbers}%
\nomenclature[C,05]{$\mathbb{Z}$}{The set of integers}%
\printnomenclature[1in]\label{chap:nomenclature}
