\begin{abstract}
    \Ac{NMR} is an analytical technique employed in many scientific
    disciplines that is able to provide insights into the structures and
    dynamics of chemical species. To maximise the utility of \ac{NMR}
    experiments, appropriate data treatment and analysis is necessary. The
    conventional route to extracting quantitative information from the raw
    experimental data\,---\,the \ac{FID}\,---\,is to convert it to an
    \ac{NMR} spectrum, through application of the \ac{FT}. Such spectra provide a
    human-interpretable representation of data, with trained practitioners able
    to rationalise their appearance by mapping peaks in the spectrum to
    chemical environments in a given sample. However, the \acs{FT} suffers from poor
    resolution, with peaks of similar frequencies often overlapping.
    Disentangling the quantitative information associated with such peaks is
    not feasible using typical methods such as spectrum integration.
    As an alternative, parametric estimation techniques aim to provide detailed
    information about each individual signal present in the data. These have
    been shown to perform effectively even in scenarios where significant
    signal overlap exists.

    This thesis focusses on the development of a parametric estimation method
    for the analysis of \acp{FID} derived from solution-state \ac{NMR}
    experiments. The guiding principle behind the method is that is should
    require as little user input as possible, yet be able to provide
    accurate predictions of the parameters which describe an \ac{FID}'s
    contributing signals.
    Beyond simply providing a breakdown of individual signal components,
    many useful applications may be realised when estimation techniques are
    employed. The initial motivation for this work was to develop a procedure
    for the generation of pure shift \ac{NMR} spectra with desirable properties
    from \ac{2DJ} datasets. Furthermore, a means of
    analysing \acs{2D} datasets such as those from inversion recovery ($T_1$),
    \ac{CPMG} ($T_2$), and diffusion experiments, in which each \ac{FID}
    exhibits a variation in its amplitude, is presented.
    The last application described is a means of producing phased,
    ultra-broadband \ac{NMR} spectra from an experiment comprising a single
    \ac{FS} (chirp) excitation pulse.
    The methods presented in this thesis are incorporated into a software
    package written in the \Python programming language, called \ac{EsPy}.
\end{abstract}
