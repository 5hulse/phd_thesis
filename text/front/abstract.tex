\begin{abstract}
    \Ac{NMR} spectroscopy is an analytical technique employed in many scientific
    disciplines that is able to provide insights into the structures and
    dynamics of chemical species. To maximise the utility of \ac{NMR},
    appropriate data treatment and analysis is necessary. The
    conventional route to extracting quantitative information from the raw
    experimental data\,---\,the \ac{FID}\,---\,is to convert it to an
    \ac{NMR} spectrum, through application of the \ac{FT}. \ac{NMR} spectra
    provide a
    human-interpretable representation of data; trained practitioners are able
    to rationalise the appearance of a given spectrum by mapping its component peaks
    to chemical environments in the sample from which the dataset was acquired.
    However, the \acs{FT} suffers from
    poor resolution, with peaks of similar frequencies exhibiting overlap.
    Disentangling the information associated with such peaks is
    not feasible using typical methods such as integrating user-defined
    regions of the spectrum.
    As an alternative approach, parametric estimation techniques aim to provide
    a detailed description of each signal which contributes to the \ac{FID}.
    These methods have been shown to perform effectively even in scenarios where
    significant spectral peak overlap exists.

    This thesis focusses on the development of a parametric estimation method
    for the analysis of \acp{FID} derived from solution-state \ac{NMR}
    experiments. The guiding principle behind the method is that it should
    require as little user input as possible, while being able to provide
    accurate and reliable signal estimates.
    Beyond simply providing a breakdown of individual signal components,
    many useful applications may be realised when estimation techniques are
    employed. The initial motivation for this work was to develop a procedure
    for the generation of broadband homodecoupled (pure shift) \ac{NMR} spectra
    with desirable properties from \ac{2DJ} datasets. Furthermore, a means of
    analysing datasets such as those from inversion recovery ($T_1$),
    \ac{CPMG} ($T_2$), and diffusion experiments, in which each \ac{FID}
    exhibits a variation in its amplitude, is presented.
    The last application described is a means of producing phased,
    ultra-broadband \ac{NMR} spectra from an experiment comprising a single
    \acl{FS} (chirp) excitation pulse.

    The methods presented in this thesis are incorporated into a software
    package written in the \Python programming language, called \ac{EsPy}, of
    which more information can be found at
    \url{https://github.com/foroozandehgroup/NMR-EsPy}.
\end{abstract}
