\begin{abstract}
    \Acfi{NMR} is an analytical technique employed in many scientific
    disciplines which is able to provide insights into the structures and
    dynamics of chemical species. To maximise the utility of \ac{NMR}
    experiments, appropriate data treatment and analysis is necessary. The
    conventional route to extracting quantitative information from the raw
    experimental data, the \acfi{FID}, is to convert it to an \ac{NMR}
    spectrum, through application of the \acf{FT}. Such spectra provide a
    human-interpretable representation of data, with trained practitioners able
    to rationalise their appearance by mapping peaks in the spectrum to
    chemical environments of species in the sample considered. Despite the
    simplicity of the representation provided, the \ac{FT} suffers from poor
    resolution, often leading to peaks with similar frequencies overlapping.
    Disentangling the quantitative information associated with such peaks is
    not feasible using typical methods such as integration.
    As an alternative, parametric estimation techniques aim to provide detailed
    information about each signal present in the data. These have been shown to
    perform effectively even in scenarios where significant signal overlap
    exists.

    This thesis focusses on the development of a parametric estimation method
    for the analysis of \acp{FID} derived from solution-state \ac{NMR}
    experiments. The guiding principle behind the method is that is should
    require as little user input as possible, while simultaneously providing
    realistic predictions of the component signals which contribute to the
    data. Beyond simply providing a breakdown of individual signal components,
    many useful applications may be realised when estimation techniques are
    employed. The initial motivation for this thesis was to develop a procedure
    for the generation of pure shift \ac{NMR} spectra with desirable properties
    from \acl{2DJ} datasets. Other applications presented here are for the
    analysis of \acs{2D} datasets in which each \ac{FID} exhibits a variation
    in its amplitude --- including inversion recovery ($T_1$), \acs{CPMG}
    ($T_2$) and diffusion experiments ---, and a means of producing phased,
    ultra-broadband \ac{NMR} spectra from an experiment comprising a single
    \acl{FS} excitation pulse.

    The methods presented in this thesis are incorporated into a software
    package written in the \Python programming language, called \acfi{EsPy}.
\end{abstract}
