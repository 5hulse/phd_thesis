\section{Overview of this Work}

\subsection{Conception and Motivation}
The focus of this work is the development of a routine which performs
parametric estimation on \ac{NMR} datasets.
Motivation initially came from discussions within the NMR
Methodology Group in Manchester involving
Dr Mohammadali Foroozandeh and co-workers (notably Prof.  Gareth Morris and
Prof. Mathias Nilsson) while Dr Foroozandeh was a
Postdoctoral researcher there. The group were interested in generating pure
shift \ac{NMR} spectra from \ac{2DJ} datasets via appropriate post-processing
of the data.  While little progress was made when Dr Foroozandeh was based in
Manchester, he wished to continue with the project after moving to Oxford to
take up a research fellowship; I took the reins of the project when I joined
his nascent research group as a PhD student.

To ensure its applicability to \ac{2DJ} datasets, the following properties were
sought while devising what a suitable estimation routine would entail:

\paragraph{Support for \ac{1D} and \ac{2D} data}
The method should be able to analyse both \ac{1D} \acp{FID} and also
hypercomplex \ac{2D} \acp{FID} (the form which \ac{2DJ} datasets take).
\ac{2D} data should be analysed holistically,
rather than as successive \ac{1D} increments, as is the case in methods like
\ac{CRAFT}~\cite{Krishnamurthy2017} and other previous approaches which will be
discussed later. \correction{
    It is often the case that signals which heavily overlap in a
    \ac{1D} \ac{NMR} dataset are well-separated in a dataset with more dimensions;
    exploiting this concept means that holistic estimation approaches are able to
    resolve and quantify individual signals more effectively.
}\label{corr:1d-vs-2d}

\paragraph{Time-domain based}
As discussed in \cref{subsec:multidim}, due to the hypercomplex nature of
\ac{2DJ} \acp{FID}, generating spectra with desirable absorption lineshapes is
not possible. Typically, resorting to displaying the spectra in magnitude-mode is
deemed optimal, as this overcomes the phase-twist peak lineshapes. However,
such spectra suffer from severe non-linearities and dispersion-mode
contributions, both of which make the task of estimating \ac{2DJ} data in the
Fourier domain challenging. For this reason, estimating the dataset by
considering its \ac{FID} rather than its spectrum is preferred.

\paragraph{Accessibility}
To achieve wide-spread adoption, especially by non-expert \ac{NMR} users,
the method should require minimal user intervention to perform effectively. As
such, a method requiring the specification of as little prior knowledge
about the data as possible is desired.
On top of this, the method should be available as software that users can gain
familiarity with easily.

\subsection{Thesis Overview}
The remainder of this thesis comprises the following chapters:
\begin{itemize}
    \item \textbf{\cref{chap:theory}} discusses the theory behind routines
        which can be applied to determine parameter estimates related to
        \ac{1D} and \ac{2D} \acp{FID}.
    \item \textbf{\cref{chap:results}} provides illustrations of the
        performance of the estimation routine on \ac{1D} \ac{NMR} datasets.
        Furthermore, means in which parametric estimation routine can be
        harnessed for two applications are explored:
        \begin{itemize}
            \item The analysis of amplitude-attenuated datasets, such as those
                derived from diffusion-, $T_1$- and $T_2$-measuring
                experiments (\cref{sec:seq}).
            \item Overcoming quadratic phase behaviour and baseline distortions
                associated with ultra-broadband excitation by a single
                \acl{FS} pulse (\cref{sec:bbqchili}).
        \end{itemize}
    \item \textbf{\cref{chap:cupid}} outlines a devised method, given the acronym
        \acs{CUPID}, for generating pure shift spectra via \ac{2DJ} estimation.
    \item \textbf{\cref{chap:nmrespy}} describes the source code developed as
        part of this project, called \ac{EsPy}.
    \item Finally, conclusions and considerations for further potential
        developments are made in \textbf{\cref{chap:conclusions}}.
\end{itemize}

Additional supporting information can be found in the Appendix:
\begin{itemize}
    % \item \textbf{\cref{chap:nmr-glossary}} provides a glossary of terms
    %     related to \ac{NMR} which are not introduced in much detail in the main
    %     text.
    \item \textbf{\cref{chap:additional-theory}} provides additional
        information on the theory related to this work, including descriptions
        of mathematical concepts, more details on the estimation
        routine, and outlines of relevant algorithms. Descriptions of some
        mathematical concepts which are mentioned but not described in detail
        in the main text are provided in \cref{sec:maths-defs}. These
        concepts are \underline{underlined} the first time they are encountered
        in the main text.
    \item \textbf{\cref{chap:code-listings}} provides code listings outlining
        how the methods described in this work can be implemented in the
        \Python programming language.
        These are effectively minimalist variants of code found in the
        \ac{EsPy} package.
    \item \textbf{\cref{chap:datasets}} outlines how the simulated and
        experimental datasets considered in this work were generated.
    \item \textbf{\cref{chap:inserts}} is a chapter insert from the
        documentation of \ac{EsPy}, comprising tutorials to help users gain
        familiarity with the software.
\end{itemize}

The following publications are related to this work:

\fullcite{Hulse2022}\mybibexclude{Hulse2022}

At the time of writing (\today) a draft for a communication is being being
worked on which describes the \ac{2DJ} estimation work outlined in
\cref{chap:cupid}. It is intended for this to be submitted to
\textit{Angewandte Chemie} or \textit{ChemComm} in the near future.
