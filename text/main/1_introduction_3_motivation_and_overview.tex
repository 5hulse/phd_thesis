\section{Overview of this work}

\subsection{Conception and motivation}
\note{Confirm whether this is accurate with Ali}
The initial motivation for this thesis came from discussions within the NMR
Methodology group in Manchester involving my principal
supervisor Dr. Mohammadali Foroozandeh and coworkers (notably Prof. Gareth
Morris and Prof. Mathias Nilsson) when Dr. Foroozandeh was a Postdoctoral
researcher there. There was an interest in generating pure shift \ac{NMR}
spectra from \acs{2DJ} spectra using an appropriate estimation and
reconstruction algorithm. While little progress was made while Dr. Foroozandeh
was in Manchester, he wished to continue with it after moving to Oxford to take
up a research fellowship. I took on a project with a broader scope
when I joined his nascent research group, starting with developing a routine
and accompanying software for estimating \ac{1D} \acp{FID}. Subsequently, a
number of applications involving \ac{FID} estimation were worked on, including
the original \ac{2DJ} project.

\subsection{Thesis Overview}
This thesis is broken into three principal themes, which span the proceeding
three chapters:
\begin{itemize}
    \item Chapter \ref{chap:theory} discusses the theory behind routines which
        can be applied to determine parameter estimates which describe \ac{1D}
        and \ac{2D} \ac{NMR} datasets.
    \item Chapter \ref{chap:results} provides illustrations of the effectiveness
        of the estimation routine on numerous \ac{NMR} datasets. Furthermore,
        means in which the established estimation routine can be extended for
        useful applications are explored, with these applications being:
        \begin{itemize}
            \item The generation of pure shift spectra with featuring peaks
                with desirable lineshapes from \ac{2DJ} datasets (Section
                \ref{sec:pure-shift}).
            \item Analysis of amplitude-attenuated datasets, such as those
                derived from diffusion and inversion recovery experiments
                (Section \ref{sec:seq}).
            \item Overcoming quadratic phase behaviour and baseline distortions
                associated with excitation by a single \acl{FS} pulse (Section
                \ref{sec:bbqchili}).
        \end{itemize}
    \item Chapter \ref{chap:nmrespy} describes the source code, called
        \acs{EsPy}, which has been written to provide access to the described
        routines presented.
\end{itemize}

