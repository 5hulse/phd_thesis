\section{Overview of this work}

\subsection{Conception and motivation}
The initial motivation for this work emerged from discussions within the NMR
Methodology group in Manchester involving my principal
supervisor Dr. Mohammadali Foroozandeh and coworkers\,---\,notably Prof.
Gareth Morris and Prof. Mathias Nilsson\,---\,while Dr. Foroozandeh was a
Postdoctoral
researcher there. There was an interest in generating pure shift \ac{NMR}
spectra from \acs{2DJ} data using an appropriate estimation and
reconstruction algorithm. While little progress was made when Dr. Foroozandeh
was based in Manchester, he wished to continue with it after moving to Oxford
to take up a research fellowship. I took on a project focussed on this,
though with a broader scope, when I joined his nascent research group as a PhD
student.

\subsection{Thesis Overview}
This thesis is broken into four principal themes, which span the proceeding
four chapters:
\begin{itemize}
    \item \textbf{\cref{chap:theory}} discusses the theory behind routines
        which can be applied to determine parameter estimates which describe
        \ac{1D} and \ac{2D} \ac{NMR} datasets.
    \item \textbf{\cref{chap:results}} provides illustrations of the
        performance of the estimation routine on \ac{1D} \ac{NMR} datasets.
        Furthermore, means in which the established estimation routine can be
        extended for two applications are explored:
        \begin{itemize}
            \item Analysis of amplitude-attenuated datasets, such as those
                derived from diffusion and inversion recovery experiments
                (\cref{sec:seq}).
            \item Overcoming quadratic phase behaviour and baseline distortions
                associated with ultra-broadband excitation by a single
                \acl{FS} pulse (\cref{sec:bbqchili}).
        \end{itemize}
    \item \textbf{\cref{chap:cupid}} outlines a method for generating pure
        shift spectra and deriving the associated multiplet structures from
        \ac{2DJ} datasets. As described above, this was the initial motivation
        for embarking on this project.
    \item \textbf{\cref{chap:nmrespy}} describes the source code developed as
        part of this project, called \acs{EsPy}.
    \item Finally, conclusions and considerations for further potential
        developments are discussed in \textbf{\cref{chap:conclusions}}.
\end{itemize}

Further supporting information can be found in the Appendix, which comprises
the following chapters:
\begin{itemize}
    \item \textbf{\cref{chap:nmr-glossary}} provides a glossary of terms
        related to \ac{NMR} which are not introduced in much detail in the main
        text.
    \item \textbf{\cref{chap:additional-theory}} provides additional
        information on the theory related to this work, including descriptions
        of mathematical concepts, and outlines of relevant algorithms.
    \item \textbf{\cref{chap:code-listings}} provides code listings outlining
        how the methods described in this work can be implemented in \Python.
        These are effectively reduced versions of code found in the \ac{EsPy}
        package.
    \item \textbf{\cref{chap:datasets}} outlines how the simulated and
        experimental datasets considered in this work were generated.
    \item \textbf{\cref{chap:walkthrough}} is an insert from the documentation
        of \ac{EsPy}, comprising tutorials on getting familiarised with using
        the package.
\end{itemize}
