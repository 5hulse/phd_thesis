\section{Overview of this work}

\subsection{Conception and motivation}
The central focus of this work is the development of a routine which performs
parametric estimation on \ac{NMR} datasets.
Motivation initially came from discussions within the NMR
Methodology Group in Manchester involving
Dr Mohammadali Foroozandeh and co-workers\,---\,notably Prof.
Gareth Morris and Prof. Mathias Nilsson\,---\,while Dr Foroozandeh was a
Postdoctoral researcher there. There was an interest in generating pure shift
\ac{NMR} spectra from \ac{2DJ} datasets via appropriate post-processing of the
data.  While little progress was made when Dr Foroozandeh was based in
Manchester, he wished to continue with the project after moving to Oxford to
take up a research fellowship, which I took the reins of when I joined his
nascent research group as a PhD student.

To ensure its applicability to \ac{2DJ} datasets, the following properties were
sought when devising what the estimation routine would entail:

\paragraph{Support for \ac{1D} and \ac{2D} data}
The method should be able to analyse both \ac{1D} \acp{FID} and also
hypercomplex \ac{2D} \acp{FID} (the form which \ac{2DJ} datasets take).
\ac{2D} data should be analysed \emph{holistically},
rather than as successive \ac{1D} increments, as is the case in methods like
\ac{CRAFT}\cite{Krishnamurthy2017}. This is since
better signal resolution is often available when both dimensions are
considered at the same time; certain signals which exhibit clear resolution in a
\ac{2D} dataset may be heavily overlapping in \ac{1D} data and are
therefore more challenging if not impossible to quantify accurately.

\paragraph{Time-domain based}
As discussed in \cref{subsec:multidim}, due to the hypercomplex nature of
\ac{2DJ} \acp{FID}, generating spectra with desirable absorption lineshapes is
not possible. Typically, resorting to displaying the spectra in magnitude-mode is
deemed optimal, as this overcomes the phase-twist peak lineshapes. Such
spectra suffer from gross non-linearities and dispersion-mode contributions,
both of which make the task of estimating \ac{2DJ} data in the Fourier domain
challenging. For this reason, estimating the dataset by considering its \ac{FID}
rather than its spectrum is preferred.

\paragraph{Accessibility}
To achieve wide-spread adoption, especially by non-expert \ac{NMR} users,
the method should require minimal user intervention to perform effectively. As
such, a method requiring the specification of as little prior knowledge
about the data as possible is desired.
On top of this, the method should be available as software that users can gain
familiarity with easily.

\subsection{Thesis Overview}
This thesis is broken into the following chapters:
\begin{itemize}
    \item \textbf{\cref{chap:theory}} discusses the theory behind routines
        which can be applied to determine parameter estimates related to
        \ac{1D} and \ac{2D} \ac{NMR} datasets.
    \item \textbf{\cref{chap:results}} provides illustrations of the
        performance of the estimation routine on \ac{1D} \ac{NMR} datasets.
        Furthermore, means in which parametric estimation routine can be
        harnessed for two applications are explored:
        \begin{itemize}
            \item The analysis of amplitude-attenuated datasets, such as those
                derived from diffusion and inversion recovery experiments
                (\cref{sec:seq}).
            \item Overcoming quadratic phase behaviour and baseline distortions
                associated with ultra-broadband excitation by a single
                \acl{FS} pulse (\cref{sec:bbqchili}).
        \end{itemize}
    \item \textbf{\cref{chap:cupid}} outlines the devised method, called
        \ac{CUPID}, for generating pure shift spectra from \ac{2DJ} datasets.
    \item \textbf{\cref{chap:nmrespy}} describes the source code developed as
        part of this project: \ac{EsPy}.
    \item Finally, conclusions and considerations for further potential
        developments are discussed in \textbf{\cref{chap:conclusions}}.
\end{itemize}

Additional supporting information can be found in the appendix, comprising:
\begin{itemize}
    \item \textbf{\cref{chap:nmr-glossary}} provides a glossary of terms
        related to \ac{NMR} which are not introduced in much detail in the main
        text.
    \item \textbf{\cref{chap:additional-theory}} provides additional
        information on the theory related to this work, including descriptions
        of mathematical concepts, and outlines of relevant algorithms.
    \item \textbf{\cref{chap:code-listings}} provides code listings outlining
        how the methods described in this work can be implemented in the
        \Python programming language.
        These are effectively bare-bones variants of code found in the
        \ac{EsPy} package.
    \item \textbf{\cref{chap:datasets}} outlines how the simulated and
        experimental datasets considered in this work were generated.
    \item \textbf{\cref{chap:walkthrough}} is an insert from the documentation
        of \ac{EsPy}, comprising tutorials to help users get started with the
        package.
\end{itemize}

The following publication\note{s} is\note{are} related to this work:

\fullcite{Hulse2022}

\note{Mention status of 2DJ paper at point of submission}
