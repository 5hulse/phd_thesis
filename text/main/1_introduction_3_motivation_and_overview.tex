\section{Overview of this work}

\subsection{Conception and motivation}
\note{Confirm whether this is accurate with Ali}
The initial motivation for this thesis came from discussions within the NMR
Methodology group in Manchester involving my principal
supervisor Dr. Mohammadali Foroozandeh and coworkers (notably Prof. Gareth
Morris and Prof. Mathias Nilsson) when Dr. Foroozandeh was a Postdoctoral
researcher there. There was an interest in generating pure shift \ac{NMR}
spectra from \acs{2DJ} spectra using an appropriate estimation and
reconstruction algorithm. While little progress was made while Dr. Foroozandeh
was in Manchester, he wished to continue with it after moving to Oxford to take
up a research fellowship. I took on a project with a broader scope
when I joined his nascent research group, starting with developing a routine
and accompanying software for estimating \ac{1D} \acp{FID}. Subsequently, a
number of applications involving \ac{FID} estimation were worked on, including
the original \ac{2DJ} project.

\subsection{Thesis Overview}
This thesis is broken into four principal themes, which span the proceeding
four chapters:
\begin{itemize}
    \item Chapter \ref{chap:theory} discusses the theory behind routines which
        can be applied to determine parameter estimates which describe \ac{1D}
        and \ac{2D} \ac{NMR} datasets.
    \item Chapter \ref{chap:results} provides illustrations of the performance
        of the estimation routine on \ac{1D} \ac{NMR} datasets. Furthermore,
        means in which the established estimation routine can be extended for
        two useful applications are explored:
        \begin{itemize}
            \item Analysis of amplitude-attenuated datasets, such as those
                derived from diffusion and inversion recovery experiments
                (\S \ref{sec:seq}).
            \item Overcoming quadratic phase behaviour and baseline distortions
                associated with excitation by a single \acl{FS} pulse (\S
                \ref{sec:bbqchili}).
        \end{itemize}
    \item Chapter \ref{chap:cupid} outlines a method for generating pure shift
        spectra and deriving the associated multiplet structures from \ac{2DJ}
        datasets. As described above, this was the central motivation for
        embarking on this project.
    \item Chapter \ref{chap:nmrespy} describes the source code developed as
        part of this project, called \acs{EsPy}.
\end{itemize}

Further supporting information can be found in the Appendix, which comprises
the following chapters:
\begin{itemize}
    \item Additional theory, including descriptions of mathematical concepts,
        and outlines of relevant algorithms, is presented in Chapter
        \ref{chap:additional-theory}.
    \item Chapter \ref{chap:code-listings} provides code listings outlining how
        the methods described in this work can be implemented in \Python.
    \item Chapter \ref{chap:datasets} outlines how the simulated and
        experimental datasets considered were generated.
    \item Chapter \ref{chap:walkthrough} is an insert from the documentation of
        \ac{EsPy}, comprising tutorials on getting familiarised with using the
        package.
\end{itemize}
