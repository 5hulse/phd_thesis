\section{An Overview of NMR Data Analysis}
To gain insights from \ac{NMR} experiments on the chemical system of interest,
extraction of the defining parameters $\bth$ is necessary, though the majority
of \ac{NMR} users are unlikely to think about the process of \ac{NMR} analysis
in this way. For example:
\begin{itemize}
    \item An understanding of the chemical environments of atoms in a molecule
        can be gained by considering the chemical shifts of the various peaks
        in the spectra, which are a proxy for the \ac{FID} frequencies
        $\symbf{f}$.
    \item  The relative stoichiometries\note{does this make sense?} of a
        molecule can be elucidated by inspecting the integrals of spectral
        peaks, which are directly related to the \ac{FID} amplitudes $\bda$
\end{itemize}

\note{
    \begin{itemize}
        \item Structure of NMR data
        \item Typical approach to analysing the data (FT, peak pick, integrate, baseline correction, window functions, zero-filling etc),
        \item Estimation techniques: LP, SVD techniques, iterative techniques (AMARES, VARPRO), Bayesian techniques (CRAFT), ML techniques
    \end{itemize}
}

\subsection{Conventional NMR Analysis}

\subsection{Estimation Techniques for NMR Analysis}
