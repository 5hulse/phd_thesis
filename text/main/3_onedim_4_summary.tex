\section{Summary}

In this chapter, numerous results have been presented, all of which feature
\ac{1D} \ac{FID} estimation.
In \cref{sec:evaluation}, it was illustrated that extending the \ac{MPM} through the
incorporation of the added step of \ac{NLP} with a phase variance constraint can
aid in the generation of parameter estimates which agree with the underlying
assumptions of the data structure. Oscillators in the \ac{MPM} often possess
spurious phases; this feature can frequently be rectified though the \ac{NLP}
method. Beyond simply providing a description of the signals which make up
a dataset, it has been shown that parametric estimation can be applied to
further useful ends; two such examples have been presented in
\cref{sec:seq,sec:bbqchili}, respectively.

While there are scenarios in which \ac{1D} estimation can perform admirably,
enhancing the information accessible to a spectroscopist, ironically \ac{1D}
data is probably the most challenging type of \ac{NMR} data to estimate from a
complexity perspective; with all information about a sample being
distilled into a single dimension, it is very common (particularly with
\ch{^{1}H} datasets) for the data to be so densely
populated with signals that extracting meaningful information at
the per-signal level is futile. It should therefore be appreciated that
there is a limited scope where estimation methods which rely minimally on
user-provided information can find use. Through the separation of signals into
more than one detection dimension, multidimensional experiments can yield
datasets which have sufficiently well-resolved signals for estimation to be
applicable. The \ac{2DJ} experiment is such an example, and estimation of
datasets acquired using it is the focus of the next chapter.
