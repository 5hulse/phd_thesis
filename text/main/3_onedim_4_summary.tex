\section{Summary}

In this chapter, numerous results have been presented, all of which feature
\ac{1D} \ac{FID} estimation either as the sole focus (\cref{sec:evaluation})
or as a key step in a specific application (\cref{sec:seq} \&
\cref{sec:bbqchili}).

In \cref{sec:evaluation}, it was illustrated that by extending the \ac{MPM} through the
incorporation of the added step of \ac{NLP} with a phase variance constraint can
aid in the generation of parameter estimates which agree with the underlying
assumptions of the data structure. Oscillators in the \ac{MPM} often possess
spurious phases; this feature can frequently be rectified though the \ac{NLP}
method.

While there are scenarios where \ac{1D} estimation performs admirably, and can
enhance the information accessible to a spectroscopist, ironically \ac{1D} data
is probably the most challenging type of \ac{NMR} data to estimate from a
complexity perspective; with all information about a sample being
distilled into a single dimension, it is very common (particularly with
\ch{^{1}H} datasets) for the data to be so densely
populated with signals that extracting meaningful information from it at
the per-signal level is futile. It should therefore be appreciated that
there is a limited scope where estimation methods which rely minimally on
user-provided information can find use. Through the separation of signals into
more than one detection dimension, multidimensional experiments can yield
datasets which have sufficiently well-resolved signals for estimation to be
applicable. The \ac{2DJ} experiment is such an example, and estimation of
datasets acquired from it is the focus of the next chapter.
