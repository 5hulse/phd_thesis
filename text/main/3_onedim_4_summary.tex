\section{Summary}

In this chapter, numerous results have been presented, all of which feature
\ac{1D} \ac{FID} estimation.
In \cref{sec:evaluation}, it was illustrated that extending the \ac{MPM}, with
the added step of a phase variance-regularised \ac{NLP} routine, can
aid in the generation of parameter estimates which agree better with the
underlying assumptions of the data structure. Oscillators in the \ac{MPM} often
possess spurious phases; this feature can frequently be rectified by the
\ac{NLP} method. Beyond simply providing a description of the signals which
make up a dataset, it has been shown that parametric estimation can be applied
to further useful ends; two such examples have been presented in
\cref{sec:seq,sec:bbqchili}, respectively.

There are scenarios in which \ac{1D} estimation can perform admirably,
enhancing the information accessible to a spectroscopist. However,
\ac{1D} datasets are typically the most challenging form of \ac{NMR} data to
estimate due to their inherent complexity; with all information about a
sample being distilled into a single dimension, it is very common
(particularly with \ch{^{1}H} datasets) for \acp{FID} to be so densely
populated with signals that extracting meaningful information at
the per-signal level is futile.
\correction{
    It should therefore be appreciated that estimation methods such as the one
    presented in this work\,---\,requiring little to no prior knowledge about
    the dataset to operate\,---\,can only be applied effectively to a limited
    set of cases.
}\label{corr:limited-scope}

Through the separation of signals into more than one detection dimension,
multidimensional experiments can yield datasets which have sufficiently
well-resolved signals for estimation to be applicable to a wider range of
samples. The \ac{2DJ} experiment is such an example, and the estimation of
datasets acquired using it is the focus of the next chapter.
