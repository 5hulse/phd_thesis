\chapter{Conclusions and Future Work}
\label{chap:conclusions}

\section{Conclusions}

\section{Future Work}
\label{sec:future-work}
Any future work related to this project would be centered around extending and
improving the \ac{EsPy} package.

\paragraph{Improving performance}
Improvements in the speed of the estimation
could be realised if the most computationally demanding parts of it were
written in a low-level compiled language like C++ or \textsc{Rust}. The may not
be the case with the \ac{MPM}; the majority of time running these routines
involves \ac{SVD}, with \textsc{NumPy}'s implementation calling well-optimised
routines from the \textsc{LAPACK} and \textsc{ARPACK} software packages.
However, in its current state, the \ac{NLP} routine could well be sped up
considerably if the current \Python implementation were ported to one of the
aforementioned languages.

\paragraph{A more general platform}
Thinking further afield from the specific estimation routine considered in this
work, the \ac{EsPy} package provides a large number of useful features which
would be applicable to the estimation using any conceived method. The includes
functionality to import data, pre-process data, and inspect the result of an
estimation routine through parameter tables and figures.
Incorporating other estimation routines into the software would therefore be
rather straightforward, such that \ac{EsPy} could become a general-purpose
platform for comparing different methods.

\paragraph{\ac{2D} and \ac{3D} datasets?}
Another potential future venture would be to extend to routine for the
consideration of other \ac{2D} datasets, such as those that comprise
amplitude- or phase-modulated pairings, as well as \ac{3D} datasets,
including many of the ``triple-resonance'' experiments popular in the
biological \ac{NMR} community\cite[Section 7.4]{Cavanagh2007}. As shown in
\cref{chap:cupid}, the method is capable of performing well on \ac{2D}
datasets. In order to achieve \ac{3D} estimation, it would be necessary to
make use of a \ac{3D} equivalent of the \ac{MMEMPM}. Such a technique exists,
and is used principally for applications in which the direction (i.e. azimuth
and elevation angles) as well as the frequency of waves arriving at an antenna
are sought\cite{Yilmazer2006}. However, there is a key issue with
multidimensional \ac{NMR} signals which would likely hamper the proposed method
in this thesis from being effective. \ac{2DJ} datasets are rather anomalous in
that they have very densely populated indirect dimensions. For this reason, it
was not necessary to be concerned about filtering the data in the indirect
dimension. However, for most \ac{2D} datasets, it would be desirable to filter
the data in both dimensions. While the filtering procedure presented works well
on direct-dimension \acp{FID}, which have usually decayed to noise at the end
of acquisition, indirect dimension \acp{FID} are often truncated. The \ac{FT}
of such signals without any treatment would produce spectra with considerable
truncation artefacts. Filtering such a spectrum would incorporate undesirable
artefacts into the final filtered \ac{FID}, rendering it worthless.
Compensating for the truncation could be achieved, by applying exponential
apodisation prior to \ac{FT}. However, this would then lead to spectra with
broader lineshapes, such that wider spectral regions need to be selected to
ensure practically all of the peaks of interest lie within it. Dealing with
truncated signals could possibly be handled in one of two ways, beyond running
the experiment to acquire the data with a huge number of increments. One
possibility could be to propagate the \ac{FID} further in time using \ac{LP}.
Another option might be to apply conventional apodisation, such as a sine-bell
function to the \ac{FID}, producing a \ac{FID} without truncation artefact, and
with acceptable resolution. Applying a non-exponential weighting to the dataset
would however render the data incompatible with the estimation routine in its
current form, such that a modified routine with a suitable model would need to
be implemented. For \ac{3D} estimation, the computational burden required would
also likely be too high for the routine to be practicable. \note{Once 2D
profiling has been done, perhaps this could be used to guestimate typical 3D
running times?}

\paragraph{Incorporating constraints into the \ac{NLP} routine}
