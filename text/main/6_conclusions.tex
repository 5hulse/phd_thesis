\chapter{Conclusions and Future Work}
\label{chap:conclusions}

\section{Conclusions}
This work has focussed on the use of parametric estimation as a means of
extracting quantitative information from \ac{NMR} \acp{FID}. The initial
motivation can from can interest in developing a technique
for quantifying \ac{2DJ} datasets, facilitating the generation of broadband
homodecoupled \ac{NMR} spectra with desirable characteristics.
To achieve this,
the \ac{MMEMPM}, a \ac{2D} extension of the original \ac{MPM}, was proposed as
a suitable technique for parametrising \ac{2DJ} datasets in a holistic fashion.
However, initial evaluations of estimation results generated by the \ac{MPM} on
\ac{1D} \acp{FID} indicated that a particular flaw was common: the
phases of individual oscillators were often inconsistent with their expected
behaviour.
For this reason, a routine was developed which incorporated the result of the
\ac{MPM} as an initial guess of an \ac{NLP} algorithm. The \ac{NLP} algorithm
included the variance of oscillator phases in the fidelity to be minimised as a
means of constraining the phases to adopt similar values.
Through a number of examples, it has been shown that improved parameter
estimates can be obtained by including this \ac{NLP} routine.

To alleviate the high computational burden of the proposed method\,---\,the
most noteworthy aspects being computing the \ac{SVD} of large matrices and
computing large numbers of derivatives to generate Hessian matrices\,---\,a
frequency-filtration technique has been incorporated, which produces
sub-\acp{FID} with reduced numbers of both constituent signals and datapoints.
Drastic reductions in both \ac{CPU} work and \ac{RAM} usage are possible by
doing this. Furthermore, it enables the user to neglect spectral regions that
are not of interest/too complex to realistically yield meaningful information
from.

Impressive parametrisations of hypercomplex \ac{2DJ} \acp{FID} were realised
with the estimation routine, leading to \ac{CUPID}, which is able generate pure
shift spectra without the loss of signal, and featuring peaks with sharp
Lorentzian lineshapes.
Furthermore, \ac{CUPID} is able to provide further insights, such as harnessing
the information from \ac{2DJ} estimation to decompose \ac{1D} spectra into
their constituent multiplet structures. Acquiring the information that
\ac{CUPID} can generate with a simple \ac{2DJ} dataset usually requires
bespoke \ac{3D} pulse sequences with a very large associated run-times.

Further to the original goal of \ac{2DJ} estimation, two other applications
have been developed and presented here. The first is an alternative means for
determining valuable information about molecular species ($T_1$ and  $T_2$
times of spins, and diffusion coefficients) via datasets which comprise a series
of \acp{FID} whose signal amplitudes vary. Whereas typical univariate
approaches to analysing these datasets involve fitting single datapoints in
frequency space, the method showcased instead aims to gleam the
information for each signal explicitly, to overcome the aggregating effect
frequently encountered due to signal overlap. While the method works admirably,
it couldn't be shown to perform noticeably better than other techniques; in
cases where there is severe signal crowding, the difficulty/impossibility of
resolving each component leads to similar issues of aggregating signals.

A second application featuring \ac{1D} estimation involves generating broadband
\ac{NMR} spectra from an experiment involving excitation by a single chirp
pulse. \note{More here}

The software package \ac{EsPy} has been developed to facilitate use of the methods
described in this thesis. The package's \ac{API} follows a straightforward
object-oriented paradigm, which should make it easily accessible to all
members of the \ac{NMR} community that have experience with scientific
computing with \Python\!. Furthermore, the availability of an accompanying \ac{GUI}
widens to accessibility of the package further, meaning \ac{EsPy} can be used
without writing a single line of code. The package has been heavily documented
too; this is a crucial aspect of any piece of user-facing software which seems
to often be lacking, especially when it comes to the field of academia.

\note{1D estimation hard: lots of overlap, 2D/3D probably more interesting: notable issues (future work)}

%%% Final bit
It has hopefully been shown that attempting to solve the inverse problem of
\ac{NMR} data estimation is a valuable pursuit; it provides richer information
that that which is typically available with the conventional \ac{FT} method,
and it opens up many fruitful applications which can enhance the value that
\ac{NMR} can offer to practitioners in many scientific disciplines.


\section{Future Work}
\label{sec:future-work}
Any future work related to this project would be centered around extending and
improving the \ac{EsPy} package. Some possible pursuits include the following:

\paragraph{Improving performance}
Improvements in the speed of the estimation
could be realised if the most computationally demanding parts of it were
written in a low-level compiled language like C++ or \textsc{Rust}. This may not
be the case for the \ac{MPM} and \ac{MMEMPM}; the majority of time running
these routines involves \ac{SVD}, with \textsc{NumPy}'s implementation calling
well-optimised routines from the \textsc{LAPACK} and \textsc{ARPACK} software
packages. However, in its current state, the \ac{NLP} routine could well be
sped up considerably if the current \Python implementation were ported to one
of the aforementioned languages.

\paragraph{A more general platform}
Thinking further afield from the specific estimation routine considered in this
work, the \ac{EsPy} package provides a large number of useful features which
would be applicable to the estimation using any conceived method. The includes
functionality to import data, pre-process data, and inspect the result of an
estimation routine through parameter tables and figures.
Incorporating other estimation routines into the software would therefore be
rather straightforward, such that \ac{EsPy} could become a general-purpose
platform for comparing different methods.

\paragraph{\ac{2D} and \ac{3D} datasets}
Another potential future venture would be to extend to routine for the
consideration of other \ac{2D} datasets, such as those that comprise
amplitude- or phase-modulated pairings, as well as \ac{3D} datasets,
including those acquired from ``triple-resonance'' experiments which are
popular in the biological \ac{NMR} community\cite[Section 7.4]{Cavanagh2007}.
As shown in
\cref{chap:cupid}, the method is capable of performing well on hypercomplex
\ac{2D} datasets. In order to achieve \ac{3D} estimation, it would be necessary to
make use of a \ac{3D} equivalent of the \ac{MMEMPM}. Such a technique exists,
and is used principally for sensing applications, in which the direction (i.e.
azimuth and elevation angles) as well as the frequency of waves arriving at an
antenna are sought\cite{Yilmazer2006}. However, there is a key issue with
multidimensional \ac{NMR} signals which would likely hamper the method outlined
in this thesis from being effective. \ac{2DJ} datasets are rather anomalous in
that they have very densely populated indirect dimensions. For this reason, it
is not necessary to be concerned about filtering the data in the indirect
dimension. However, for most \ac{2D} datasets, it would be desirable to filter
the data in both dimensions, as both will generally be sparsely populated.
While the filtering procedure presented works well on direct-dimension
\acp{FID}, which have usually decayed to noise at the end
of acquisition, indirect-dimension \acp{FID} are often truncated. The \ac{FT}
of such signals without any treatment would produce spectra with considerable
truncation artefacts; filtering such spectra would incorporate undesirable
artefacts into a filtered sub-\ac{FID}, rendering it worthless.
Compensating for the truncation could be achieved, by applying exponential
apodisation prior to \ac{FT}. However, this would then lead to spectra with
broader lineshapes, such that wider spectral regions would need to be selected,
and the resolution of the data would be decreased.
Dealing with
truncated signals could possibly be handled in one of two ways, beyond running
the experiment to acquire the data with a huge number of increments. One
option would be to propagate the \ac{FID} further in time using \ac{LP}.
Another would be to apply conventional apodisation, such as a sine-bell
function to the \ac{FID}, producing a \ac{FID} without truncation artefacts and
acceptable resolution. However, applying a non-exponential weighting to the
dataset however render the data incompatible with the estimation routine in its
current form, such that a modified routine with a suitable model would need to
be implemented. On to of issues with data applicability, for \ac{3D}
estimation the computational burden required would also likely be too high for
the routine to be practicable beyond the simplest datasets with few points and
minimal numbers of signals.

\paragraph{Incorporating constraints into the \ac{NLP} routine}
The estimation routine in its current form is designed to require as little
user input as possible while producing faithful parameter estimates. However,
in particularly challenging circumstances, most notably when
there is extreme signal overlap, the inclusion of oscillator phase variance
alone can be insufficient to produce estimates which closely agree with the
experiment. When this is the case, it would be useful to enable users to
specify additional knowledge, incorporated into the routine as additional
regularising terms, \emph{after} the initial estimation has been performed.
This concept, while similar to \ac{VARPRO} and \ac{AMARES}, differs in that
these require very large quantities or prior knowledge \emph{before}
estimation. A lot of the heavy lifting towards an accurate parameter estimate
could be achieved using phase variance-regularised \ac{NLP}, with the user
indicating erroneous features subsequently.

Take the cyclosporin A result (\cref{fig:cyclosporin}) as an example. As
discussed already, due to severe signal overlap, the assigned oscillators
related to spin (E), while at the correct frequencies, do not have appropriate
amplitudes for a dq multiplet structure. As such, re-running \ac{NLP}, with a
new fidelity of
\[
    \FphithY + \Var(
        a_{\text{E}1},
        \tfrac{1}{3} a_{\text{E}2},
        a_{\text{E}3},
        \tfrac{1}{3} a_{\text{E}4},
        \tfrac{1}{3} a_{\text{E}5},
        a_{\text{E}6},
        \tfrac{1}{3} a_{\text{E}7},
        a_{\text{E}8}
    )
\]
would likely lead to an improved result.
$\lbrace \text{E}1, \cdots, \text{E}8 \rbrace \subset \lbrace 1, \cdots, M \rbrace$ are the
indices of the oscillators corresponding to spin (E), i.e. the green
oscillators in the figure.
