\section{Pure Shift \acs{NMR}}

In this section, a survey of some of the most prominent procedures for
producing pure shift spectra are presented.

\subsection{The \acs{2DJ} Experiment}
The \ac{2DJ} experiment\cite{Aue1976, Morris2009} provided the first means of
achieving pure shift spectra. It has a simple pulse sequence:
\[
    \ang{90} \xrightarrow{\nicefrac{\tone}{2}} \ang{180} \xrightarrow{\nicefrac{\tone}{2}} \ttwo.
\]
After excitation of magnetisation onto the transverse plane, the indirect
dimension evolution consists of a spin echo, with acquisition following
immediately afterwards. Fourier transformation in both dimensions leads to a
spectrum in which only scalar couplings contribute in $\Fone$, as the chemical
shifts are refocussed by the spin echo, while both scalar couplings and
chemical shifts contribute in $\Ftwo$.  \Iac{FID} generated by the \ac{2DJ}
experiment is hypercomplex, taking the form of \cref{eq:general-fid} with
$D=2$ and $\zeta^{(1)} = \exp(\iu\cdot)$, i.e.
\begin{equation}%
    \begin{split}%
        y_{\none,\ntwo} =
        &\sum_{m=1}^{M} a_m \exp(\iu \phi_m)
            \exp\left(\left(2 \pi \iu \fonem - \etaonem\right) \none \Dtone\right) \times \\
        &\exp\left(\left(2 \pi \iu  \left(\ftwom - \foff\right)
            - \etatwom\right) \ntwo \Dttwo\right)
            + w_{\none,\ntwo}.
    \end{split}%
    \label{eq:jres-fid}
\end{equation}%
The transmitter offset term has been neglected in the indirect dimension, since
chemical shift evolution does not occur.
For each signal in the \ac{FID}, the indirect- and direct-dimension
frequencies are intimately linked. Consider a \ac{2DJ} dataset generated by a
spin system with $S$ distinct spins. The signals giving rise to a particular
spin $s \in \lbrace 1, \cdots, S \rbrace$ form a grouping $G_s
\subset \lbrace 1, \cdots, M \rbrace$. All of the signals in $G_s$
have (angular) frequencies given by
\begin{subequations}
    \begin{gather}
        2 \pi \fonem = \Updelta \omega_m,\\
        2 \pi \ftwom = \omega_{0,s} + \Updelta \omega_m,
    \end{gather}
    \label{eq:f1-f2-2dj}
\end{subequations}
$\forall m \in G_s$, where $\omega_{0,s}$ is the Larmor frequency of
the spin, and $\Updelta \omega_m$ is the displacement
of the signal from $\omega_{0,s}$, as a result of J-couplings\footnote{
    $\Updelta \omega_m$ will be a linear combination of all the scalar
    couplings associated with the spin giving rise to the signal, with all the
    coefficients being $\pm \nicefrac{1}{2}$.
}. Due to the relationship between the direct- and indirect-dimension
frequencies, all signals which are part of the same multiplet lie along
a line which bisects (i.e. makes a \ang{45} angle with) both the $\Fone$ and
$\Ftwo$ axes, as depicted in \cref{fig:jres_spectrum}.a.
\begin{figure}%
    \centering%
    \includegraphics{jres_spectrum/jres_spectrum_new.pdf}%
    \caption[
        Region of a \acs{2DJ} spectrum of strychnine.
    ]
    {%
        Region of a simulated \acs{2DJ} spectrum of strychnine.
        Each panel depicts the spectrum following different processing
        procedures. Below: contour plots of the spectrum. Above: the
        summation of the spectrum along the indirect ($y$) axis.
        \textbf{a.} Spectrum produced by applying sine-bell apodisation
        followed by \ac{FT} in both dimensions.
        Coloured lines denote \ang{45} cross-sections along which the present
        multiplet structures lie.
        \textbf{b.} Magnitude-mode spectrum.
        \textbf{c.} Spectrum generated after application of a \ang{45} shear on
        the magnitude-mode spectrum. Peaks marked with asterisks panel c arise
        from the presence of strong coupling artefacts.
   }%
    \label{fig:jres_spectrum}%
\end{figure}%

One limitation of the \ac{2DJ} experiment is the fact that
spectra with pure absorption lineshapes cannot be produced. This is since, due
to the absence of a mixing period, it is not possible to produce a
complementary pair of phase- or amplitude-modulated \acp{FID}, which are
required to nullify dispersive contributions (see
\cref{subsec:mulitdim}).
The FT of a \ac{2DJ} \ac{FID} produces a spectrum with phase-twist
peaks (\cref{fig:jres_spectrum}.a). As with other experiments which
produce hypercomplex signals, such as \ac{COSY}, the data is conventionally
displayed in ``magnitude-mode'' (\cref{fig:jres_spectrum}.b) in which the
absolute value of each point in the spectrum is plotted.
A pure shift spectrum is generated from the \ac{2DJ} spectrum by performing a
\ang{45} shear\,---\,often referred to as a tilt\,---\,on the spectrum array,
leading to the separation of chemical shifts and scalar couplings onto
orthogonal axes (\cref{fig:jres_spectrum}.c). Each slice through the
direct dimension of the \ac{2DJ} spectrum is subjected to a right circular
rotation such that
\begin{subequations}
    \begin{gather}
        s_{\none,\ntwo}^{\text{tilt}} =
            s_{\none,n^{(2)\prime}},\\
        n^{(2)\prime} = \left(\ntwo + \left\lfloor
                \frac
                    {\fswone \Ntwo\vpsub{\mathrm{sw}}}
                    {\fswtwo \None\vpsub{\mathrm{sw}}}
                \left(
                    \frac{\None\vpsub{\mathrm{sw}}}{2} - \none
                \right)
            \right\rceil
        \right) \bmod \Ntwo.
    \end{gather}
\end{subequations}
This achieves the mapping $s\left(\Fone,\Ftwo\right) \rightarrow s(\Fone, \Ftwo
- \Fone)$, which leads to a spectrum in which all peaks arising from a
given spin reside at the same direct-dimension frequency. The
effectiveness of the shear is maximised when both $\nicefrac{\fswtwo}{\fswone}$
and $\nicefrac{\Ntwo}{\None}$ are powers of 2\note{check this}. Summing the
sheared spectrum along $\Fone$ leads to the pure shift spectrum.
If the spectrum wasn't in magnitude-mode, shearing and summing would lead to
the absorptive and dispersive components of the spectrum cancelling each other
out, such that a vector of noise would be obtained.
With a magnitude-mode spectrum, the process leads to undesirable pure shift
spectra with broad ``wings'' on account of the presence of dispersive
character, and non-linearities. These effects can be suppressed by appropriate
processing to make the FID envelope symmetric in both dimensions, such as with
sine-bell apodisation or pseudo-echo reshaping\cite{Bax1981}, though this
results in a significant reduction in sensitivity being incurred, along with
distortions in relative peak amplitudes.

Another feature which limits the effectiveness of the \ac{2DJ} experiment to
produce pure shift spectra are
\emph{strong coupling artefacts}\footnote{
    As stressed in \cite{Thrippleton2005}, these are not strictly artefacts,
    but rather genuine signals, which are expected to be present in the
    \ac{2DJ} dataset. Despite this, the term is widespread in the literature.
},
which arise due to mixing effects induced by the \ang{180} pulse in the
\ac{2DJ} sequence\cite{Wider1983,Thrippleton2005}. Examples of these are seen
in the spectra of \cref{fig:jres_spectrum}, on account of the three spins
giving rise to the signals seen being strongly coupled. These artefacts always
have direct-dimension frequencies which match those of the conventional signals
in the spectrum\,---\,a feature which will be exploited in the \ac{CUPID}
procedure\,---\,however they do not lie along the same \ang{45} cross
sections. As a result, the final spectrum produced by shearing and summing will
feature extra low intensity signals that do not agree with the chemical shift
of a particular spin (see peaks marked with asterisks in
\cref{fig:jres_spectrum}.c).

\subsection{The \acl{ZS} Method}
\label{subsec:ZS}
Zangger and Sterk introduced a pulse sequence element which achieves
\emph{slice-selective excitation}, by applying a low \ac{RF} power \ang{180}
pulse\footnote{Conventionally, a R-SNOB pulse is used\cite{Kupce1995}.} in the
presence of a \ac{PFG} along the $z$-axis\cite{Zangger1997}. Such an element
excites a
given spin only in a narrow range of heights in the sample, as the \ac{PFG}
induces a shift in resonance frequency according to $\Updelta \omega(z) = \gamma
gz$, where $g$ is the magnitude of the \ac{PFG}. By placing a hard
\ang{180} pulse adjacent to the selective pulse, the
``active'' spin in a given slice is rotated by \ang{360} (i.e. no net
rotation), while all other (``passive'') spins are only rotated by \ang{180}.
Placing such a element in the middle of the $\tone$ evolution therefore
achieves refocussing of the J-couplings associated with the active
spin\cite{Aguilar2010}. In order to achieve effective decoupling of any given
pair of spins, it is necessary that the bandwidth of the selective π-pulse is
smaller than the difference in their Larmor frequencies. However, with more
selective pulses, a smaller proportion of the available spin magnetisation will
contribute to the final FID, and hence sensitivity will be diminished
\footnote{
    The reduction in sensitivity is $\propto \nicefrac{\Updelta F}{\gamma g l_z}$,
    where $\Updelta F$ is the selective pulse bandwidth, and $l_z$ is the length of
    the sample lying within the receiver coil ($\approx
    \qty{1.5}{\centi\meter}$).
}.
Therefore a trade-off exists between effective decoupling of all spins, and
achieving the greatest sensitivity possible. In the case of strong coupling,
the \ac{ZS} method tends to perform poorly relative to other options for this
reason. The \ac{ZS} element has been utilised in order to generate \ac{2DJ}
datasets comprising phase-modulated pairs, enabling the generation of pure
absorption-mode spectra\cite{Pell2007}. Pure shift spectra with far more
desirable lineshapes can be achieved relative to using a typical magnitude-mode
spectrum \ac{2DJ}, though with a significant loss of sensitivity.

\subsection{The \acs{BIRD} Method}
The \ac{BIRD} pulse sequence element\cite{Garbow1982,Bax1983} also takes
advantage of the idea of selectively inverting passive spins, while leaving
active spins unaffected.
However the active spins are those which are directly bound to a low natural
abundance
heteronucleus, with the two most common heteronuclei used being
\textsuperscript{13}C (1.1\% abundance) and \textsuperscript{15}N (0.37\%
abundance).
The passive spins are those bound to far more abundant nucleus (i.e.
\textsuperscript{12}C or \textsuperscript{14}N). The reduction in sensitivity
of the experiment relative to a full-sensitivity experiment is therefore known
and constant across samples. In scenarios where strong coupling exists, \ac{BIRD} can
achieve improved sensitivity over \ac{ZS}, since with the latter a very weak
selective pulse would be required to ensure it is of a sufficiently small
bandwidth. The \ac{BIRD} method is particularly attractive in scenarios where
the sensitivity penalty due to the involvement of a low-abundance nucleus has
already been paid, for example in sequences where an \ac{INEPT} element is
present\cite{Paudel2013}. One of \ac{BIRD}'s primary drawbacks is the fact that
geminal protons (i.e. protons bound to the same heteroatom) cannot be decoupled
from each other, since such protons are always in the same subset of either
active or inactive nuclei. Doublets rather than singlets will arise in such
cases.

\subsection{\acs{PSYCHE}}
\label{subsec:psyche}
The most recent major development in pure shift spectroscopy is the \ac{PSYCHE}
experiment\cite{Foroozandeh2014,Foroozandeh2018}.
\note{Description... Element, How it works (very simple), Effectiveness}

With \ac{PSYCHE}, the proportion of active and passive spins is
dependent on the flip angle of the chirp pulses; for a
\ac{PSYCHE} element featuring chirp pulses with flip angles $\beta$, the
proportions of are $\cos^2 \beta$ and $\sin^2 \beta$, respectively.

The \ac{PSYCHE} element has also been employed in conjunction with the \ac{2DJ}
experiment in order to produce spectra which already feature orthogonal
separation of the chemical shifts and couplings along the two frequency
axes\cite{Foroozandeh2015,Kiraly2017}. Being a 3D experiment, the
\ac{PSYCHE}-\ac{2DJ} requires long experiment times (typically tens of hours)
in order to produce a spectrum with well-resolved multiplet structures in the
indirect dimension.

\subsection{Pure shift spectra from 2DJ estimation}
\note{Mandelstahm?}

Beyond specialised pulse sequences, procedures based on the estimation of
\ac{2DJ} datasets have also been developed to achieve broadband homodecoupling.
Nuzillard introduced \ac{ALPESTRE}\cite{Nuzillard1996,Martinez2012}, in which
the parameters of each indirect-dimension FID are estimated using \ac{LPSVD},
such that a set of parameters $\symbf{\Theta} \in \mathbb{R}^{\Ntwo
\times 4M}$ is generated.
\begin{equation}
    \symbf{\theta}_{\ntwo} =
    \begin{bmatrix}
        \bda_{\ntwo}\T &
        \bdphi_{\ntwo}\T &
        \bdf_{\ntwo}\T &
        \bdeta_{\ntwo}\T
    \end{bmatrix}\T.
\end{equation}
The parameters generated are used to propagate each FID backward into
$-\tone$, producing a ``full-echo'':
\begin{equation}
    \begin{split}
        y^{\text{full}}_{\none,\ntwo} = \sum_{m=1}^{M}
            a_{\ntwo,m}
            \exp(\iu \phi_{\ntwo,m})
            \exp\left(\left(2 \pi \iu f_{\ntwo,m} \none
            -\eta_{\ntwo,m}  \left\lvert \none \right\rvert \right)\Dtone\right), \\
        \forall \none \in \lbrace -\None + 1, \cdots, 0, \cdots, \None - 1 \rbrace,\ \forall \ntwo \lbrace 0, \cdots, \Ntwo - 1 \rbrace.
    \end{split}
    \label{eq:full-echo}
\end{equation}
\ac{FT} of \cref{eq:full-echo} generates a spectrum whose real component
comprises absorption-mode
Lorentzian character in both dimensions. This opens up the means of producing
pure-shift spectra from the \ac{2DJ} experiment with sharp lineshapes and
without signal loss. A similar approach proposed by Mutzenhardt et al.
instead constructs full echoes via \ac{LP} of each direct-dimension
\ac{FID}, and generates a full echo by propagating into
$-\ttwo$\cite{Mutzenhardt1999}.


