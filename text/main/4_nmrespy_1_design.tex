\section{The Structure of \acs{EsPy}}

\subsection{Why \textsc{Python}?}
There a number of reasons why \textsc{Python} was the chosen programming
language for \ac{EsPy}:
\begin{itemize}
    \item It has a large user-base, particularly within the scientific
        community.
    \item The \textsc{SciPy} ecosystem\cite{Virtanen2020}, including the packages
        \textsc{NumPy}\cite{Harris2020} and
        \textsc{Matplotlib}\cite{Hunter2007} is a powerful tool which enables
        high-performance scientific computation in \textsc{Python}
    \item Being a scripting language makes \textsc{Python} ideal for
        exploring datasets in a step-by-step fashion. This is useful in the
        context of \ac{NMR} estimation, as the user may want to
        (a) inspect and pre-process then data, ten (b) determine the regions
        they wish to estimate, then (c) setup the estimation routine, and
        finally (d) output the estimation result. This can be achieved rather
        easily by hacking and re-running \textsc{Python} scripts or by using
        ``notebook'' environments, such as \textsc{Jupyter}.
    \item It is free and open-source, as opposed to well-known scientific
        computing platforms such as \textsc{Matlab}\textregistered\ and
        \textsc{Mathematica}.
    \item \textsc{Python} supports sophisticated object-oriented programming
        features, such as multiple levels of inheritance.
\end{itemize}

Probably the biggest drawback of \textsc{Python} is its slow performance on
account of it being an interpreted\footnote{
    With interpreted languages, the source code is processed line-by-line at
    the time of running by an interpreter. This differs from compiled
    languages, where prior to being run, the source code is converted to
    machine-readable byte-code.
}
, dynamically typed\footnote{
    Dynamically typed languages, as opposed to statically typed languages like
    \textsc{C}, \textsc{C++}, \textsc{Java}, \textsc{Rust} etc. allow for a
    variable which is initially assigned a given type to be re-assigned to a
    completely different type.
}, language with relies on garbage collection\footnote{
    Garbage collection involves a program routinely checking for any memory
    that has become dereferenced, and clearing this memory up.
}
for memory management. While \textsc{NumPy} provides interfaces to run fast
computations with pre-compiled C-code, a significant performnace benefit would
likely be realised if a low-level compiled language like C, C++, or
\textsc{Rust} were used. However the development time in writing programs with
these lower-level languages is typically a lot greater than with a language
with a higher level of abstraction like \textsc{Python}.

\subsection{???}
The fundamental user-facing object that \ac{EsPy} provides is the
\texttt{Estimator} class and its numerous inherited classes, which facilitate
the estimation of different types of \ac{NMR} data. A
complete list of estimator objects at the time of writing is given in Table
\ref{tab:estimators}

\begin{longtable}{c p{9cm}}
\caption[
    A complete list of estimator objects provided by \ac{EsPy} at the time of writing.
]{
    A complete list of estimator objects provided by \ac{EsPy} at the time of writing.
    \note{Maybe indicate that the inv rec and diffusion estimators share
    similar functionality, except for the amplitude fitting. Diffusion
experiments only differ in definition of fitting constant.}
}
\label{tab:estimators}\\
\hline
Type & Description \\
\hline
\texttt{Estimator1D} & For consideration of 1D datasets. \\
\texttt{Estimator2D} & For consideration of 2D datasets comprising a pair of
States (amplitude-modulated) signals. \\
\texttt{Estimator2DJ} & For consideration of \ac{2DJ} datasets. This object
provides the functionality to generate pure shift spectra and assign multiplet
structures using \ac{CUPID}.\\
\texttt{EstimatorInvRec} & For consideration of inversion recovery ($T_1$)
experiments. \\
\texttt{EstimatorDiffusionMonopolar} & For consideration of monopolar gradient
diffusion experiments. \\
\texttt{EstimatorDiffusionBipolar} & For consideration of bipolar gradient
diffusion experiments. \\
\texttt{EstimatorDiffusionOneshot} & For consideration of one-shot \ac{DOSY}
experiments. \\
\hline
\end{longtable}
