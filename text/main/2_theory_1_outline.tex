\section{Outline of the Problem}
\label{sec:theory-outline}
For the purposes of this work, it is always assumed that an \ac{FID} to be
estimated is hypercomplex in form, meaning that it obeys
\eqref{eq:general-fid} with $\zeta = \exp(\iu \cdot)\ \forall d \in \lbrace 1,
\cdots D \rbrace$:
\begin{subequations}
    \begin{gather}
        \bY \left[\none, \cdots, \nD\right] =
            \bX \left( \bth \right) \left[\none, \cdots, \nD\right] +
            \bW \left[\none, \cdots, \nD\right] \\
        \bX \left(\bth\right) \left[\none, \cdots, \nD\right] =
        \sum_{m=0}^{M-1} \bdam \exp\left(
                \iu \bdphim
            \right)
            \prod_{d=1}^D
            \exp\left(
                \left[ 2 \pi \iu \left(\bdfdm - \foffd\right) - \bdetadm \right] \nd \Dtd
            \right),
            \label{eq:x}
            \\
        \bW \left[\none, \cdots, \nD\right] \sim
            \mathcal{N_C}\left(0, 2\sigma^2\right),
    \end{gather}
    \label{eq:hypercomplex-fid}
\end{subequations}%
where $\Dtd = \nicefrac{1}{\fswd}$. Under this model, it is assumed that
\iac{FID} consists of a summation of $M$ damped complex sinusoids in the
presence in \ac{AWGN}. It is the goal of parametric estimation to establish the
identity of all the quantities which describe the model component $\bX$. These
can be distilled into the vector $\bth \in \mathbb{R}^{2(D + 2)M}$:

Due to the assumed \ac{AWGN} nature of the noise array, the \ac{pdf} of an individual noise component
$w \coloneq \symbf{W}\left[\none, \cdots, \nD\right]$ is
\begin{equation}
    p(w) =
        \frac{1}{2\pi \sigma^2}
        \exp\left( -\frac{\left\lvert w \right\rvert^2}{2\sigma^2}\right).
\end{equation}
As the elements are independent and identically distributed, the joint \ac{pdf}
describing the entire noise array is given by the product of each element's
\ac{pdf}:
\begin{equation}
    \begin{split}
        p\left(\bW\right) &=
            \prod_{\none=0}^{\None - 1}
            \cdots
            \prod_{\nD=0}^{\ND - 1}
            \frac{1}{2\pi \sigma^2}
            \exp\left(
                -\frac
                {\left\lvert w \right\rvert^2}
                {2\sigma^2}\right) \\
            &= \frac{1}{\left(2\pi \sigma^2\right)^{\mathfrak{N}}}
            \exp\left( -\frac{\left\lVert \bW \right\rVert^2}{2\sigma^2}\right)
    \end{split}
\end{equation}
As the noise array is the difference between the data and model, the
likelihood function of $\bth$ given $\bY$, $\mathcal{L}\left(\bth \vert
\bY\right)$, is given by
\begin{equation}
    \mathcal{L}\left(\bth \vert \bY\right) =
    \frac{1}{\left(2\pi \sigma^2\right)^{\mathfrak{N}}}
        \exp\left( -\frac{\left\lVert \bY - \bX(\bth) \right\rVert^2}{2\sigma^2}\right).
\end{equation}
It is common to consider instead the log-likelihood function of $\bth$ given
$\bY$, $\ell\left(\bth \vert \bY\right)$. As application of the logarithm is a
monotonic transformation, the arguments of the maxima of $\mathcal{L}$ and
$\ell$ are equivalent.
\begin{equation}
    \ell\left(\bth \vert \bY\right) =
        -\mathfrak{N} \ln\left(2 \pi \sigma^2\right)
        -\frac{\left\lVert \bY - \bX(\bth) \right\rVert^2}{2\sigma^2}.
    \label{eq:log-likeihood}
\end{equation}
Equation \ref{eq:log-likeihood} implies that the optimal set of parameters
$\bth^{(*)}$
is that which minimises the
(square) norm of the difference between the data and model
\begin{equation}
    \bthstar = \argmax_{\bth \in \mathbb{R}^{2(D+2)M}}
        \ell\left(\bth \vert \bY\right) \equiv
        \argmin_{\bth \in \mathbb{R}^{2(D+2)M}} \left\lVert \bY - \bX(\bth) \right\rVert^2.
    \label{eq:argmin_y-x}
\end{equation}
The application of \ac{NLP} is a well-established approach to solve such a
problem\cite{Fletcher1987,Nocedal2006}. The basic principle behind \ac{NLP} is
to iteratively explore - in a methodical way - how a function varies with its
arguments, using information about the function and optionally its
derivatives, and to terminate once conditions have been achieved which imply an
extremum has been found. While derivative-free approaches to \ac{NLP} do
exist,\note{(cite examples like simplex, simmulated annealing, BOBYQA?)} in
scenarios where the function under consideration has well-defined,
computationally tractable derivatives, the use of these can be valuable to
solving optimisation problems. The problem outlined in Equation
\ref{eq:argmin_y-x} is such an example.


