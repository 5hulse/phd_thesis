\section{Outline of the Problem}
\label{sec:theory-outline}
Through the application of \ac{NMR} theory, one can
rationalise how a particular spin system, subjected to a given pulse sequence,
is mapped to an \ac{FID}. This is an example of a \emph{forward problem}, in
which one wishes to determine how a set of parameters is mapped to the
resulting data.
The chemical shifts, J-couplings, dipolar couplings, etc. of the spin system may
be recast as a list of signal amplitudes, phases, frequencies and
damping factors, which define the form of the \ac{FID}.
Many sophisticated pieces of software such as \textsc{Spinach}\cite{Hogben2011}
solve this forward problem, yielding \ac{NMR} experiment simulations.
Simulating an \ac{NMR} experiment is a \emph{well-posed} problem, since it
satisfies the following properties:
\begin{enumerate}
    \item The problem has a solution.
    \item The solution is unique.
    \item The solution's behaviour changes continuously with the parameters
        provided. For example, continuously changing the chemical shift of a
        given spin will lead to the position of its signal(s) in the resulting
        spectrum to continuously vary. Other phenomena such as strong coupling
        effects may manifest as a given spin's chemical shift changes;
        these will also evolve in a continuous fashion as well.
\end{enumerate}
The parametric estimation of \acp{FID} is an example of an \emph{inverse problem};
given an acquired dataset, one seeks to determine the set of parameters which
went into constructing it.
Parametric estimation, as with many inverse problems, is
\emph{ill-posed}\cite{Kabanikhin2008}. These problems do not satisfy at least
one of the three properties above. For example, given \iac{FID}, it is possible
to conceive of numerous signal parameter specifications which would lead to a
faithful representation of the data in a least squares sense.
Therefore, determining the ``optimal'' parameter set is a rather difficult
challenge.

For the purposes of this work, it is always assumed that an \ac{FID} to be
estimated
$\bY \in \mathbb{C}^{\None \times \cdots \times \ND}$
is hypercomplex in form, meaning that it obeys
\cref{eq:general-fid} with $\zeta^{(d)} = \exp(\iu \cdot)\ \forall d \in
\lbrace 1, \cdots, D \rbrace$:
\begin{subequations}
    \begin{gather}
        \ynonenD = \xnonenD(\bth) + \wnonenD,\\
        \xnonenD(\bth) =
        \sumM \amexpphim
        \prodD \exp\left(\left(
            2 \pi \iu \left(f^{(d)}_m - \foffd\right)
            -\eta_m^{(d)}\right)
            \nd \Dtd\right),\label{eq:x}\\
        \wnonenD \sim \mathcal{N_C}\left(0, 2\sigma^2\right),%
    \end{gather}%
    \label{eq:hypercomplex-fid}%
\end{subequations}%
where $\Dtd = \nicefrac{1}{\fswd}$.
It should be noted that, prior to estimating the dataset, it is normalised
such that the signal actually under consideration is $\nicefrac{\bY}{\lVert \bY
\rVert}$.
To make the final result reflect the unnormalised dataset, the estimated
amplitudes can be multiplied by $\lVert \symbf{Y} \rVert$.
Under this model, it is assumed that
\iac{FID} consists of a summation of $M$ damped complex sinusoids in the
presence in \ac{AWGN}.
It is the goal of parametric estimation to establish the
identity of all the quantities which describe the model component $\bX$, which
are distilled into the vector $\bth \in \mathbb{R}^{2(D + 1)M}$, given by
\cref{eq:theta}.
\Cref{eq:x} can be expressed in terms of complex amplitudes and signals poles
as follows:
\begin{subequations}%
    \begin{gather}%
        \xnonenD(\bth) = \sumM \alpha_m \prodD {z^{(d)}_m}^{\nd},\\
        \alpha_m = \amexpphim,\\
        z_m^{(d)} = \exp\left(
            \left(2 \pi \iu \left(f_m^{(d)} - \foffd\right) - \eta^{(d)}_m\right) \Dtd
        \right).%
    \end{gather}%
    \label{eq:x-alpha-z}%
\end{subequations}%
Due to the assumed \ac{AWGN} nature of the noise array, the \acf{PDF} of an
individual noise component is
\begin{equation}
    p(\wnonenD) =
        \frac{1}{2\pi \sigma^2}
        \exp\left( -\frac{\left\lvert \wnonenD \right\rvert^2}{2\sigma^2}\right).
\end{equation}
As the elements are independent and identically distributed, the joint \ac{PDF}
describing the entire noise array is given by the product of all the elements'
\acp{PDF}:
\begin{equation}
    \begin{split}
        p\left(\bW\right) &=
            \prod_{\none=0}^{\None - 1}
            \cdots
            \prod_{\nD=0}^{\ND - 1}
            \frac{1}{2\pi \sigma^2}
            \exp\left(
                -\frac
                {\left\lvert \wnonenD \right\rvert^2}
                {2\sigma^2}\right) \\
            &= \frac{1}{\left(2\pi \sigma^2\right)^{\mathfrak{N}}}
            \exp\left( -\frac{\left\lVert \bW \right\rVert^2}{2\sigma^2}\right).
    \end{split}
\end{equation}
As the noise array is the difference between the data and model, the
likelihood function of $\bth$ given the \ac{FID} $\bY$ is
\begin{equation}
    \mathcal{L}(\bth \vert \bY) =
    \frac{1}{\left(2\pi \sigma^2\right)^{\mathfrak{N}}}
        \exp\left( -\frac{\left\lVert \bY - \bX(\bth) \right\rVert^2}{2\sigma^2}\right).
\end{equation}
It is common to consider instead the log-likelihood function,
$\ell(\bth \vert \bY) \coloneq \ln \mathcal{L}(\bth \vert
\bY)$:
\begin{equation}
    \ell(\bth \vert \bY) =
        -\mathfrak{N} \ln\left(2 \pi \sigma^2\right)
        -\frac{\left\lVert \bY - \bX(\bth) \right\rVert^2}{2\sigma^2}.
    \label{eq:log-likeihood}
\end{equation}
As application of the logarithm is a monotonic transformation, the
arguments of the maxima of $\mathcal{L}$ and $\ell$ are equivalent.
\Cref{eq:log-likeihood} implies that the optimal set of parameters
$\bth^{(*)}$, i.e. the \ac{MLE},
is that which minimises the \ac{RSS} between the data and the model:
\begin{equation}
    \bthstar = \argmax_{\bth \in \mathbb{R}^{2(D+1)M}}
        \ell\left(\bth \vert \bY\right) \equiv
        \argmin_{\bth \in \mathbb{R}^{2(D+1)M}} \left\lVert \bY - \bX(\bth) \right\rVert^2.
    \label{eq:argmin_y-x}
\end{equation}
The application of \ac{NLP} is a well-established approach to solve such a
problem\cite{Fletcher1987,Nocedal2006}. The basic principle behind \ac{NLP} is
to iteratively explore, in a methodical way, how a function varies with its
arguments. By using information about the function and optionally its
derivatives, such a routine attempts to find a minimum in the function, and
terminates once this has been achieved. While derivative-free approaches to
\ac{NLP} do exist\cite{Nelder1965,Kirkpatrick1983,Powell2009},
in scenarios where the function under consideration has well-defined,
computationally tractable derivatives, the use of these can be valuable to
solving optimisation problems; the problem outlined in
\cref{eq:argmin_y-x} is such an example.

As discussed already, for \ac{NLP} to perform effectively, a large amount of
\textit{a priori} information is typically required, in the form of an initial
guess, possibly alongside other constraints. To achieve this, the method
employed in this work makes use of the \ac{MPM}, the subject of the next
section.
