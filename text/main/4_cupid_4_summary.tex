\section{Summary}
In this chapter, \ac{CUPID}, a procedure for the construction of pure shift
spectra via the holistic estimation of \ac{2DJ} datasets, is presented.
Such spectra possess myriad beneficial features relative to alternative
methods, albeit with the requirement of a complex post-processing procedure.
\correction{
\label{corr:cupid-steps}
The following steps must be carried out in order to use \ac{CUPID}:
\begin{enumerate}
    \item Acquire a conventional hypercomplex \ac{2DJ} dataset of the sample of
        interest. If a \textsc{Bruker} system is being used, this can be
        achieved using the built-in \texttt{jresqf} pulse sequence.
    \item Inspect the (magnitude mode) \ac{2DJ} spectrum to locate
        direct-dimension regions of
        interest. If particular regions are very crowded, featuring overlapping
        multiplet structures, a rough estimate of the number of peaks present
        in these should be made. It is better for an over-estimate of the
        number of peaks to be given to the estimation routine, rather than an
        under-estimate.
    \item Apply phase correction to the direct dimension of the dataset; this
        ensures that the incorporation of phase variance into the \ac{NLP}
        fidelity is valid.
    \item Estimate the regions of interest in turn (using the \ac{MMEMPM}
        followed by \ac{NLP}).
    \item Inspect the estimation result, and generate the final pure shift
        spectrum via the \ang{-45} signal.
\end{enumerate}
As will be illustrated in the next chapter, steps 2 onwards can be carried out
easily using the \ac{EsPy} package.
}

Through a number of examples, it has been shown that by employing parametric
estimation, a simple \ac{2DJ} experiment can be harnessed to generate pure
shift spectra with sharp absorption Lorentzian peaks which retain the same
signal intensity as the \ac{2DJ} experiment. \ac{CUPID} is able to perform
admirably even when state of the art techniques like \ac{PSYCHE} produce
spectra with such low \acp{SNR} as to render them unusable. Frequently, \ac{CUPID}
is able to automatically discard oscillators present in the model which either
correspond to strong coupling artefacts or noise, leading to simplified spectra
which appear to adhere to the weak coupling regime. There are cases where
strong coupling artefacts do end up in the estimation result. It has been shown
that there are circumstances where these can be manually neglected from the
parameter set to prevent them from having an unwanted influence on final pure
shift spectrum, though this requires manual intervention from a knowledgeable
user.

Simultaneously, \ac{CUPID} can assign multiplet structures,
by grouping oscillators which lie along a specific \ang{45} cross section in
frequency space. Achieving this experimentally\,---\,effectively involving
a \ac{2DJ} experiment in conjunction with a pure shift element\,---\,requires
running an extremely long (hours or even days) \ac{3D} pulse sequence. The
usefulness of the multiplet structures generated by \ac{CUPID} is dependent on
the level of the estimation routine's accuracy. On
numerous occasions within the examples presented here, the estimation routine
was unable to resolve certain, similar frequency signals. A complete
understanding of the coupling network associated with a given spin is not
attainable when this is so. However, such multiplet structures can still provide
valuable insights into the sample being studied.

\ac{CUPID} is limited by the complexity of the dataset of interest. The reasons
for this are two-fold. First, for datasets comprising progressively more peaks
in a given spectral region, the difficulty in generating accurate parameter
estimates becomes harder. Second, with an increased model order required to
estimate the dataset, the computation time increases drastically.
This feature is most clearly observed when comparing the times required in
estimating the different regions considered in the estradiol example. As a rule
of thumb, it is anticipated that \ac{CUPID} will perform admirably on datasets
derived from small molecules, though datasets derived from large molecules such
as proteins are likely to be too complex and demanding for good results.
