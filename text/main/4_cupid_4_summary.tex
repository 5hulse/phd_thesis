\section{Summary}
In this chapter, \ac{CUPID}, a procedure for the construction of pure shift
spectra via the holistic estimation of \ac{2DJ} datasets, is presented.
Such spectra can be possess myriad beneficial features relative to alternative
methods, albeit with the requirement of a complex post-processing procedure.

% The original method for pure shift spectrum generation consisted of shearing a
% magnitude-mode \ac{2DJ} spectrum by \ang{45}, and computing the projection onto
% the $\Ftwo$ axis. To overcome the grotesque lineshapes which arise due to the
% presence of dispersion character, and non-linearities in the magnitude-mode
% spectrum, severe data treatment such as the used of sine-bell apodisation are
% applied. The resulting pure shift spectra suffer from reduced intensities
% because of this. On top of this, the intensities of peaks are attenuated by
% different extents, such that relative peak integrals become meaningless. The
% presence of strong coupling in the spin system also introduces unwanted
% artefacts into the spectra.

% Experimental procedures  based on ``chunking'' the initial sections of the
% \acp{FID} in a \ac{2D} experiment\,---\,including \ac{ZS}, \ac{BIRD} and
% \ac{PSYCHE}\,---\,have largely superseded the shear and summation approach.
% One key disadvantage of all of these is that only a fraction of the available
% spin magnetisation contributes to the final pure shift spectrum, leading to
% poorer sensitivity.

Through a number of examples, it has been shown that by employing parametric
estimation, a simple \ac{2DJ} experiment can be harnessed to generate pure
shift spectra with sharp absorption Lorentzian peaks which retain the same
signal intensity as the \ac{2DJ} experiment. It it is able to perform
admirably even when state of the art techniques like \ac{PSYCHE} produce
spectra with such low \acp{SNR} as to render them unusable. Frequently, \ac{CUPID}
is able to automatically discard oscillators present in the model which either
correspond to strong coupling artefacts or noise, leading to simplified spectra
which appear to adhere to the weak coupling regime. There are cases where
strong coupling artefacts do end up in the estimation result. It has been shown
that these can be manually neglected from the parameter set so they don't have
an unwanted influence on final pure shift spectrum, though this requires manual
intervention from a knowledgeable user.

Simultaneously, \ac{CUPID} can assign multiplet structures,
by grouping oscillators which lie along a specific \ang{45} cross section in
frequency space. Achieving this experimentally\,---\,effectively involving
a \ac{2DJ} experiment in conjunction with a pure shift element\,---\,requires
running an extremely long (hours or even days) \ac{3D} pulse sequence. The
usefulness of the multiplet structures generated by \ac{CUPID} is dependent on
the level of the estimation routine's accuracy. On
numerous occasions within the examples presented here, the estimation routine
was unable to resolve certain, similar frequency signals. A complete
understanding of the coupling network associated with a given spin in not
attainable when this is so. However, such multiplet structures can still provide
valuable insights into the sample being studied.

\ac{CUPID} is limited by the complexity of the dataset of interest. The reasons
for this are two-fold. First, for datasets comprising progressively more peaks
in a given spectral region, the difficulty in generating effective parameter
estimates becomes harder. Second, with an increased model order required to
estimate the dataset, the time required for computation increases drastically.
This feature is most clearly observed when comparing the times required in
estimating the different regions considered in the estradiol example. As a rule
of thumb, it is anticipated that \ac{CUPID} will perform admirably on datasets
derived from small molecules, though datasets derived from large molecules such
as proteins are likely to be too complex and demanding for good results.
