\section{Frequency Filtration}
\label{sec:filtering}
The previous section provides motivation for finding ways to reduce both the
number of datapoints and the number of signals present in an \ac{FID},
without compromising the ability of the estimation routine
to parameterise it. A means of generating frequency-filtered ``sub-\acp{FID}''
is presented in this section, which is able to achieve both of these. In
essence, this sub-\ac{FID} approach transforms the problem of \ac{FID} estimation from a
single large-scale estimation problem to a plurality of smaller-scale problems.
As well as realising vast improvements in computational speed, filtering also
enables a user to focus solely on spectral regions that are of interest.
It is common for certain spectral regions to be so densely populated
that it is futile to attempt to extract meaningful quantitative information at
the per-signal level, especially with \ac{1D} \ac{NMR} data. Through data filtration,
all focus can be devoted to those frequency regions which can realistically be
studied and are of interest in the first place.

% The concept of estimating \ac{NMR} through the consideration of ``subbands''
% has existed for a long time, and can be traced back to Tang and Norris's
% \ac{LP}-ZOOM method, and extension \ac{LPSVD}\cite{Tang1988}.
% A number of methods, including adaptive ones in which the estimated subbands
% are adjusted according to certain criteria, have also
% emerged\cite{Djermoune2004}\note{More citations!}.
\subsection{The \acl{VE}}
\label{subsec:ve}
The key steps of the filtering procedure are
(a) transforming the time-domain data to the frequency domain,
(b) applying a band-pass filter to the spectral region of interest, and
(c) returning the spectrum back to the time-domain for estimation.
For a filtered sub-\ac{FID} to still be faithfully modelled by a
summation of exponentially damped complex sinusoids, it is necessary that the
spectral peaks of interest lie (effectively) entirely within the filter
region\footnote{
    A Lorentzian function tends to but never explicitly reaches zero as the
    distance from its maximum tends to $\infty$ (\cref{eq:lorentzian}).
    However, as long as a sufficiently wide filter region is defined, the
    extent to which the ``tails'' of the Lorentzian fall outside the filter
    window can be assumed to be negligible, especially when in the presence of
    noise.
}.
Due to their narrower linewidths relative to dispersion Lorentzians, a phased
spectrum solely comprising absorption Lorentzians is therefore desired.
The \ac{VE} has been employed here, which has found application in the field of
compressed sensing NMR\cite{Mayzel2014,Golowicz2020,Luo2020}. The \ac{VE} is a
signal with double the size as the original \ac{FID}, with the key
characteristic that its \ac{FT} has an imaginary component of zeros. The
\ac{VE} concept can be applied to data of any number of dimensions. However,
discussion here will be limited to \ac{1D} \acp{VE}.
An account of the \ac{2D} \ac{VE} is provided in \cref{sec:multidim-ve}.

Assuming that a \ac{1D} \ac{FID} $\by \in \mathbb{C}^N$ is phased, such that
$\bdphi = \symbf{0} \in \mathbb{R}^M$, it can be described by
\begin{subequations}
    \begin{gather}
        y_n = \xi_n (c_n + \iu s_n) + w_n,\\
        \xi_n = \sum_m a_m \exp(-\eta_m n \Dt),\\
        c_n / s_n = \sum_m \cos / \sin(2 \pi f_m n \Dt).
    \end{gather}
\end{subequations}
The frequency-dependence has been decomposed into its real and imaginary
components. With this in mind, a conjugate pair of signals $\symbf{\psi}_{\pm}
\in \mathbb{C}^N$ are defined:
\begin{equation}
    \psi_{\pm,n} = \xi_n (c_n \pm \iu s_n) + w_n \equiv \Re(y_n) \pm \iu \Im(y_n)
\end{equation}
Two vectors $\symbf{t}_{1}, \symbf{t}_2 \in \mathbb{C}^{2N}$
are constructed using the conjugate pair.
$\symbf{t}_1$ is given by $\symbf{\psi}_+$ padded with zeros from below:
    \begin{equation}
        \symbf{t}_1 = \begin{bmatrix}
            \symbf{\psi}_+ \\ \symbf{0} \in \mathbb{C}^{N}
        \end{bmatrix}.
    \end{equation}
$\symbf{t}_2$ is given by $\symbf{\psi}_{-}$ with its elements in
    reversed order ($\cdot^{{\leftrightsquigarrow}}$), padded with zeros
    from above, and finally subjected to a right circular shift by one
    element ($\cdot^{{\circlearrowright}}$):
    \begin{equation}
        \symbf{t}_2 = \begin{bmatrix}
            \symbf{0} \in \mathbb{C}^{N} \\ \symbf{\psi}_-^{{\leftrightsquigarrow}}
    \end{bmatrix}^{{\circlearrowright}}.
   \end{equation}
The \ac{VE} $\by_{\text{VE}}$ is then given by $\symbf{t}_1 +
\symbf{t}_2$, with the first element divided by $2$. This entire process is
equivalent to
\begin{equation}
    \by_{\text{VE}} =
    \begin{bmatrix}
        \Re(y_0^{\vphantom{*}}) &
        y_1^{\vphantom{*}} &
        \cdots &
        y_{N-1}^{\vphantom{*}} &
        0 &
        y_{N-1}^* &
        \cdots &
        y_1^*
    \end{bmatrix}\T.
\end{equation}
As eluded to already, the \ac{FT} of $\by_{\text{VE}}$ produces a spectrum
$\symbf{s}_{\text{VE}}$ such that $\Im\left(\symbf{s}_{\text{VE}}\right) =
\symbf{0}$, with $\Re\left(\symbf{s}_{\text{VE}}\right)$ featuring absorption
Lorentzian peaks.

\subsection{The filtering process}
To filter the spectrum $\symbf{s}_{\text{VE}}$, it is subjected multiplication
with a function which acts as a band-pass filter. An example of a suitable
filter is a \emph{super-Gaussian} $\symbf{g} \in \mathbb{C}^{2N}$ defined by a
central index  $c \in \lbrace 0, \cdots, 2N-1 \rbrace$ and a bandwidth $b \in
\lbrace 0, \cdots, 2N-1 \rbrace$:
\begin{equation}
    g_n = \exp \left(-2^{p+1} \left(\frac{n - c}{b}\right)^p\right).
    \label{eq:super-Gaussian-onedim}
\end{equation}
The scalar $p \in \mathbb{R}_{>0}$ dictates the steepness
of the filter at the boundaries, with the function becoming more rectangular
as it increases.
The central index and bandwidth of the super-Gaussian filter function are given
by the following expressions:
\begin{subequations}
    \begin{gather}
        c = \tfrac{1}{2} \left(l_{\text{idx}} + r_{\text{idx}}\right), \\
        b = r_{\text{idx}} - l_{\text{idx}},
    \end{gather}
\end{subequations}
where $l_{\text{idx}}, r_{\text{idx}} \in \lbrace 0, \cdots, 2N-1 \rbrace,
r_{\text{idx}} > l_{\text{idx}}$ denote the left and right
boundaries of the region of interest, defined by the user, expressed as vector
indices.
The vector indices can be obtained from the corresponding spectral frequencies
$f_{\unit{\hertz}}$ via
\begin{equation}
    \begin{gathered}
        f_{\text{idx}} =
            \left \lfloor
                \frac
                {
                    \left(2N - 1\right)
                    \left(\fsw + 2 \left(\foff - f_{\unit{\hertz}}\right) \right)
                }
                {2 \fsw}
            \right \rceil \\
        \forall f_{\unit{\hertz}} \in
            \left[\foff - \tfrac{1}{2} \fsw, \foff + \tfrac{1}{2} \fsw\right].
        \label{eq:fidx}
    \end{gathered}
\end{equation}
Alternatively, conversion from \unit{\partspermillion} to array indices can be
achieved by replacing  $f_{\unit{\hertz}}$ in \cref{eq:fidx} with
$f_{\unit{\partspermillion}} f_{\text{sfo}}$, where $f_{\text{sfo}}$ is the
transmitter frequency (\unit{\mega \hertz}) and $f_{\unit{\partspermillion}}$
is the frequency expressed as a chemical shift.

Application of the
super-Gaussian filter to $\symbf{s}_{\text{VE}}$
would lead to large sections of the filtered spectrum being $0$. This has an
undesired impact on the \ac{MDL}, as noise that resides within the filter
region will now seem to resemble true signal,
as its amplitude is infinitely greater than the zeroed regions. A massive
over-estimation of model order results from this.
In order to obtain better predictions from model order selection, an array of
synthetic \ac{AWGN} is
added to the filtered spectrum. To achieve this, a region in
$\symbf{s}_{\text{VE}}$ is specified by the user which contains no discernible
signal peaks (referred to as the \emph{noise region}). The variance of this region
$\sigma^2$ is determined, and used to construct a vector of values sampled from
a normal distribution with mean $0$ and variance $\sigma^2$,
$\symbf{w}_{\sigma^2} \in \mathbb{R}^{2N}$.
The filtered spectrum is then given by
\begin{equation}
    \widetilde{\symbf{s}}_{\text{VE}} = \symbf{s}_{\text{VE}} \odot \symbf{g} + \symbf{w}_{\sigma^2} \odot \left(\symbf{1} - \symbf{g} \right).
    \label{eq:Sve-tilde}
\end{equation}
Note that the noise array's magnitude at each point is attenuated based on the
value of the super-Gaussian filter, as a means of ensuring the noise variance
remains consistent across the frequency space.

After filtering, $\widetilde{\symbf{s}}_{\text{VE}}$ is returned to the
time-domain by \ac{IFT}, defined for a generic frequency-domain vector $\symbf{s} \in \mathbb{C}^N$ as
\begin{equation}
    y_n = \frac{1}{N} \sum_{k=0}^{N-1} s_k
        \exp\left(\frac{2 \pi \iu k n}{N}\right)
        \quad \forall n \in \lbrace 0, \cdots, N-1 \rbrace.
\end{equation}
The \ac{IFT} of a real-valued spectrum generates a
conjugate-symmetric signal, which is also a \ac{VE}. This is sliced so as to
retain the first half, which is the final filtered sub-FID $\widetilde{\by} \in
\mathbb{C}^{N}$.
A depiction of the key elements involved in the filtering process is provided
by \cref{fig:filtering}, while a pseudo-code description is provided by
\cref{alg:filter-1d}.
\begin{figure}
     \centering
     \includegraphics{filtering/filtering.pdf}
     \caption[
         An illustration of the filtering procedure applied to a \acs{1D}
         \acs{FID}.
     ]{
         An illustration of the filtering procedure applied to a \ac{1D}
         \ac{FID}.
         \textbf{a.} A \ac{VE} $\by_{\text{VE}}$, with the first and last
         $N$ points coloured red and blue, respectively. The middle of the
         \ac{VE} is magnified to highlight its conjugate symmetry.
         \textbf{b.} The \ac{FT} of the \ac{VE}, $\symbf{s}_{\text{VE}}$.
         The region of interest (orange) and noise region (grey) are denoted.
         \textbf{c.} A super-Gaussian function used as a band-pass filter,
         $\symbf{g}$.
         \textbf{d.} Synthetic noise vector to be added to the filtered
         spectrum, $\symbf{w}_{\sigma^2} (\symbf{1} - \symbf{g})$.
         \textbf{e.} The filtered spectrum $\widetilde{\symbf{s}}_{\text{VE}}$,
         formed by applying the super-Gaussian filter, and adding the noise
         vector.
         \textbf{f.} The \ac{IFT} of the filtered spectrum,
         $\widetilde{\symbf{y}}_{\text{VE}}$, from which the final filtered
         signal $\widetilde{\symbf{y}}$ is obtained by extracting
         the first $N$ (red) points.
     }
     \label{fig:filtering}
\end{figure}

Thus far, the method described is able to reduce the number of signals,
though the filtered sub-\ac{FID} still comprises the same number of datapoints.
However, it is clear that there are a large number of points outside the region
of interest in $\widetilde{\symbf{s}}_{\text{VE}}$ that do not possess any
meaningful information. Discarding such points will then lead to a sub-\ac{FID}
with the same information, but in a more compressed \ac{FID}. To achieve this,
a slicing ratio is defined, $\chi > 1$, which dictates the
left and right boundaries of a region outside of which points will be discarded:
\begin{subequations}
    \begin{gather}
        l_{\text{slice}} = \max\left(
            c - \left \lfloor \frac{b \chi}{2} \right \rfloor, 0
            \right),\\
        r_{\text{slice}} = \min\left(
            c + \left \lceil \frac{b \chi}{2} \right \rceil, 2N - 1
            \right).
    \end{gather}
\end{subequations}
The filtered spectrum is then sliced accordingly:
\begin{equation}
    \mathbb{R}^{r_{\text{slice}} - l_{\text{slice}}} \ni
    \widetilde{\symbf{s}}_{\text{VE,slice}} =
    \widetilde{\symbf{s}}_{\text{VE}}[l_{\text{slice}} : r_{\text{slice}} + 1]
    % \equiv
    % \begin{bmatrix}
    %     \widetilde{s}_{\text{VE},\hspace*{1pt}l_{\text{slice}}} &
    %     \widetilde{s}_{\text{VE},\hspace*{1pt}l_{\text{slice}} + 1} &
    %     \cdots &
    %     \widetilde{s}_{\text{VE},\hspace*{1pt}r_{\text{slice}}} &
    %     \widetilde{s}_{\text{VE},\hspace*{1pt}r_{\text{slice}} + 1}
    % \end{bmatrix}\T
    .
\end{equation}
Generation of the final sub-\ac{FID} is then achieved in a similar fashion to
before, by performing \ac{IFT}, and retaining the first half of the signal.
It is also necessary to scale the sub-\ac{FID} by the ratio of the number of
points in the sliced spectrum and it's unsliced counterpart, in order to ensure
that the amplitudes of each signal are unaffected:
\begin{subequations}
    \begin{gather}
        \widetilde{\by} =
            \frac{r_{\text{slice}} - l_{\text{slice}}}{2N}
            \IFT(\widetilde{\symbf{s}}_{\text{VE,slice}})
            [0 : N_{\text{slice}}],\\
            N_{\text{slice}} = \left \lfloor \frac{r_{\text{slice}} - l_{\text{slice}}}{2} \right \rfloor
    \end{gather}
\end{subequations}
The associated sweep width and transmitter offset of the \ac{FID} will have
been altered by this process, and in order to derive accurate frequencies and
damping factors for the sliced signal, it is necessary to determine these. The
corrected values can be computed using
\begin{subequations}
    \begin{gather}
        f_{\text{sw,slice}} = \frac{r_{\text{slice}} - l_{\text{slice}}}{2N - 1} \fsw\\
        f_{\text{off,slice}} = \foff + \frac{\fsw}{2} \left(
            1 - \frac{l_{\text{slice}} + r_{\text{slice}}}{2N - 1}
        \right)
    \end{gather}
\end{subequations}

\begin{algorithm}
    \begin{algorithmic}
        \caption[
            Filtering procedure for 1D data.
        ]
        {
            Filtering procedure for 1D data.
            $l_{\text{idx}}$ and $r_{\text{idx}}$ are indices of the left and
            right bounds of the region of interest.
            $l_{\text{idx,noise}}$ and $r_{\text{idx,noise}}$ are the analogous
            bounds for the noise region. All of these values should be $\in
            \lbrace 0, \cdots, 2N - 1 \rbrace$.
            These would typically be provided in units of \unit{\hertz} or
            \unit{\partspermillion} by a user; conversion to indices can
            be carried out using \cref{eq:fidx}.
        }
        \label{alg:filter-1d}
        \Procedure{Filter$1$D}{
            $\by \in \mathbb{C}^{N},
            l_{\text{idx}},
            r_{\text{idx}},
            l_{\text{idx,noise}},
            r_{\text{idx,noise}},
            \chi \in \mathbb{R}: \chi > 1
            $}
            \State $\by_{\text{VE}} \gets \textsc{VirtualEcho$1$D}\left(\by\right)$;
            \State $\symbf{s}_{\text{VE}} \gets \FT\left(\by_{\text{VE}}\right)$;
            \State $c_{\text{idx}} \gets \nicefrac{\left(l_{\text{idx}} + r_{\text{idx}}\right)}{2}$;
            \State $b_{\text{idx}} \gets r_{\text{idx}} - l_{\text{idx}}$;
            \State $\symbf{g} \gets \textsc{SuperGaussian$1$D}\left(2N, c_{\text{idx}}, b_{\text{idx}}\right)$;
            \State $\symbf{s}_{\text{noise}} \gets \symbf{s}_{\text{VE}} \left[
                l_{\text{idx,noise}} : r_{\text{idx,noise}} + 1
            \right]
            $;
            \State $\sigma^2 \gets \Var\left(\symbf{s}_{\text{noise}}\right)$;
            \State $\symbf{w}_{\sigma^2} \gets \symbf{0} \in \mathbb{R}^{2N}$;
            \For {$n = 0, \cdots, 2N - 1$}
            \State $w_{\sigma^2,n} \gets \textsc{RandomSample}\left(\mathcal{N}\left(0, \sigma^2\right)\right)$;
            \EndFor
            \State $\widetilde{\symbf{s}}_{\text{VE}} \gets \symbf{s}_{\text{VE}} \odot \symbf{g} + \symbf{w}_{\sigma^2} \odot \left(\symbf{1} - \symbf{g}\right)$;
            \State $l_{\text{slice}} \gets \max\left( c - \left \lfloor \frac{b \chi}{2} \right \rfloor, 0 \right)$;
            \State $r_{\text{slice}} \gets \min\left( c + \left \lceil \frac{b \chi}{2} \right \rceil, 2N - 1 \right)$;
            \State $\widetilde{\symbf{s}}_{\text{VE,slice}} \gets \widetilde{\symbf{s}} [l_{\text{slice}} : r_{\text{slice}} + 1]$;
            \State $\widetilde{\symbf{y}}_{\text{VE}} \gets
                \frac{r_{\text{slice}} - l_{\text{slice}}}{2N} \IFT \left(
                \widetilde{\symbf{s}}_{\text{VE,slice}} \right)$;
            \State $\widetilde{\symbf{y}} \gets \widetilde{\symbf{y}}_{\text{VE}}
                \left[:\left \lfloor \frac{r_{\text{slice}} - l_{\text{slice}}}{2}\right \rfloor \right]$;
            \State \textbf{return} $\widetilde{\symbf{y}}$;
        \EndProcedure
        \Statex
        \Procedure{VirtualEcho$1$D}{$\by \in \mathbb{C}^N$}
            \State \textbf{return} $
            \begin{bmatrix}
                \Re(y_0^{\vphantom{*}}) & y_1^{\vphantom{*}} & \cdots & y_{N-1}^{\vphantom{*}} & 0 & y_{N-1}^* & \cdots & y_1^*
            \end{bmatrix}\T
            $
        \EndProcedure
        \Statex
        \Procedure{SuperGaussian$1$D}{$N \in \mathbb{N}, c_{\text{idx}}, b_{\text{idx}}$}
            \State $\symbf{g} \gets \symbf{0} \in \mathbb{R}^{N}$;
            \For {$n = 0, \cdots, N - 1$}
                \State $g_n \gets \exp\left(
                    -2^{41} \left(
                        \frac{n - c_{\text{idx}}}{b_{\text{idx}}}
                    \right)^{40}
                    \right)
                $;
                \Comment{$p$ in \cref{eq:super-Gaussian-onedim} has been set to 40.}
            \EndFor
            \State \textbf{return} $\symbf{g}$
        \EndProcedure
    \end{algorithmic}
\end{algorithm}
