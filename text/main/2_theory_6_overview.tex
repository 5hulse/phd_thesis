\section{Summary}
The \ac{MPM} is well established as an effective for parametric estimation of
signals in a number of disciplines including \ac{NMR}.
Due to considerable advances in computational processing power since
its introduction, estimates of \acp{FID} can be acquired in reasonable times.
One notable downside of the technique that has been realised while assessing
its effectiveness in parametrising \ac{NMR} \acp{FID} is its propensity to
return oscillators with unexpected phase behaviour, especially in scenarios
involving signals with similar frequencies, exhibiting considerable overlap in
the Fourier domain.
For this reason, using the result of the \ac{MPM} as an initial guess to feed
into a phase-variance regularised \ac{NLP} routine is proposed as a means of
returning improved parameter estimates. The theory underpinning the procedure
has been explored in this chapter.

The computational burden of running the procedure is large and often
intractable for complete \ac{NMR} signals, which often comprise thousands of
points, and at least hundreds of contributing signals. This has been
illustrated through profiling both the \ac{CPU} time and the peak memory
consumption for the \ac{1D} and \ac{2D} methods. \note{Say more when
completed?} For this reason, a method to break the estimation problem into a
series smaller problems, through the construction of frequency-filtered
sub-\acp{FID}, is introduced.

Having established an estimation routine, the next chapter focusses on its
performance, as well as applications which are possible through parametric
estimation.
% Algorithm \ref{alg:1d-2d-summary} provides an overview of the principal steps
% involved in the \ac{1D} estimation procedure. Detailed algorithms for each step
% are presented elsewhere in this text.

% \begin{algorithm}
%     \begin{algorithmic}[1]
%         \caption[
%             An overview of the estimation procedure outlined in this work.
%         ]{
%             An overview of the estimation procedure outlined in this work, for
%             the consideration of \ac{1D} \acp{FID}.
%         }
%         \label{alg:1d-2d-summary}
%         \Procedure{Estimate$1$D}{$
%             \by \in \mathbb{C}^{\None},
%             \symbf{r}_{\text{interest}} \in \mathbb{R}^2,
%             \symbf{r}_{\text{noise}} \in \mathbb{R}^2,
%             M \in \mathbb{N}_0
%             $
%         }
%             \State $\widetilde{\by} \gets \textsc{Filter}1\textsc{D}\left(
%                 \by,
%                 \symbf{r}_{\text{interest}},
%                 \symbf{r}_{\text{noise}}
%                 \right)
%             $;
%             \Comment{Algorithm \ref{alg:filter-1d}}
%             \If{$M=0$}
%                 \State $M \gets \textsc{MDL}\left(\widetilde{\symbf{y}}\right)$;
%                 \Comment{\note{TODO}}
%             \EndIf
%             \State $\bthzero \gets \textsc{MPM}\left(\widetilde{\by}, M\right)$;
%             \Comment{Algorithm \ref{alg:mpm}}
%             \State $\bthstar, \symbf{\epsilon}^{(*)} \gets \textsc{NLP}\left(\widetilde{\by}, \bthzero\right)$;
%             \Comment{Algorithm \ref{alg:nlp}}
%             \State \textbf{return} $\bthstar, \symbf{\epsilon}^{(*)}$;
%         \EndProcedure
%         \Statex
%         \Procedure{Estimate$2$D}{$
%             \bY_{\cos} \in \mathbb{C}^{\None \times \Ntwo},
%             \bY_{\sin} \in \mathbb{C}^{\None \times \Ntwo},
%             \symbf{R}_{\text{interest}} \in \mathbb{R}^{2 \times 2},
%             \symbf{R}_{\text{noise}} \in \mathbb{R}^{2 \times 2},
%             M \in \mathbb{N}_0
%             $
%         }
%             \State $\widetilde{\bY} \gets \textsc{Filter}2\textsc{D}\left(
%                 \bY_{\cos},
%                 \bY_{\sin},
%                 \symbf{R}_{\text{interest}},
%                 \symbf{R}_{\text{noise}}
%                 \right)
%             $;
%             \Comment{Algorithm \ref{alg:filter-2d}}
%             \If{$M=0$}
%                 \State $M \gets \textsc{MDL}\left(\widetilde{\bY}[:, 0]\right)$;
%                 \Comment{Run the MDL on the first direct-dimension slice.}
%             \EndIf
%             \State $\bthzero \gets \textsc{MMEMPM}\left(\widetilde{\bY}, M\right)$;
%             \Comment{Algorithm \ref{alg:mmempm}}
%             \State $\bthstar, \symbf{\epsilon}^{(*)} \gets \textsc{NLP}\left(\widetilde{\bY}, \bthzero\right)$;
%             \Comment{Algorithm \ref{alg:nlp}}
%             \State \textbf{return} $\bthstar, \symbf{\epsilon}^{(*)}$;
%         \EndProcedure
%     \end{algorithmic}
% \end{algorithm}

