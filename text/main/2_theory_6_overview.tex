\section{Summary}
The matrix pencil-based methods described above are well established as
effective procedures for parametric estimation of time-domain signals in a
number of disciplines, including radar detection\cite{Hua1994},
acoustics\cite{TODO}, and \ac{NMR}\cite{Lin1997}, with specific application to
relaxometry being a recently introduced application \cite{Fricke2020,
Wortge2023}. Due to considerable advances computational processing power since
the introduction of the technique, estimates can be acquired from \ac{NMR}
signals in reasonable times. One notable downside of the technique that has
been realised while assessing its effectiveness in parametrising \ac{NMR}
\ac{FID}s is its propensity to return oscillators with unexpected phase
behaviour, especially in scenarios involving resonances with close frequencies.
For this reason, estimating phase-corrected \acp{FID}, using the result of the
\ac{MPM} as an initial guess to feed into a phase-variance penalised \ac{NLP}
routine was proposed as a means of improving parameter estimates. The theory
underpinning the procedure has been explored in this chapter.

The computational burden of running the procedure is large and often
intractable for complete \ac{NMR} signals, which often comprise thousands of
samples, and at least hundreds of contributing resonances. This has been
illustrated through profiling both the \ac{CPU} time and the peak memory
requirements for the \ac{1D} and \ac{2D} methods. \note{Say more when
completed?} For this reason, a method to break \acp{FID} into
frequency-filtered sub-signals, by filtering spectra derived using virtual
echoes, is introduced.  This method enables to formation of signals with far
fewer resonances and samples as compared to the full signal.

Algorithm \ref{alg:1d-2d-summary} provides an overview of the principal steps
involved in the \ac{1D} and \ac{2D} estimation procedures. Detailed algorithms
for each step are presented elsewhere in this text.

\begin{algorithm}
    \begin{algorithmic}[1]
        \caption[
            An overview of the estimation procedure outlined in this work.
        ]{
            An overview of the estimation procedure outlined in this work, for
            the consideration of \ac{1D} and \ac{2D} \ac{NMR} signals.
        }
        \label{alg:1d-2d-summary}
        \Procedure{Estimate$1$D}{$
            \by \in \mathbb{C}^{\None},
            \symbf{r}_{\text{interest}} \in \mathbb{R}^2,
            \symbf{r}_{\text{noise}} \in \mathbb{R}^2,
            M \in \mathbb{N}_0
            $
        }
            \State $\widetilde{\by} \gets \textsc{Filter}1\textsc{D}\left(
                \by,
                \symbf{r}_{\text{interest}},
                \symbf{r}_{\text{noise}}
                \right)
            $;
            \Comment{Algorithm \ref{alg:filter-1d}}
            \If{$M=0$}
                \State $M \gets \textsc{MDL}\left(\widetilde{\symbf{y}}\right)$;
                \Comment{\note{TODO}}
            \EndIf
            \State $\bthzero \gets \textsc{MPM}\left(\widetilde{\by}, M\right)$;
            \Comment{Algorithm \ref{alg:mpm}}
            \State $\bthstar, \symbf{\epsilon}^{(*)} \gets \textsc{NLP}\left(\widetilde{\by}, \bthzero\right)$;
            \Comment{Algorithm \ref{alg:nlp}}
            \State \textbf{return} $\bthstar, \symbf{\epsilon}^{(*)}$;
        \EndProcedure
        \Statex
        \Procedure{Estimate$2$D}{$
            \bY_{\cos} \in \mathbb{C}^{\None \times \Ntwo},
            \bY_{\sin} \in \mathbb{C}^{\None \times \Ntwo},
            \symbf{R}_{\text{interest}} \in \mathbb{R}^{2 \times 2},
            \symbf{R}_{\text{noise}} \in \mathbb{R}^{2 \times 2},
            M \in \mathbb{N}_0
            $
        }
            \State $\widetilde{\bY} \gets \textsc{Filter}2\textsc{D}\left(
                \bY_{\cos},
                \bY_{\sin},
                \symbf{R}_{\text{interest}},
                \symbf{R}_{\text{noise}}
                \right)
            $;
            \Comment{Algorithm \ref{alg:filter-2d}}
            \If{$M=0$}
                \State $M \gets \textsc{MDL}\left(\widetilde{\bY}[:, 0]\right)$;
                \Comment{Run the MDL on the first direct-dimension slice.}
            \EndIf
            \State $\bthzero \gets \textsc{MMEMPM}\left(\widetilde{\bY}, M\right)$;
            \Comment{Algorithm \ref{alg:mmempm}}
            \State $\bthstar, \symbf{\epsilon}^{(*)} \gets \textsc{NLP}\left(\widetilde{\bY}, \bthzero\right)$;
            \Comment{Algorithm \ref{alg:nlp}}
            \State \textbf{return} $\bthstar, \symbf{\epsilon}^{(*)}$;
        \EndProcedure
    \end{algorithmic}
\end{algorithm}

Having established an estimation routine, the next chapter focusses on its
performance, as well as applications which are possible through parametric
estimation.
