\chapter{Theory}
\label{chap:theory}

This chapter provides a description of the theory behind an estimation
routine which has been developed for the consideration of time domain \ac{NMR}
data.
In brief, the routine consists of generating a parameter estimate using the
\ac{SVD}-based \ac{MPM}, followed by the application of a \ac{NLP} routine to
produce a final result. The \ac{MPM} is employed to generate an initial
guess of parameters, while the \ac{NLP} routine behaves as a ``sanity
check''\note{better phase?} by attempting to make the parameter estimate more consistent with
known features of the data. The technique can be thought of as a compromise
between ``black-box'' methods which require little to no prior knowledge about
the data, and iterative methods like \ac{VARPRO} and \ac{AMARES}, which require
vast amounts of prior knowledge, but are typically able to estimate
complex datasets more effectively.

Furthermore, after profiling the run time and memory consumption of the
technique, a method for producing filtered \acp{FID}, featuring fewer signals
and (optionally) datapoints is presented, which can drastically reduce the
burden on computational resources.

Outlines of relevant algorithms and \Python implementations are
provided to supplement this chapter. As well as those present
within this chapter, more algorithms can be found in \cref{sec:algs}, while
\Python implementations can be found in \cref{chap:code-listings}.

\section{Outline of the Problem}
\label{sec:theory-outline}
For the purposes of this work, it is always assumed that an \ac{FID} to be
estimated
$\bY \in \mathbb{C}^{\None \times \cdots \times \ND}$
is hypercomplex in form, meaning that it obeys
\cref{eq:general-fid} with $\zeta^{(d)} = \exp(\iu \cdot)\ \forall d \in
\lbrace 1, \cdots D \rbrace$:
\begin{subequations}
    \begin{gather}
        \ynonenD = \xnonenD(\bth) + \wnonenD,\\
        \xnonenD(\bth) =
        \sumM \amexpphim
        \prodD \exp\left(\left(
            2 \pi \iu \left(f^{(d)}_m - \foffd\right)
            -\eta_m\right)
            \nd \Dtd\right),\label{eq:x}\\
        \wnonenD \sim \mathcal{N_C}\left(0, 2\sigma^2\right),%
    \end{gather}%
    \label{eq:hypercomplex-fid}%
\end{subequations}%
where $\Dtd = \nicefrac{1}{\fswd}$.
Under this model, it is assumed that
\iac{FID} consists of a summation of $M$ damped complex sinusoids in the
presence in \ac{AWGN}.
It is the goal of parametric estimation to establish the
identity of all the quantities which describe the model component $\bX$, which
are distilled into the vector $\bth \in \mathbb{R}^{2(D + 1)M}$, given by
\cref{eq:theta}.
\Cref{eq:x} can be expressed in terms of complex amplitudes and signals poles
as follows:
\begin{subequations}%
    \begin{gather}%
        \xnonenD(\bth) = \sumM \alpha_m \prodD {z^{(d)}_m}^{\nd},\\
        \alpha_m = \amexpphim,\\
        z_m^{(d)} = \exp\left(
            \left(2 \pi \iu \left(f_m^{(d)} - \foffd\right) - \eta^{(d)}_m\right) \Dtd
        \right).%
    \end{gather}%
    \label{eq:x-alpha-z}%
\end{subequations}%
Due to the assumed \ac{AWGN} nature of the noise array, the \ac{pdf} of an
individual noise component is
\begin{equation}
    p(\wnonenD) =
        \frac{1}{2\pi \sigma^2}
        \exp\left( -\frac{\left\lvert \wnonenD \right\rvert^2}{2\sigma^2}\right).
\end{equation}
As the elements are independent and identically distributed, the joint \ac{pdf}
describing the entire noise array is given by the product of each element's
\ac{pdf}:
\begin{equation}
    \begin{split}
        p\left(\bW\right) &=
            \prod_{\none=0}^{\None - 1}
            \cdots
            \prod_{\nD=0}^{\ND - 1}
            \frac{1}{2\pi \sigma^2}
            \exp\left(
                -\frac
                {\left\lvert \wnonenD \right\rvert^2}
                {2\sigma^2}\right) \\
            &= \frac{1}{\left(2\pi \sigma^2\right)^{\mathfrak{N}}}
            \exp\left( -\frac{\left\lVert \bW \right\rVert^2}{2\sigma^2}\right),
    \end{split}
\end{equation}
where $\mathfrak{N} \coloneq \None \times \cdots \times \ND$ is the total
number of points the \ac{FID} comprises.
As the noise array is the difference between the data and model, the
likelihood function of $\bth$ given $\bY$, $\mathcal{L}\left(\bth \vert
\bY\right)$, is
\begin{equation}
    \mathcal{L}\left(\bth \vert \bY\right) =
    \frac{1}{\left(2\pi \sigma^2\right)^{\mathfrak{N}}}
        \exp\left( -\frac{\left\lVert \bY - \bX(\bth) \right\rVert^2}{2\sigma^2}\right).
\end{equation}
It is common to consider instead the log-likelihood function,
$\ell\left(\bth \vert \bY\right)$. As application of the logarithm is a
monotonic transformation, the arguments of the maxima of $\mathcal{L}$ and
$\ell$ are equivalent.
\begin{equation}
    \ell\left(\bth \vert \bY\right) =
        -\mathfrak{N} \ln\left(2 \pi \sigma^2\right)
        -\frac{\left\lVert \bY - \bX(\bth) \right\rVert^2}{2\sigma^2}.
    \label{eq:log-likeihood}
\end{equation}
\Cref{eq:log-likeihood} implies that the optimal set of parameters
$\bth^{(*)}$, often referred to as the \acfi{MLE}\acused{MLE},
is that which minimises the
squared norm of the difference between the data and model, often call the
\acfi{RSS}:
\begin{equation}
    \bthstar = \argmax_{\bth \in \mathbb{R}^{2(D+1)M}}
        \ell\left(\bth \vert \bY\right) \equiv
        \argmin_{\bth \in \mathbb{R}^{2(D+1)M}} \left\lVert \bY - \bX(\bth) \right\rVert^2.
    \label{eq:argmin_y-x}
\end{equation}
The application of \ac{NLP} is a well-established approach to solve such a
problem\cite{Fletcher1987,Nocedal2006}. The basic principle behind \ac{NLP} is
to iteratively explore, in a methodical way, how a function varies with its
arguments. By using information about the function and optionally its
derivatives, such a routine attempts to find a minimum in the function, and
terminates once this has been achieved. While derivative-free approaches to
\ac{NLP} do exist\cite{Nelder1965,Kirkpatrick1983,Powell2009},
in scenarios where the function under consideration has well-defined,
computationally tractable derivatives, the use of these can be valuable to
solving optimisation problems; the problem outlined in
\cref{eq:argmin_y-x} is such an example.

As discussed already, for \ac{NLP} to perform effectively, a large amount of
\textit{a priori} information is typically required, in the form of an initial
guess.
To achieve this, the method employed in this work use the \ac{MPM}, the subject
of the next section.

\section{Generating an Initial Guess: Matrix Pencil Method}
\label{sec:mpm}
\note{Brief introductory text. Mention applications w. citations}

\subsection{1D Matrix Pencil Method}
\label{subsec:mpm}

\begin{remark}
    In contexts where \ac{1D} datasets are considered specifically, the
    redundant dimension index $^{(1)}$ will be neglected for conciseness.
\end{remark}
The \acfi{MPM}, developed by Hua and Sarkar\cite{Hua1990,Hua1990b,Hua1991}, provides a
route to extracting the signal poles of a \ac{1D} dataset, based on the
assumption that the number or oscillators $M$ is known.
To motivate how the \ac{MPM} works, first consider a dataset which is devoid of
noise, given by \cref{eq:x-alpha-z} with $D=1$:
\begin{equation}
    x_n(\bth) = \sumM \alpha_m^{\vphantom{n}} z_m^n,\\
\end{equation}
Consider the Hankel matrix $\Hx \in \mathbb{C}^{(N-L) \times (L+1)}$:
\begin{equation}
    \Hx =
    \begin{bmatrix}
        x_0 & x_1 & \cdots & x_L\\
        x_1 & x_2 & \cdots & x_{L+1}\\
        \vdots & \vdots & \ddots & \vdots\\
        x_{N-L-1} & x_{N-L} & \cdots & x_{N-1}
    \end{bmatrix}.
\end{equation}
This matrix comprises windowed segments of the FID, with each row comprising
the segment shifted to the right by one point relative to the row above.
$L \in \mathbb{N}$ is the \emph{pencil parameter}, which dictates the size of
each window. From $\Hx$, two matrices are defined, $\Hxone$ and $\Hxtwo$,
formed by the removal of the last or first column of $\Hx$, respectively:
\begin{subequations}
   \begin{gather}
        \Hxone =
        \begin{bmatrix}
            x_0 & x_1 & \cdots & x_{L-1} \\
            x_{1} & x_{2} & \cdots & x_{L} \\
            \vdots & \vdots & \ddots & \vdots\\
            x_{N-L-1} & x_{N-L} & \cdots & x_{N-2}\
        \end{bmatrix}, \\
        \Hxtwo =
        \begin{bmatrix}
            x_{1} & x_{2} & \cdots & x_{L} \\
            x_{2} & x_{3} & \cdots & x_{L+1} \\
            \vdots & \vdots & \ddots & \vdots\\
            x_{N-L} & x_{N-L+1} & \cdots & x_{N-1}\\
        \end{bmatrix}.
   \end{gather}
\end{subequations}
These matrices can be deconstructed into the following forms involving matrices
containing the $M$ signal poles and complex amplitudes that the data comprises:
\begin{subequations}
   \begin{gather}
       \Hxone = \symbf{Z}_{\text{L}} \symbf{A} \symbf{Z}_{\text{R}},\\
       \Hxtwo = \symbf{Z}_{\text{L}} \symbf{A} \symbf{Z}_{\text{D}} \symbf{Z}_{\text{R}},\\
       \mathbb{C}^{\left(\None - \Lone\right) \times M} \ni
       \symbf{Z}_{\text{L}} =
       \begin{bmatrix}
           \symbf{1} &
           \symbf{z} &
           \symbf{z}^2 &
           \cdots &
           \symbf{z}^{N-L-1}
        \end{bmatrix}\T,\\
        \mathbb{C}^{M \times L} \ni
        \symbf{Z}_{\text{R}} =
           \begin{bmatrix}
               \symbf{1} & \symbf{z} & {\symbf{z}}^{2} & \cdots & {\symbf{z}}^{L-1}
           \end{bmatrix} ,\\
        \mathbb{C}^{M \times M} \ni
        \symbf{Z}_{\text{D}} = \diag\left(\symbf{z}\right), \label{eq:ZD}\\
        \mathbb{C}^{M \times M} \ni
        \symbf{A} = \diag\left(\symbf{\alpha}\right),\label{eq:A}\\
        \symbf{\alpha} =
        \begin{bmatrix}
            \alpha_1 & \alpha_2 & \cdots & \alpha_M
        \end{bmatrix}\T,\\
        \symbf{z} =
        \begin{bmatrix}
            z_1 & z_2 & \cdots & z_M
        \end{bmatrix}\T.
   \end{gather}
    \label{eq:HX-decomp}
\end{subequations}
The \emph{matrix pencil} $\Hxtwo - \lambda\Hxone$, with $\lambda \in
\mathbb{C}$, can therefore be expressed as
\begin{equation}
    \Hxtwo - \lambda \Hxone = \symbf{Z}_{\text{L}} \symbf{A} \left(
        \symbf{Z}_{\text{D}} - \lambda \symbf{I}_M
    \right) \symbf{Z}_{\text{R}},
\end{equation}
where $\symbf{I}_M \in \mathbb{C}^{M \times M}$ is the identity matrix.
Assuming that the following condition is met:
\begin{equation}
    M \leq L \leq N - M,\label{eq:pencil_condition}
\end{equation}
the rank of the matrix pencil will be $M$. \cref{eq:pencil_condition}
must be obeyed to ensure that both the number of rows and columns of the matrix
pencil are at least $M$. Now consider the case when the scalar $\lambda$ is
equal to one of the signal poles i.e.  $\lambda = z_m\ \forall m \in
\lbrace 1, \cdots, M \rbrace$. The element $[\symbf{Z}_{\text{D}} -
\lambda \symbf{I}_M]_{m,m}$ will be set to $0$, which will lead to the
determinant of the matrix pencil being $0$. The eigenvalues of the matrix
pencil are the solution of the so-called \emph{generalised eigenvalue problem},
and are defined as\cite[Section 7.7]{Golub2013}
\begin{equation}
    \symbf{z} = \left\lbrace
        z \in \mathbb{C} : \det\left(\Hxtwo - z \Hxone\right) = 0
    \right\rbrace
\end{equation}
One means of finding the signal poles is by finding the eigenvalues of the
matrix $\Hxone^+ \Hxtwo^{\vphantom{+}}$. Deriving the corresponding complex
amplitudes can then be achieved by solving the set of linear equations
\begin{subequations}
    \begin{gather}
        \symbf{\alpha} = \symbf{Z}^+ \symbf{x},\\
        \symbf{Z} =
        \begin{bmatrix}
            \symbf{1} &
            {\symbf{z}} &
            {\symbf{z}}^2 &
            \cdots &
            {\symbf{z}}^{N-1}
        \end{bmatrix}\T.
    \end{gather}
    \label{eq:comp-amps}
\end{subequations}
Extraction of the amplitudes, phases, frequencies, and damping factors from the
signal poles and complex amplitudes can then take place:
\begin{subequations}
    \begin{gather}
        \symbf{a} = \left \lvert \symbf{\alpha} \right \rvert,\\
        \symbf{\phi} = \arctan \left(\frac{\Im (\symbf{\alpha})}{\Re(\symbf{\alpha})}\right),\\
        \symbf{f} = \frac{\fsw}{2 \pi} \Im\left(\ln \symbf{z} \right) + \foff, \\
        \symbf{\eta} = -\fsw \Re\left(\ln \symbf{z}\right).
    \end{gather}
\end{subequations}

\subsubsection{Noisy data}
The presence of noise in the signal complicates the process of
determining the $M$ signal poles, as $\Hy$\,---\,$\Hx$'s equivalent with
elements replaced by the noisy data $\symbf{y}$\,---\, is likely to be
full-rank, given by $\min(N - L, L + 1)$. To minimise the influence of noise on
the estimated signal poles, it is necessary to
generate a rank-reduced matrix $\Hytilde$. By employing the \ac{EYM}
theorem\cite[Section~2.2]{Golub2013}, an appropriate matrix is can be obtained
through \ac{SVD} (see \cref{subsec:linear-algebra}):
\begin{subequations}
    \begin{gather}
    \Hytilde =
        \symbf{U}_M^{\vphantom{\dagger}}
        \symbf{\Sigma}_M^{\vphantom{\dagger}}
        \symbf{V}_M^{\dagger},\\
    \mathbb{C}^{(\None - \Lone) \times M} \ni
        \symbf{U}_M^{\vphantom{\dagger}} =
        \begin{bmatrix}
            \symbf{u}_1 &
            \symbf{u}_2 &
            \cdots &
            \symbf{u}_M
        \end{bmatrix},\\
    \mathbb{C}^{(\Lone + 1) \times M} \ni
        \symbf{V}_M^{\vphantom{\dagger}} =
        \begin{bmatrix}
            \symbf{v}_1 &
            \symbf{v}_2 &
            \cdots &
            \symbf{v}_M
        \end{bmatrix},\\
    \mathbb{C}^{M \times M} \ni
        \symbf{\Sigma}_M^{\vphantom{\dagger}} =
        \diag \left( \sigma_1, \sigma_2, \cdots, \sigma_M \right).
    \end{gather}
\end{subequations}
$\sigma_m$ is the $m$ \textsuperscript{th} largest singular value of $\Hy$;
$\symbf{u}_m \in \mathbb{C}^{N - L}$ and $\symbf{v}_m \in
\mathbb{C}^{L+1}$ are the corresponding left and right singular vectors,
respectively. The \ac{EYM} proves that $\Hytilde$ is the closest matrix of rank
$M$ to $\Hy$ in a Frobenius norm sense, i.e.
\begin{equation}
    \Hytilde = \argmin_{\symbf{A}:\ \rank(\symbf{A}) = M} \left \lVert \symbf{A} - \Hy \right \rVert.
\end{equation}
With a rank-reduced matrix produced from the noisy matrix, the signal poles can
then be derived from the eigenvalues of $\Hytildeone^+
\Hytildetwo^{\vphantom{+}}$, where $\Hytildeone$ and $\Hytildetwo$ have the
same relation to $\Hytilde$ as  $\Hxone$ and  $\Hxtwo$ do to  $\Hx$. As a
less expensive alternative, the same result can be achieved by
computing the eigenvalues of $\symbf{V}_{M1}^+\symbf{V}_{M2}^{\vphantom{+}}$,
with
\begin{subequations}
    \begin{gather}
        \symbf{V}_{M1} =
        \begin{bmatrix}
            \symbf{v}_1 & \symbf{v}_2 & \cdots & \symbf{v}_{M-1}
        \end{bmatrix},\\
        \symbf{V}_{M2} =
        \begin{bmatrix}
            \symbf{v}_2 & \symbf{v}_3 & \cdots & \symbf{v}_{M}
        \end{bmatrix}.
    \end{gather}
\end{subequations}
\cref{alg:mpm} provides a pseudo-code description of the \ac{MPM}, while
\cref{lst:mpm} outlines a \Python implementation of it. For optimal
results, the pencil parameter should adhere to $\lfloor \nicefrac{N}{3} \rfloor
\leq L \leq \lfloor\nicefrac{2N}{3}\rfloor$\cite{Hua1990}. In this work,
$\lfloor\nicefrac{N}{3}\rfloor$ is always used, primarily since the computational
complexity of the method is at a maximum when $L = \nicefrac{N}{2}$\footnote{
    With $L = \lfloor\nicefrac{N}{2}\rfloor$, the matrix
    $\symbf{H}_{\symbf{Y}}$ is at its most ``square'' i.e. the number of rows
    and columns are at their most similar. Matrices which are more
    square will increase the demands on computing the \ac{SVD}, with a
    complexity $\mathcal{O}(\min(L+1,N-L+1)^2 \times \max(L+1,N-L+1))$.
}

\begin{algorithm}
    \caption[
        The \acl{MPM}, with the optional prediction of model order using the
        \acl{MDL}.
    ]
    {
        The \acs{MPM}, with the optional prediction of model order using the
        \acs{MDL}, if $M$ is set to $0$.
    }\label{alg:mpm}
    \begin{algorithmic}[1]
        \Procedure{MPM}{$\by \in \mathbb{C}^{N}, \fsw \in \mathbb{R}_{>0}, \foff \in \mathbb{R}, M \in \mathbb{N}_0$}
            \State $L \gets \left\lfloor \nicefrac{N}{3} \right\rfloor$;
            \State $\symbf{H}_{\symbf{y}} \gets
                \begin{bmatrix}
                    y_{0} & y_{1} & \cdots & y_{L}\\
                    y_{1} & y_{2} & \cdots & y_{L+1}\\
                    \vdots & \vdots & \ddots & \vdots\\
                    y_{N-L-1} & y_{N-L} & \cdots & y_{N-1}\\
                \end{bmatrix}
            $;
            \State $\symbf{U}, \symbf{\sigma}, \symbf{V}^{\dagger} \gets
            \SVD(\symbf{H}_{\symbf{y}})$;
            \If {$M = 0$}
                \State $M \gets \operatorname{MDL}(\symbf{\sigma}, L, N)$;
            \EndIf
            \State $\symbf{V} \gets \left[\symbf{V}^{\dagger}\right]^{\dagger}$;
            \State $\symbf{V}_M \gets \symbf{V}\left[:, :M\right]$;
            \Comment{Retain first $M$ right singular vectors.}
            \State $
                \symbf{V}_{M1}, \symbf{V}_{M2} \gets
                \symbf{V}_M\left[:,:M-1\right],
                \symbf{V}_M\left[:,1:\right]
            $;
            \Comment{Remove last/first column}
            \State $\symbf{z} \gets \textsc{Eigenvalues}\left(\symbf{V}_{M1}^+ \symbf{V}_{M2}^{\vphantom{+}}\right)$;
            \State $\symbf{Z} \gets
                \begin{bmatrix}
                    \symbf{1} & \symbf{z} & {\symbf{z}}^2 & \cdots & {\symbf{z}}^{N}
                \end{bmatrix}\T
            $;
            \State $\bdalpha \gets \symbf{Z}^+ \bY$;
            \State $
                \symbf{a} \gets \left\lvert\symbf{\alpha}\right\rvert$;
            \State $\symbf{\phi} \gets \arctan
                \left(\frac{\Im(\symbf{\alpha})}{\Re(\symbf{\alpha})}\right)
            $;
            \State $\symbf{f} \gets \frac{\fsw}{2\pi} \Im \left( \ln \symbf{z} \right) + \foff$;
            \State $\symbf{\eta} \gets -\fsw \Re \left( \ln \symbf{z} \right)$;
            \If {$\symbf{\eta}$ contains negative values}
            \Comment{Purge any oscillators with negative damping}
                \State Remove these from $\symbf{\eta}$, and remove the
                corresponding values from
                $\symbf{a}$, $\symbf{\phi}$, and $\symbf{f}$;
            \EndIf
            \State $\bthzero \gets
                \begin{bmatrix}
                    \symbf{a}\T &
                    \symbf{\phi}\T &
                    \symbf{f}\T &
                    \symbf{\eta}\T
                \end{bmatrix}\T
            $;
            \State \textbf{return} $\bthzero$;
        \EndProcedure
        \Statex
        \Procedure{MDL}{$\symbf{\sigma} \in \mathbb{R}^{L+1}, L \in \mathbb{N}, N \in \mathbb{N}$}
            \State $\operatorname{MDL}_{k-1} \gets \infty$
            \Comment{Ensure that on first iteration ($k=0$), the conditional does not hold}
                \For {$k = 0, \cdots, L$}
                    \State $\operatorname{MDL}_k \gets
                    -\ln\left(
                        \frac
                        {\prod_{r=k}^{L-1} \sigma_{r+1}^{\nicefrac{1}{L-k}}}
                        {\frac{1}{L-k} \sum_{r=k}^{L-1} \sigma_{r+1}}
                    \right)^{(L-k)N}$;
                    \If {$\operatorname{MDL}_k > \operatorname{MDL}_{k-1}$}
                        \State $M \gets k-1$;
                        \State \textbf{break};
                        \Else\State$\operatorname{MDL}_{k-1} \gets \operatorname{MDL}_k$;
                    \EndIf
                \EndFor
                \State \textbf{return} $M$;
        \EndProcedure
    \end{algorithmic}
\end{algorithm}


\subsection{2D Matrix Enhancement and Matrix Pencil Method}
\label{subsec:mmempm}
The \ac{MPM} was extended for the consideration of \ac{2D} data by Hua with the
\acfi{MEMPM}\cite{Hua1992}. The method centers around the enhanced matrix $\EY
\in \mathbb{C}^{\left(\Lone \Ltwo\right) \times \left(\None - \Lone +
1\right)\left(\Ntwo - \Ltwo + 1\right)}$, a block Hankel matrix of the form
\begin{subequations}
    \begin{gather}
        \EY =
        \begin{bmatrix}
            \symbf{H}_{\symbf{Y},0} & \symbf{H}_{\symbf{Y},1} & \cdots & \symbf{H}_{\symbf{Y},\None - \Lone} \\
            \symbf{H}_{\symbf{Y},1} & \symbf{H}_{\symbf{Y},2} & \cdots & \symbf{H}_{\symbf{Y},\None - \Lone + 1} \\
            \vdots & \vdots & \ddots & \vdots \\
            \symbf{H}_{\symbf{Y},\Lone - 1} & \symbf{H}_{\symbf{Y},\Lone} & \cdots & \symbf{H}_{\symbf{Y},\None - 1}
        \end{bmatrix}, \\
        \def\arraystretch{1.3}
        \symbf{H}_{\symbf{Y},\none} =
        \begin{bmatrix}
            y_{ \none, 0 } & y_{ \none, 1 } & \cdots & y_{ \none, \Ntwo - \Ltwo } \\
            y_{ \none, 1 } & y_{ \none, 2 } & \cdots & y_{ \none, \Ntwo - \Ltwo + 1 } \\
            \vdots & \vdots & \ddots & \vdots \\
            y_{ \none, \Ltwo - 1 } & y_{ \none, \Ltwo } & \cdots & y_{ \none, \Ntwo - 1 }
        \end{bmatrix}.
    \end{gather}
\end{subequations}
In a similar fashion to \cref{eq:HX-decomp},
$\symbf{H}_{\symbf{X},\none}$, the noiseless equivalent to
$\symbf{H}_{\symbf{Y},\none}$, can be expressed as
\begin{equation}
    \symbf{H}_{\symbf{X},\none} =
        \symbf{Z}^{(2)}_{\text{L}}
        \symbf{A}
        {\symbf{Z}^{(1)}_{\text{D}}}^{\none}
        \symbf{Z}^{(2)}_{\text{R}}.
\end{equation}
This then enables the noiseless enhanced matrix to be expressed as
\begin{subequations}
    \begin{gather}
        \symbf{E}_{\symbf{X}} =
        \symbf{E}_{\text{L}}
        \symbf{A}
        \symbf{E}_{\text{R}},\\
        \mathbb{C}^{\Lone \Ltwo \times M} \ni
        \symbf{E}_{\text{L}} =
        \begin{bmatrix}
            \symbf{Z}^{(2)}_{\text{L}} \\
            \symbf{Z}^{(2)}_{\text{L}} \symbf{Z}^{(1)}_{\text{D}} \\
            \vdots \\
            \symbf{Z}^{(2)}_{\text{L}} {\symbf{Z}^{(1)}_{\text{D}}}^{\Lone - 1} \\
        \end{bmatrix},\label{eq:EL}\\
        \mathbb{C}^{M \times \left(\None - \Lone + 1\right)\left(\Ntwo - \Ltwo + 1\right)} \ni
        \symbf{E}_{\text{R}} =
        \begin{bmatrix}
            \symbf{Z}^{(2)}_{\text{R}} &
            \symbf{Z}^{(1)}_{\text{D}} \symbf{Z}^{(2)}_{\text{R}} &
            \cdots &
            {\symbf{Z}^{(1)}_{\text{D}}}^{\None - \Lone} \symbf{Z}^{(2)}_{\text{R}} \\
        \end{bmatrix}.
    \end{gather}
\end{subequations}
As was the case in the \ac{1D} \ac{MPM}, \ac{SVD} can be utilised to generate a
filtered matrix $\EYtilde$ with its rank reducd to $M$, in accordance witht he
\ac{EYM} theorem:
\begin{equation}
    \EYtilde =
        \symbf{U}_M^{\vphantom{\dagger}}
        \symbf{\Sigma}_M^{\vphantom{\dagger}}
        \symbf{V}_M^{\dagger}
\end{equation}
If the conditions $\Nd - L^{(d)} + 1 \geq M\ \forall d \in \lbrace 1, 2
\rbrace$ are met, $\range\left(\symbf{U}_M\right) =
\range\left(\symbf{E}_{\text{L}}\right)$. This implies that there is some
nonsingular matrix $\symbf{T} \in \mathbb{C}^{M \times M}$ such that
\begin{equation}
    \symbf{U}_M = \symbf{E}_{\text{L}} \symbf{T}.
\end{equation}
Now consider the following two matrices:
\begin{subequations}
    \begin{gather}
        \symbf{U}_{M1} = \symbf{E}^{\vphantom{(1)}}_{\text{L}1} \symbf{T},\\
        \symbf{U}_{M2} = \symbf{E}^{\vphantom{(1)}}_{\text{L}1} \symbf{Z}^{(1)}_{\text{D}} \symbf{T},\\
        \mathbb{C}^{\Lone \left(\Ltwo - 1\right) \times M} \ni
        \symbf{E}_{\text{L}1} =
        \begin{bmatrix}
            \symbf{Z}^{(2)}_{\text{L}} \\
            \symbf{Z}^{(2)}_{\text{L}} \symbf{Z}^{(1)}_{\text{D}} \\
            \vdots \\
            \symbf{Z}^{(2)}_{\text{L}} {\symbf{Z}^{(1)}_{\text{D}}}^{\Lone - 2}
        \end{bmatrix}.
    \end{gather}
\end{subequations}
$\symbf{U}_{M1}$ and $\symbf{U}_{M2}$ correspond the $\symbf{U}_M$ with the
last and first $\Ltwo$ rows removed, respectively. The matrix pencil for
$\symbf{U}_{M1}$ and $\symbf{U}_{M2}$ can be expressed as
\begin{equation}
    \symbf{U}_{M1} - \lambda \symbf{U}_{M2} =
    \symbf{E}_{\text{L}1} \left( \symbf{Z}^{(1)}_{\text{D}} - \lambda \symbf{I}_M \right) \symbf{T}.
\end{equation}
As seen previously, this matrix structure implies that the elements of
$\bdzone$ are the solutions to the generalised eigenvalue problem, such that
they are the eigenvalues of $\symbf{U}_{M1}^{+} \symbf{U}_{M2}^{\vphantom{+}}$.

To extract the signal poles in the other dimension, $\bdztwo$, the permutation
matrix is defined:
\begin{equation}
    \mathbb{R}^{\Lone \Ltwo \times \Lone \Ltwo} \ni
    \symbf{P} =
    \begin{bmatrix}
        \symbf{e}\left(1\right)\T \\
        \symbf{e}\left(\Ltwo + 1\right)\T \\
        \vdots \\
        \symbf{e}\left(1 + \left(\Lone - 1\right)\Ltwo\right)\T \\
        \symbf{e}\left(2\right)\T \\
        \symbf{e}\left(2 + \Ltwo\right)\T \\
        \vdots \\
        \symbf{e}\left(2 + \left(\Lone - 1\right)\Ltwo\right)\T \\
        \vdots \\
        \vdots \\
        \symbf{e}\left(\Ltwo\right)\T \\
        \symbf{e}\left(2\Ltwo\right)\T \\
        \vdots \\
        \symbf{e}\left(\Lone \Ltwo\right)\T \\
    \end{bmatrix}.
\end{equation}
$\symbf{e}\left(i\right) \in \mathbb{R}^{\Lone \Ltwo}$ corresponds to a unit
vector comprising zeros except for $e_i = 1\ \forall i \in \lbrace 1, \cdots,
\Lone\Ltwo \rbrace$.
Multiplying $\symbf{E}_{\text{L}}$ by the permutation matrix leads to a matrix
in which the roles of the two sets of signal poles are effectively swapped:
\begin{equation}
    \symbf{E}_{\text{LP}} \coloneq \symbf{P} \symbf{E}_{\text{L}} =
    \begin{bmatrix}
        \symbf{Z}^{(1)}_{\text{L}} \\
        \symbf{Z}^{(1)}_{\text{L}} \symbf{Z}^{(2)}_{\text{D}} \\
        \vdots \\
        \symbf{Z}^{(1)}_{\text{L}} {\symbf{Z}^{(2)}_{\text{D}}}^{\Ltwo - 1} \\
    \end{bmatrix}.\label{eq:ELP}
\end{equation}
Note the similarity of \cref{eq:ELP} with \cref{eq:EL}, which
implies that with the same reasoning as given above, $\bdztwo$ can be derived
by extracting the eigenvalues of $\symbf{U}_{M\text{P}1}^+
\symbf{U}_{M\text{P}2}^{\vphantom{+}}$, where $\symbf{U}_{M\text{P}1}$ and
$\symbf{U}_{M\text{P}2}$ correspond to $\symbf{P} \symbf{U}_M$
with the last and first $\Lone$ rows removed, respectively.

In the original account on the \ac{MEMPM}, the final stage involved employing a
pairing algorithm in order to assign the uncorrelated signal poles in $\bdzone$
with $\bdztwo$\cite{Hua1992}. The \emph{modified} \ac{MEMPM} (\acs{MMEMPM}) was
developed in order to overcome two issues with the pairing algorithm: (a) it is
computationally expensive (b) it is prone to return incorrect
pairings\cite{Chen2007}.  As well as the eigenvalues of
$\symbf{U}^{+}_{M1}\symbf{U}_{M2}^{\vphantom{+}}$, the \ac{MMEMPM} requires
extraction of the eigenvectors too, which are contained in the matrix
$\symbf{W}^{(1)}$. Assuming that there are no repeated poles in $\bdzone$, the
correctly paired second set of poles is then generated via
\begin{equation}
    \bdztwo = \diag\left(
        \symbf{W}^{-1}
        \symbf{U}_{M\text{P}1}^+
        \symbf{U}_{M\text{P}2}^{\vphantom{+}}
        \symbf{W}
    \right)
\end{equation}
\note{Talk about case of paired eigenvalues.}

See \cref{alg:mmempm} for a pseudo-code outline, and \cref{lst:mmempm} for a
\Python implementation of the \ac{MMEMPM}.

\subsection{Model Order Selection}
\label{subsec:model-order}
The \ac{MPM} and \ac{MMEMPM} operate under the assumption that the model order
$M$ is known, or at least has been predicted.
It is possible that an individual inspecting the \ac{FID}'s spectrum could
predict $M$ based on the number of peaks visible, however subjective means of
predicting model order are typically viewed as disadvantageous as they have bias
associated with them.
There are various non-subjective criteria which have been established for
estimating the model order of a given signal, with probably the two most
prominent being the \ac{AIC}\cite{Akaike1974} and
\ac{MDL}\cite{Schwarz1978,Rissanen1978}. Both of these consider a family of
potential models which describe a given set of observations, parametrised by
the vector $\bth$. For the purpose of \ac{1D} \ac{FID} estimation, the family
of potential models comprise \cref{eq:x}, with variable $M$. Both the \ac{AIC}
and \ac{MDL} take the same general form:
\begin{equation}
    \mathcal{C}(k) = -c \ln \left(\mathcal{L} \left(\bthstar | \by \right)
    \right) + \mathcal{P}(k) \text{ with } \bthstar \in \mathbb{R}^{4k},
\end{equation}
$\forall k \in \lbrace 0, 1, \cdots \rbrace$. $\mathcal{L} \left(\bthstar |
\by \right)$ is the likelihood
function of a given model, with order $k$, at the \ac{MLE}
$\bthstar$, $c \in \mathbb{R}_{>0}$ is a scaling
constant, and $\mathcal{P}$ is a penalising function, which acts to correct
for bias. As the model order increases, the likelihood function at the \ac{MLE}
will increase in size, as a model with more parameters will be able to fit a
given dataset more accurately. However, as the model order increases, there will
become a point where practically all of the deterministic part of the signal
has been incorporated into the model, and increasing the model order further
leads to the model also accounting for noise. The penalising term, which
is larger for higher $k$, is required in order to estimate a model
order which is parsimonious. Wax and Kailath derived an expression for
the likelihood at the \ac{MLE} for models comprising a summation of
complex sinusoids\cite{Wax1985}\footnote{
    The expression in original paper considers the eigenvalues of the
    covariance matrix for the signal, rather than the singular values of $\Hy$.
    These are equivalent however.
}:
\begin{equation}
    \mathcal{L}\left(\bthstar \in \mathbb{R}^{4k} | \by\right) = \left(
        \frac{
            \prod_{r=k}^{L-1} \sigma_{r+1}^{\nicefrac{1}{L-k}}
        }{
            \frac{1}{L-k} \sum_{r=k}^{L-1} \sigma_{r+1}
        }
        \right)^{(L - k) N},
        \label{eq:wax-pdf}
\end{equation}
$\forall k \in \lbrace 0, 1, \cdots, L - 1 \rbrace$. $\symbf{\sigma} \in
\mathbb{R}^{\Lone}$ is the set of singular values of $\symbf{H}_{\symbf{y}}$,
in decreasing order. The forms of the \ac{AIC} and \ac{MDL} are given by
\begin{subequations}
    \begin{gather}
        \operatorname{AIC}(k) = -2 \ln\left( \mathcal{L} \left(\hat{\bth} | \by\right) \right) + 2k(2 L - k), \\
        \operatorname{MDL}(k) = -\ln\left( \mathcal{L} \left(\hat{\bth} | \by\right) \right) + \tfrac{1}{2} k(2 L - k) \ln N. \label{eq:mdl}
    \end{gather}
\end{subequations}
The \ac{AIC} has been shown to be inconsistent in that it tends to overestimate
the model order as the number of samples increases\cite{Wax1985}. For this
reason, the \ac{MDL} has found greater favour in signal processing
applications. As such, by default the estimation routine employed in this work
utilises the \ac{MDL}:
\begin{equation}
    M = \argmin_{k \in \mathbb{N}_0 :\ k < L} \operatorname{MDL} (k).
\end{equation}
\begin{figure}
    \centering
    \includegraphics{mdl/mdl.pdf}
    \caption[
        A visualisation of the behaviour of the \acs{MDL} for three different
        \acsp{FID} comprising the same deterministic component, but with
        different noise variances.
    ]{
        A visualisation of the behaviour of the \acs{MDL} for three different
        \acsp{FID} comprising the same deterministic component ($\bx$) but
        with different noise instances of differing variances. The model used
        to construct the \acp{FID} features 7
        signals. The three \acsp{SNR} used were
        \qty{7}{\deci\bel} (red), \qty{12}{\deci\bel} (blue), and
        \qty{20}{\deci\bel} (yellow). The \acsp{FID} were generated with $N
        = 256$.
        \textbf{a.} Spectra of the three \acsp{FID}.
        \textbf{b.} The values of the 14 most significant singular values
        associated with the Hankel matrix $\Hy$, with the
        pencil parameter $L$ set to $\lfloor \nicefrac{N}{3} \rfloor =
        85$.
        \textbf{c.} Square points with dotted lines: The negative log-likeliood
        at the \ac{MLE}, i.e. the first term of \cref{eq:mdl}.
        Grey line: the penalty component of the \ac{MDL}, given by the second
        term in \cref{eq:mdl}.
        Circular points with solid lines: the \ac{MDL}.
        Stars denote the minimum of the \ac{MDL} for a given \ac{FID}. The
        \qty{20}{\deci\bel}
        signal is correctly deemed to have a model order of 7, while the other
        two are underestimated (predicted models orders are 5 and 3 for the
        \qty{12}{\deci\bel} and \qty{7}{\deci\bel} \acsp{FID}, respectively).
    }
    \label{fig:mdl}
\end{figure}
Applying the \ac{MDL} for model order selection, and subsequently using the
\ac{MPM} for parameter estimation is the basis of the \ac{ITMPM}\cite{Lin1997}.
\Cref{fig:mdl} illustrates the form of the \ac{MDL} for three signals
with equivalent underlying models, with $M=7$, and noise instances with
different variances. The first 14 singular values of $\Hy$
are plotted in panel b, where it can be seen that beyond the first 7,
which account for signal components, the subsequent singular values, decrease
at a far slower rate. The noise subspace for \acp{FID} with higher \acp{SNR}
have singular values which are (a) smaller in magnitude and (b) more
consistent, such that distinguishing the noise and signal subspaces is an
easier task (cf. the yellow and red lines in panel b). As such, the
\ac{MDL} is more likely to provide a faithful estimate of the true number of
components in the \ac{FID} (panel c) when the \ac{SNR} is higher.

\note{Mention that multidimensional criteria are available, though not
implemented in this work.}

\section{\Acl{NLP}}
\label{sec:nlp}

\subsection{An overview of \ac{NLP}}
\label{subsec:nlp-overview}
In an optimisation problem, the goal is to determine the minimum\footnote{
    In certain applications, the interest could actually be to find the maximum of
    a function. However, it is trivial to transform a maximisation problem into
    a minimisation problem by finding the minimum of negative of the function.
}
of a function $\Fth: \mathbb{R}^n \rightarrow \mathbb{R}, n \in \mathbb{N}$, often
called the \emph{cost function} or \emph{fidelity}.
This is typically with the goal of determining the argument $\bthstar$ at
which the minimum is found:
\begin{equation}
    \bthstar = \argmin_{\bth \in \mathbb{R}^n} \Fth.
    \label{eq:minF}
\end{equation}
The above problem is \emph{unconstrained}, as there are no limitations that the
parameter vector is subjected to. Unless $\Fth$ has particular properties, such
as convexity\footnote{
    A convex function is one such that a line segment through any two points of
    the function lies above it.
}, it is generally only possible to determine a \emph{local} minimum,
rather than a \emph{global} minimum for high-dimensionality problems such as
the one of interest here. $\bthstar$ is a local
minimiser of $\Fth$ if there is a neighbourhood $V \ni \bthstar$ for which
\begin{equation}
    \Fthstar \leq \Fth\ \forall \bth \in V.
  \label{def:local-minimiser}
\end{equation}
$V \subset \mathbb{R}^n$ is a continuous space such that one can move some
amount in any direction away from $\bthstar$ and still be in $V$.
Key to \ac{NLP} are the \emph{necessary conditions}, which define whether a
given vector $\bth$ is a local minimum of the fidelity.
The \emph{first necessary condition} states
that if $\Fth$ is continuously differentiable, and $\bthstar$ is a local extremum\footnote{
    ``Extremum'' is used here instead of ``minimum'', as the first necessary
    condition applies to maxima of a function as well as minima.
} of $\Fth$, then the gradient vector $\bdgthstar \coloneq \nabla \Fthstar$ is the
zero vector:
\begin{equation}
    \bdgthstar = \symbf{0} \in \mathbb{R}^n
\end{equation}
The \emph{second necessary condition} subsequently states that
if $\Fth$ and $\bdgth$ are continuously differentiable, and $\bthstar$ is a
local minimiser of $\Fth$, then the Hessian matrix $\bdHthstar \coloneq
\nabla^2 \Fthstar$ is positive semidefinite, i.e.
\begin{equation}
  \symbf{v}^{\mathrm{T}} \bdHthstar \symbf{v} \geq 0\ \forall \symbf{v} \in \mathbb{R}^n.
\end{equation}
Furthermore, it is a \emph{unique} local minimiser if the \emph{second-order
sufficient condition} is also satisfied, i.e. that the Hessian is positive
definite:
\begin{equation}
    \symbf{v}^{\mathrm{T}} \bdHthstar \symbf{v} > 0\ \forall \symbf{v} \in \mathbb{R}^n.
\end{equation}

A plethora of approaches have been established to determine local minima of
scalar functions. One of the better-known strategies is \emph{Newton's method},
in which a quadratic approximation of the fidelity is considered.
For a given iteration $k \in \mathbb{N}_0$, the fidelity is approximated using
\begin{equation}
    \FQth =
        \Fthk +
        \symbf{h}\T \bdgthk +
        \tfrac{1}{2} \symbf{h}^{\mathrm{T}} \bdHthk \symbf{h},
    \label{eq:quad-approx}
\end{equation}
where $\symbf{h} = \bth - \bthk$.  An updated prediction of the parameter
vector is derived by finding the minimum of this quadratic approximation:
\begin{gather}
    \frac{\partial \Fth}{\partial \symbf{h}} =
        \bdgthk + \bdHthk \symbf{h} \notag\\
    \implies 0 = \bdgthk + \bdHthk \left(\bthkplusone - \bthk\right) \notag\\
    \therefore\ \bthkplusone =
        \bthk - \bdHthk^{-1}
        \bdgthk.\label{eq:newton-update}
\end{gather}
This process is repeated, until the convergence criterion as been met:
\begin{equation}
    \left\lVert \bdgthk \right\rVert \leq \epsilon.
\end{equation}
The convergence threshold $\epsilon > 0$ can be tuned based on the desired
accuracy of the result.
\Cref{eq:newton-update} tends not to be used as the update formula in real
optimisation problems; one of the major downsides of the Newton update is the
possibility that is not a minimising update if the Hessian is not positive
definite. Two primary strategies have emerged which are typically used instead:
\begin{itemize}
    \item \emph{Line search methods}\cite[Chapter 3]{Nocedal2006} determine an
        appropriate direction $\symbf{p}^{(k)}$ along which the updated
        parameter vector is sourced.  After this, an appropriate step length
        $\alpha^{(k)}$ is determined\,---\,typically in an efficient, though not
        optimal manner\,---\,leading to $\bthkplusone = \bthk - \alpha^{(k)}\symbf{p}^{(k)}$.
    \item \emph{Trust region methods}\cite[Chapter 4]{Nocedal2006} define a
        radius $\Updelta^{(k)} > 0$, and determine the minimum of
        \cref{eq:quad-approx} subject to the constraint that
        $\left\lVert\symbf{h}\right\rVert \leq \Updelta^{(k)}$.
\end{itemize}
A trust region method is applied in this work, and as such further
consideration of it will now be made.

\begin{algorithm}
    \caption[
        \acs{NLP} routine employed in this work.
    ]
    {
        \ac{NLP} routine employed in this work. This makes use of
        Algorithms 4.1 \& 7.2 in \cite{Nocedal2006}, with a extra check
        inserted to deal with any negative-amplitude oscillators which may be
        generated as the routine evolves
        (\crefrange{state:neg-amp-start}{state:neg-amp-end}).
    }
    \label{alg:nlp}
    \begin{algorithmic}[1]
        \Procedure {NLP}{$\bY \in \mathbb{C}^{\None \times \cdots \times \ND}, \bthzero \in \mathbb{R}^{2(D + 1)M}$}
            \State $\trustradius{0} \gets \nicefrac{1}{10} \left\lVert \bdgthzeroY \right\rVert$;
            \State $\trmax \gets 16 \trustradius{0}$;
            \For {$k = 0, 1, \cdots $}
            \State $\symbf{p}^{(k)} \gets \textsc{SteihaugToint}\left(\symbf{Y}, \symbf{\theta}^{(k)}, \trustradius{k}\right)$;
                \Comment{See \cref{alg:steihaug-toint}}
                \State $\rho^{(k)} \gets
                    \frac
                        {\Fphithk - \Fphithkpk}
                        {\FphiQthk - \FphiQthkpk}$;
                \If {$\rho_k < \nicefrac{1}{4}$}
                \label{state:decrease-tr-start}
                \State $\trustradius{k+1} \gets \nicefrac{1}{4} \trustradius{k}$;
                    \label{state:decrease-tr-end}
                    \ElsIf {$\rho_k > \nicefrac{3}{4}$ \textbf{ and } $\left\lVert \symbf{p}^{(k)} \right\rVert = \trustradius{k}$}
                \label{state:increase-tr-start}
                \State $\trustradius{k+1} \gets \min\left(2 \trustradius{k}, \trmax\right)$;
                    \label{state:increase-tr-end}
                \Else
                \State $\trustradius{k+1} \gets \trustradius{k}$;
                \EndIf
                \If{$\rho^{(k)} > \nicefrac{3}{20}$}
                \label{state:large-rho-start}
                    \State $\bthkplusone \gets \bthk + \symbf{p}^{(k)}$;
                    \label{state:large-rho-end}
                \Else
                \label{state:small-rho-start}
                    \State $\symbf{\theta}^{(k+1)} \gets \symbf{\theta}^{(k)}$;
                    \label{state:small-rho-end}
                \EndIf
                \If{$k \bmod 25 = 0 \textbf{ and } \symbf{\theta}^{(k+1)}$ contains negative amplitudes}\label{state:neg-amp-start}
                    \State $\symbf{\theta}^{(0)} \gets \symbf{\theta}^{(k+1)}$ with negative-amplitude oscillators removed;
                    \State $\symbf{\theta}^{(*)}, \symbf{\epsilon}^{(*)} \gets \operatorname{NLP}\left(\symbf{Y}, \symbf{\theta}^{(0)}\right)$;
                \EndIf\label{state:neg-amp-end}
                \If{$\left\lVert \bdgthkplusone \right\rVert < \num[print-unity-mantissa=false]{1e-8}$}
                    \State \textbf{break};
                \EndIf
            \EndFor
            \State $\symbf{\theta}^{(*)} \gets \symbf{\theta}^{(k+1)}$
            \State $\symbf{\epsilon}^{(*)} \gets
                \sqrt{
                    \frac
                    {
                        \Fthstar \diag \left(
                            \left[\bdHthstar\right]^{-1}
                        \right)
                    }
                    {(\None \cdots \ND) - 1}
                }$
            \State \textbf{return} $\symbf{\theta}^{(*)}, \symbf{\epsilon}^{(*)}$;
        \EndProcedure
    \end{algorithmic}
\end{algorithm}
The structure of a typical trust region method is presented in
\cref{alg:nlp} (ignoring \crefrange{state:neg-amp-start}{state:neg-amp-end}, which is a custom addition,
see \cref{subsec:phase-variance}). An initial radius for the trust region
$\trustradius{0}$ is defined, along with a maximum permitted radius
$\trmax$, to ensure that excessively adventurous steps do not take place.
For each iteration, a solution to the following sub-problem is sought:
\begin{equation}
    \begin{split}
        \bpk = \argmin_{\bp \in \mathbb{R}^{n}}
            \Fthk +
            (\bthk + \bp)\T \bdgthk +
            \tfrac{1}{2} (\bthk + \bp)\T \bdHthk (\bthk + \bp) \\
        \text{subject to } \left \lVert \bp \right \rVert \leq \trustradius{k}.
    \end{split}
\end{equation}
This sub-problem is not usually minimised exactly, but instead an efficient
means of determining a sufficiently good update is used.
Common approaches include computing the Cauchy point, the Dogleg
method, and a truncated conjugate-gradient
approach commonly called the \ac{ST} method\cite[Chapter 7]{Nocedal2006}.
The latter is employed in this work (see \cref{alg:steihaug-toint} and
\cref{lst:tr}). In the \ac{ST} approach, iterates of the conjugate-gradient
method\cite[Chapter 5]{Nocedal2006} are computed, either until an iterate which
is outside the trust region is computed, or negative curvature is discovered.

Once a provisional update $\bthkplusone = \bthk + \bpk$ is determined using the
\ac{ST} method, a metric is considered which indicates how well the
quadratic estimate agrees with the true value of the fidelity:
\begin{equation}
    \rho^{(k)} = \frac
        {\Fthk - \Fthkpk}
        {\FQthk - \FQthkpk}.
\end{equation}
$\rho^{(k)}$ is the ratio between the actual reduction of the fidelity caused
by taking the proposed step, and the predicted reduction based on the quadratic
model. If $\rho^{(k)}$ is sufficiently close to $1$, the quadratic model being
used to generate new iterates is deemed to be acting well enough to warrant
accepting the proposed update
(\cref{state:large-rho-start,state:large-rho-end}).
Furthermore, if $\rho^{(k)}$ is particularly close to 1, and the proposed
update is at the boundary of the trust radius, it is appropriate to enlarge the
radius of the trust region for the next iteration in an attempt to increase the
rate of convergence
(\cref{state:increase-tr-start,state:increase-tr-end}).
On the other hand, a small value of $\rho^{(k)}$ implies that the
quadratic model reflects the true fidelity poorly, such that the proposed
update should be rejected
(\cref{state:small-rho-start,state:small-rho-end}).
As well as this, the trust region's radius should be
decreased such that the model is more likely to behave faithfully
(\cref{state:decrease-tr-start,state:decrease-tr-end}). In general, the
thresholds which dictate whether to accept an update, and whether to adjust the
trust region radius are customisable. The hard-coded numerical values found in
\cref{alg:nlp} are the values used for the results presented in this work.

\subsection{\acs{NLP} applied to \acs{FID} estimation}
Focus now turns to the specific problem of FID estimation using \ac{NLP}, for
which a general $D$-dimensional dataset will be considered. As established in
\cref{sec:theory-outline}, the fidelity $\FthY : \mathbb{C}^{\None \times
\cdots \times \ND} \times \mathbb{R}^{2(1 + D)M} \rightarrow \mathbb{R}$ is
given by
\begin{equation}
    \FthY = \left \lVert \bY - \bXth \right \rVert^2.
    \label{eq:fidelity}
\end{equation}
The elements of the gradient vector $\bdgthY \in \mathbb{R}^{2(1+D)M}$ and
the Hessian matrix $\bdHthY \in \mathbb{R}^{2(1+D)M \times 2(1+D)M}$ are
derived by taking the first and second partial derivatives of the fidelity with
respect to the elements in $\bth$:
\begin{subequations}
    \begin{gather}
        g_i = -2 \Re
                \left\langle
                    \left(\bY - \bX\right),
                    \frac{\partial \bX}{\partial \theta_i}
                \right\rangle,
        \label{eq:grad} \\
        h_{i,j} = 2 \Re
            \biggl(
                \underbrace{
                    \left\langle
                        \frac{\partial \bX}{\partial \theta_i},
                        \frac{\partial \bX}{\partial \theta_j}
                    \right\rangle
                }_{\circled{1}}
                -
                \underbrace{
                    \left\langle
                        \left(\bY - \bX\right),
                        \frac{\partial^2 \bX}{\partial \theta_i \partial \theta_j}
                    \right\rangle
                }_{\circled{2}}
            \biggl),
            \label{eq:hess}
    \end{gather}
    \label{eq:fidelity-grad-hess}%
\end{subequations}
$\forall i,j \in \lbrace 1, \cdots, 2(1+D)M \rbrace$.
The complete set of first and second derivatives of a particular element of the
model $x \coloneq \xnonenD$, given by \cref{eq:x}, is as follows
$\forall m \in \lbrace 1, \cdots, M \rbrace$,
$\forall d, d^{\prime} \in \lbrace 1, \cdots, D \rbrace$:
\paragraph{First dreivatives}
\begin{subequations}
    \begin{gather}
        \xderiv{\theta_m} \equiv
            \xderiv{a_m} =
            \frac{x}{a_m},
            \label{eq:deriv-a}\\
        \xderiv{\theta_{m + M}} \equiv
            \xderiv{\phi_m} =
            \iu x,\\
        \xderiv{\theta_{m + (d + 1)M}} \equiv
            \xderiv{\fdm} =
            2 \pi \iu \Dtd \nd x,\\
        \xderiv{\theta_{m + (d + D + 1)M}} \equiv
            \xderiv{\etadm} =
            - \Dtd \nd x.
    \end{gather}
    \label{eq:first-derivs}
\end{subequations}
\paragraph{Second derivatives}
\begin{subequations}
    \begin{gather}
        \xderivtwosame{\theta_{m}} \equiv
            \xderivtwosame{a_m} =
            0,
            \label{eq:amp-second-deriv}\\
        \xderivtwodiff{\theta_{m}}{\theta_{m + M}} \equiv
            \xderivtwodiff{a_m}{\phi_m} =
            \frac{\iu x}{a_m},\\
        \xderivtwodiff{\theta_{m}}{\theta_{m + (d + 1)M}} \equiv
            \xderivtwodiff{a_m^{\vphantom{(d)}}}{\fdm} =
            \frac{2 \pi \iu \Dtd \nd x}{a_m},\\
        \xderivtwodiff{\theta_{m}}{\theta_{m + (d + D + 1)M}} \equiv
            \xderivtwodiff{a_m^{\vphantom{(d)}}}{\etadm} =
            \frac{-\Dtd \nd x}{a_m},\\
        \xderivtwosame{\theta_{m + M}} \equiv
            \xderivtwosame{\phi_m} =
            -x,\\
        \xderivtwodiff{\theta_{m + M}}{\theta_{m + (d + 1)M}} \equiv
            \xderivtwodiff{\phi_m^{\vphantom{(d)}}}{\fdm} =
            -2 \pi \Dtd \nd x,\\
        \xderivtwodiff{\theta_{m + M}}{\theta_{m + (d + D + 1)M}} \equiv
            \xderivtwodiff{\phi_m^{\vphantom{(d)}}}{\etadm} =
            -\iu \Dtd \nd x,\\
        \xderivtwodiff{\theta_{m + (d + 1)M}}{\theta_{m + (d^{\prime} + 1)M}} \equiv
            \xderivtwodiff{\fdm}{\fdmp} =
            -4\pi^2 \left(\Dtd \nd \right) \left(\Dtdp \ndp \right) x,\\
        \xderivtwodiff{\theta_{m + (d + 1)M}}{\theta_{m + (d^{\prime} + D + 1)M}} \equiv
            \xderivtwodiff{\fdm}{\etadmp} =
            -2 \pi \iu \left(\Dtd \nd \right) \left(\Dtdp \ndp \right) x,\\
        \xderivtwodiff{\theta_{m + (d + D + 1)M}}{\theta_{m + (d^{\prime} + D + 1)M}} \equiv
            \xderivtwodiff{\etadm}{\etadmp} =
            \left(\Dtd \nd \right) \left(\Dtdp \ndp \right) x,\\
        \xderivtwodiff{\theta_{i}}{\theta_{j}} =
            \xderivtwodiff{\theta_{j}}{\theta_{i}},
            \label{eq:symmetric-second-derivs}\\
        \xderivtwodiff{\theta_{i}}{\theta_{j}} = 0\ \text{ if not specified above.}
        \label{eq:zero-second-deriv}
    \end{gather}
    \label{eq:second-derivs}
\end{subequations}
\Cref{eq:zero-second-deriv} indicates that any second derivative
with respect to two parameters which do not belong to the same oscillator in
then model will always be $0$. This, along with the symmetrical nature of the
second derivatives (\cref{eq:symmetric-second-derivs}) drastically reduces the
required number to explicitly compute, from $4 (1 + D)^2 M^2$ per data-point
to  $(1+D)\left(3 + 2D\right)M$. Finally, \cref{eq:amp-second-deriv}
indicates that another $M$ second derivatives do not need to be computed, as
they are always $0$. See \cref{tab:number-of-derivatives} for the total
number of derivatives that need to be computed for datasets with different
numbers of dimensions.
\begin{table}
    \begin{center}
        \begin{tabular}{ c c c }
            \toprule
            Dimensions &
                \raisebox{\depth}{\#} 1\textsuperscript{st} derivatives &
                \raisebox{\depth}{\#} 2\textsuperscript{nd} derivatives\\
            \midrule
            $1$ & $4M\None$ & $9M\None$\\
            $2$ & $6M\None\Ntwo$ & $20M\None\Ntwo$\\
            $3$ & $8M\None\Ntwo\Nthree$ & $35M\None\Ntwo\Nthree$\\
            $D$ &  $2(1 + D)M \mathfrak{N}$ &  $((1 + D)(3 + 2D) - 1)M \mathfrak{N}$\\
            \bottomrule
        \end{tabular}
    \end{center}
    \caption{
        The number of first and second derivatives that are necessary to
        compute the gradient vector and Hessian matrix of the fidelity for
        1- 2- and 3-dimensional datasets, as well as a general $D$-dimensional
        dataset.
    }
    \label{tab:number-of-derivatives}
\end{table}

\subsection{Approximating the Hessian}
\label{subsec:hess-approx}
Despite many of the model second derivatives being $0$, computation of those
that are not zero, and subsequently using these the form the Hessian matrix,
can be an expensive part of the optimisation routine.
Numerous optimisation problems exist where this is the case, and as such
there is considerable precedent for improving the efficiency of optimisation
algorithms by generating approximations of the Hessian which are demanding to
compute.
Examples include the \ac{GN} method and \ac{LM} algorithm,
which are specifically for \ac{RSS} problems\cite[Chapter
10]{Nocedal2006}, as well as quasi-Newton methods such as the \ac{BFGS}
method\cite[Chapter 6]{Nocedal2006}.

The \ac{GN} and \ac{LM} approaches replace the true Hessian matrix at each
iteration with the following expression:
\begin{equation}
    h_{i,j} = 2 \Re
        \left\langle
            \frac{\partial \bX}{\partial \theta_i},
            \frac{\partial \bX}{\partial \theta_j}
        \right\rangle,
    \label{eq:hess-approx}
\end{equation}
i.e. term \circled{2} in \cref{eq:hess}, which involves the model second
derivatives, is neglected. All that needs to be generated is the Jacobian
$\symbf{J} = \nicefrac{\partial \bX}{\partial \bth}$. This can
bring a very large reduction in the computational cost, as no extra
derivatives need to be computed for the Hessian at all, since the Jacobian is
already required for generating the gradient vector.
In situations where the residuals between the data and model are small, term
\circled{1} will tend to dominate term \circled{2}, and as such these methods
often enjoy a convergence rate which is comparable to that of Newton's method
when close to local minima. Despite this, by invoking this approximation, the
rate of convergence, i.e. the number of iterations required to reach
$\bthstar$, tends to be adversely affected. See \cref{subsec:optim-vis} for an
example of this phenomenon.

\subsection{Visualisation of a simple example}
\label{subsec:optim-vis}
\begin{figure}
    \centering
    \includegraphics{optimisation_visualisation/optimisation_visualisation.pdf}
    \caption[
        A visualisation of the trajectory of a 2-parameter optimisation
        involving a simulated \acs{FID} comprising a single signal.
    ]
    {
        A visualisation of the trajectory of a 2-parameter optimisation
        involving a simulated \acs{FID} comprising a single signal.
        \textbf{a.} \& \textbf{b.} Representations of the signal in
        the time domain and Fourier domain, respectively.
        Black dots: the signal to be estimated $\by$.
        Solid grey line: the model generated
        using the initial guess $\bx \left( \bthzero \right)$.
        Dotted grey line: the model generated using the optimised result, $\bx
        \left( \bthstar \right)$.
        \textbf{c.} A contour plot of the fidelity.
        Blue line: the trajectory of the parameter vector with the true
        Hessian matrix used in computing each update.
        Red line: the analogous trajectory using the Hessian approximation
        in place of the true Hessian.
    }
    \label{fig:optim-vis}
\end{figure}
\Cref{fig:optim-vis} provides a visual example of the application of \ac{NLP}
to estimate a simulated \ac{1D} \ac{FID} comprising a single signal.
The FID was constructed using \cref{eq:general-fid} with $M=1$,
$N = 64$, $\fsw = \qty{5.2}{\hertz}$ ($\Dt \approx
\qty{0.192}{\second}$), and $\foff = \qty{0}{\hertz}$.
The signal was parameterised by $\bth \in \mathbb{R}^4$ comprising $a=1$,
$\phi=\qty{0}{\radian}$, $f=\qty{1}{\hertz}$, $\eta=\qty{0.2}{\per\second}$.
\ac{AWGN} was added to the \ac{FID} to give it \iac{SNR} of approximately
\qty{10}{\deci\bel}. As the visualisation of 5D space is beyond the scope of
this work, only two parameters, the frequency and damping factor, were optimised
from an initial guess $\bthzero$; the amplitude and phase were fixed to
their true values throughout. The initial guess comprised a frequency of
\qty{1.1}{\hertz}, and a damping factor of \qty{0.8}{\per\second}, with the
solid grey lines in panels a \& b denoting the model generated in the time- and
Fourier-domains, respectively. $\bthzero$ was subjected to \ac{NLP} twice. In
the first instance, the exact Hessian matrix, given by \cref{eq:hess} was used
in order to compute each update step, while in the second the Hessian
approximation given by \cref{eq:hess-approx} was used. The initial
radius of the trust region was set to $\nicefrac{1}{10}$ of the gradient norm
($\approx 0.3$), which has a precedent in the literature\cite{Gould2005}. The
trajectories of the parameter vector are denoted by coloured lines in panel c.
In both cases, the \ac{NLP} routine successfully converged at a result $\bthstar$
in agreement with the true frequency and damping factor used to construct the
\ac{FID}. However, it is clear that using the true Hessian matrix (blue)
led to a far better rate of convergence compared with the
approximated analogue (red), which exhibited ``zig-zagging''\footnote{
    This phenomenon is often seen in gradient descent methods, in which each
    update occurs along the opposite direction to the gradient.
}.
14 iterations were required to reach the
convergence criterion $\epsilon \leq \num[print-unity-mantissa=false]{1e-8}$
when the true Hessian was used, while 81 were required for the approximated
case. Despite being an anecdotal example, this highlights that use of the true
Hessian matrix tends to allow a better rate of convergence. However, for
\acp{FID} comprising many signals and far more points, the approximated form
often requires a shorter time to converge overall, especially for \ac{2D}
\acp{FID}, as will be illustrated in \cref{sec:profiling}.

\subsection{Phase Variance Minimisation}
\label{subsec:phase-variance}
\Iac{NLP} procedure tasked with minimising the discrepancy between the model
$\bX$ and the observed data $\bY$ is well-suited to produce an accurate
holistic
representation of the data, assuming a sufficiently large
model order is used. However, as has already been discussed, this is not
necessarily sufficient due to the ill-posed nature of the estimation problem.
It is desirable to produce an estimate which not only achieves a good fit to
the data in \iac{RSS} sense, but which is also in agreement with the process
underpinning the observation. It is for this reason that iterative procedures
typically require significant quantities of prior knowledge, beyond basic
assumptions of the underlying model, in order to produce meaningful estimation
results.  This is also why they are often able to produce results which agree
better with a spectroscopist's conception of what the ``correct'' parameter
estimate should look like, relative to other methods where such detailed
information is not exploited.

While the \ac{MPM} is often able to generate seemingly reasonable parameter
estimates,
one particular feature has been noticed in many of its results:
often, oscillators in the result exhibit spurious phase behaviour.
As has been discussed (\cref{subsec:nmr-proc}),
\ac{NMR} datasets for most experiments comprise signals whose phases
depend on their resonance frequencies to first order. This is routinely
corrected in conventional \ac{NMR} spectral processing, such that all signals
are adjusted to acquire a phase of \qty{0}{\radian}. This feature of the
dataset can be exploited in order to overcome the aforementioned shortcoming of
the \ac{MPM}, through appropriate regularisation of the \ac{NLP} routine.
Assuming that the data has been phase corrected\footnote{
    Rather than rely on the data being phase-corrected, one could envisage
    replacing the phase variance with a term which guides the
    oscillators to adopt a first-order phase relationship. The reason why the
    phase variance has been chosen is two-fold: (i) Applying phase-correction
    to \ac{NMR} data is straightforward and can be automated,
    meaning the user would experience minimal burden. (ii) As will be discussed
    in \cref{sec:filtering}, it is beneficial to have a spectrum comprising
    pure absorption-mode Lorentzians in order to produce frequency-filtered
    ``sub-\acp{FID}'' from the original data, so the data being estimated will
    be phase-corrected anyway.
}, incorporating the variance of oscillator
phases into the fidelity can lead to improved estimation results; examples of
this will be provided later (\cref{sec:evaluation}).
The updated fidelity becomes
\begin{equation}
    \FphithY = \left \lVert \bY - \bXth \right \rVert^2 + \circvar,
    \label{eq:fidelity-phasevar}
\end{equation}
where $\circvar$ is the \emph{circular variance} of the oscillator phases.
Oscillator phases are an example of a circular variable, as all
phases are wrapped within an interval of size 2$\pi$\,\unit{\radian}. Given an
unwrapped phase $\widetilde{\phi} \in \mathbb{R}$, the
corresponding wrapped phase $\phi \in \left( -\pi, \pi \right]$ is given by
\begin{equation}
    \phi = (\widetilde{\phi} + \pi) \bmod 2 \pi - \pi.
    \label{eq:phase_wrap}
\end{equation}
This makes the conventional (linear)
definition of variance, given by
\begin{subequations}
    \begin{gather}
        \Var_{\shortmid}\hspace*{-3pt}\left(\symbf{\phi}\right) =
            \frac{1}{M} \sum_{m=1}^{M} \left(\phi_m - \mu\left(\symbf{\phi}\right)\right)^2, \\
        \mu\left(\symbf{\phi}\right) = \frac{1}{M} \sum_{m} \phi_m,
    \end{gather}
\end{subequations}
unsuitable as a metric to define the variation in phases. Consider as a simple
example a scenario
where there are two oscillators with phases $\widetilde{\bdphi} = \left[ \pi +
\delta\:\:\pi - \delta \right]\T$ for some small $\delta$.
The phase variance is expected to be small as the phases are similar.
However, with the inclusion of wrapping through application of
\cref{eq:phase_wrap}, these phases would actually be set to $\bdphi = \left[
    -\pi
+ \delta\:\:\pi - \delta \right]\T$, and the linear phase
variance would be large. It is therefore apparent that a definition of variance
which accounts for the periodicity of the phases is needed. The circular
variance is defined as\cite[Chapter 3]{Fisher1993}
\begin{subequations}
    \begin{gather}
        [0, 1] \ni \circvar = 1 - \frac{R}{M},\\
        R = \sqrt{c_{\Sigma}^2 + s_{\Sigma}^2}, \\
        c_{\Sigma} = \sum_{m=1}^M \cos \phi_m, \\
        s_{\Sigma} = \sum_{m=1}^M \sin \phi_m.
    \end{gather}
\end{subequations}
$R$ is the length of the resultant vector produced by summing $M$ unit vectors
with angles given by $\bdphi$. In the case that all the vectors have the
same angle, $R=M$, leading to the variance being $0$ as expected. At the other
extreme, with $M$ vectors uniformly separated about the unit circle\,---\,such
that there is an angle of $\nicefrac{2 \pi}{M - 1}$\,\unit{\radian} between all
pairs of adjacent vectors\,---\,the
vectors will perfectly cancel, leading to $R=0$. In this case, the maximum
variance of $1$ is obtained.
The inclusion of the phase variance into the fidelity is one of the
motivating reasons for normalising the data prior to estimation (see
\cref{rem:norm-data}). Since $\circvar$ is constrained to the interval $[0,
1]$, if the data were not normalised it is likely that $\lVert \bY - \bX
\rVert^2$ would dominate $\circvar$ in \cref{eq:fidelity-phasevar}, such that
the influence of the phase variance would be negligible.

The first and second derivatives of the circular variance are required for
the computation of the gradient vector and Hessian matrix, whose updated forms
are
\begin{subequations}
    \begin{gather}
        g_i = -2 \Re
                \left\langle
                    \left(\bY - \bX\right),
                    \frac{\partial \bX}{\partial \theta_i}
                \right\rangle
                + \frac{\partial \circvar}{\partial \theta_i}, \\
        h_{i,j} = 2 \Re
            \biggl(
                    \left\langle
                        \frac{\partial \bX}{\partial \theta_i},
                        \frac{\partial \bX}{\partial \theta_j}
                    \right\rangle
                -
                \underbrace{
                    \left\langle
                        \left(\bY - \bX\right),
                        \frac{\partial^2 \bX}{\partial \theta_i \partial \theta_j}
                    \right\rangle
                }_{\parbox{6em}{\scriptsize{Neglected if approximation used}}}
            \biggl)
            + \frac{\partial^2 \circvar}{\partial \theta_i \partial \theta_j}.
    \end{gather}
\end{subequations}
The derivatives of the phase variance are given by:
\begin{subequations}
    \begin{gather}
        \frac{\partial \circvar}{\partial \theta_i} =
        \begin{cases}
            \frac{1}{RM}
            \left(
                c_{\Sigma} \sin \phi_{i-M} -
                s_{\Sigma} \cos \phi_{i-M}
            \right) & M \leq i < 2M\\
            0 & \text{otherwise}
        \end{cases}\\
        \frac{\partial^2 \circvar}{\partial \theta_i \partial \theta_j} =
        \begin{cases}
            \begin{split}
                \tfrac{1}{RM}\left[
                    \tfrac{1}{R^2}
                    \left(c_{\Sigma} \sin \phi_{i-M}  - s_{\Sigma} \cos \phi_{i-M} \right)^2 \right. \\
                    \left. + c_{\Sigma} \cos \phi_{i-M} + s_{\Sigma} \sin \phi_{i-M}
                    - 1
                \vphantom{\tfrac{1}{RM}}\right]
            \end{split}
            & M \leq i, j < 2M, i = j\\
            \begin{split}
                \tfrac{1}{RM}\left[
                    \tfrac{1}{R^2}
                    \left(c_{\Sigma} \sin \phi_{i-M} - s_{\Sigma} \cos \phi_{i-M} \right) \right.\\
                    \times \left(c_{\Sigma} \sin \phi_{j-M} - s_{\Sigma} \cos \phi_{j-M} \right) \\
                    \left. - \cos\left( \phi_{i-M} - \phi_{j-M} \right)
                    \vphantom{\tfrac{1}{R^2}}
                \right]
            \end{split}
            & M \leq i, j < 2M, i \neq j\\
            0 & \text{otherwise}
        \end{cases}
    \end{gather}
\end{subequations}

The phase variance-regularised fidelity (\cref{eq:fidelity-phasevar}) is
minimised according the unconstrained \ac{NLP} routine described above. It is
therefore possible for oscillators to acquire parameters which are unrealistic
as the optimiser evolves (see \cpageref{pg:param-constraints} for an outline of
the expected ranges that the parameters reside in). With the inclusion of
the variance of oscillator phases, there are situations where oscillators
acquire negative amplitudes. Typically, this occurs when there are
oscillators in the \ac{MPM} result which start out with phases that are far from \qty{0}{\radian}. By acquiring a negative amplitude and a phase close to
\qty{0}{\radian}\,---\,the expected phase of most oscillators in the parameter
set\,---\,little change to the \ac{RSS} term is made, while
$\circvar$ will be reduced. The presence of such oscillators is
undesirable, as they are spurious in the context of phased data. As a
result, the \ac{NLP} routine periodically checks for negative-amplitude oscillators
(\crefrange{state:neg-amp-start}{state:neg-amp-end} in \cref{alg:nlp}). After a
given number of iterations ($25$ is the value given in \cref{alg:nlp})
if any oscillators have acquired negative
amplitudes, these are removed from the model, and the routine continues with a reduced number
of oscillators. While not infallible, this can help to to get rid of excessive oscillators in the
model which do not correspond to true signals in the data. For example, in
scenarios where an overestimate of model order has occurred, noise components
will be incorporated into the \ac{MPM} result. These typically have greater
variability in their phases; often, such oscillators gain negative amplitudes
as the \ac{NLP} routine evolves, leading to unwanted signals being removed and
a more parsimonious result being obtained.

\section{Profiling the \acs{MPM} and \acs{NLP}}
\label{sec:profiling}
The routine described for \ac{FID} estimation involves operations which can be
computationally demanding, with the burden on computational resources
increasing with the number of points in the \ac{FID}, as well as the number of
oscillators in the model. This is the case both in terms of the amount of work
done by the \ac{CPU}, and the amount of \ac{RAM} needed to store all the
required data as the routine runs. For the \ac{MPM}, the most
demanding aspect is \ac{SVD} calculations while for numerical optimisation, it
is generation of the Hessian matrix in each iteration. Detailed accounts of the
computational complexity of the \ac{MPM} and \ac{MMEMPM} have been
presented\cite{Hua1992,Chen2007}.  However, it is useful to consider what the
actual run times of these routines are on a modern computer; a lot of
accounts on the \ac{MPM} are from decades before this work, and so the
run time will have decreased considerably thanks to
improvements in processing power. As an example, the account by Pines a
co-workers from 1997 outlining the \ac{ITMPM} states that a signal comprising
$1024$ points would take about
\qty{4.5}{\minute} to be processed by the \ac{MDL} and \ac{MPM}, using a
\qty{100}{\mega\hertz} \ac{CPU}\cite{Lin1997}. On the system used for all
results generated for this work (see \cref{rem:workstation}) an
equivalent computation takes about \qty{100}{\milli\second}.
\begin{remark}
    \label{rem:workstation}
    All results generated in this work were acquired using a workstation
    featuring a Intel\textregistered\ Core\texttrademark\ i9-10900X \ac{CPU} @
    \qty{3.7}{\giga\hertz}, and \qty{32}{\gibi\byte} of \ac{RAM}.
\end{remark}

To acquire the results presented in this section, third-party \Python profilers
were employed to assess both the line-by-line execution times\cite{LineProf},
and the time-dependent \ac{RAM} usage\cite{MemProf}.

\subsection{The \acs{MPM} and \acs{MMEMPM}}
\begin{figure}
    \includegraphics{mpm_profiling/mpm_profiling.pdf}
    \caption[
        Outlines the run times and peak memory consumption of
        the \acs{1D} \acs{MPM} and \acs{2D} \acs{MMEMPM}.
    ]
    {
        Outlines of the run times and peak memory consumption of
        the \acs{1D} \acs{MPM} and \acs{2D} \acs{MMEMPM}.
        The \acp{FID} that were used to acquire these results are described in
        the main text.
        \textbf{a.} panels are related to the \acs{MPM}, while \textbf{b.}
        panels are related to the \acs{MMEMPM}.
        \textbf{a1.} The amount of time required to compute the \ac{MPM}, as a
        function of number of points. Also plotted is a cubic fit of the
        circular points.
        \textbf{a2.} Peak memory consumption in performing the \ac{MPM} as a
        function of the number of points.
        \textbf{b1.} The run times for computing the \ac{MMEMPM} of
        \acp{FID} with $\None = 64$, and variable $\Ntwo$ and $M$.
        \textbf{b2.} The time required to compute the \ac{SVD} of $\EY$ for the
        $M=40$ \acp{FID}. The solid line is a quadratic fit of the circular
        points, while the dashed line is a quadratic fit of the square points.
        \textbf{b3.} Peak memory consumption in performing the \ac{MMEMPM} for
        the $M=40$ \acp{FID}.
    }
    \label{fig:mpm-profiling}
\end{figure}

A series of synthetic \ac{1D} \acp{FID} were constructed, comprising $10$ evenly-spaced
signals, with a
variable number of time-points $N \in \lbrace 512k \hspace*{2pt} \vert
\hspace*{2pt} k \in \lbrace 1, 2, \cdots, 16 \rbrace \rbrace$.
For each \ac{FID}, the \ac{MPM} routine outlined in \cref{lst:mpm} was
performed 5 times, with a pencil parameter $L = \lfloor
\nicefrac{N}{3} \rfloor$.
The mean complete time to run the \ac{MPM} is plotted as a function
of $N$ in panel a1 of \cref{fig:mpm-profiling}, where it can be seen that for
the larger values of $N$ considered, the \ac{MPM} is computed in approximately
$\mathcal{O}({N}^3)$ time. This is because the rate-limiting step of the
\ac{MPM} is the \ac{SVD} of $\Hy$, whose size is to a very good approximation
$\tfrac{2N}{3} \times \tfrac{N}{3}$\footnote{
    \label{fn:svd-complexity}
    The time complexity for the \ac{SVD} of generic a $m \times n$ matrix is
    $\mathcal{O}(\operatorname{min}(m, n)^2 \cdot \operatorname{max}(m, n))$,
    while the space complexity is $\mathcal{O}(mn)$.
}. For smaller values of $N$, a deviation
away from a cubic relationship is observed;
a fit of a cubic function of the form $aN^3 + b$ to the data satisfying $7 \leq k \leq
16$ is plotted in panel a1 to highlight this behaviour.
This arises because the computation of the complex amplitudes using
\cref{eq:complex-amplitudes} has a comparatively significant run time
relative to \ac{SVD} in the low-$N$ regime;
for a $512$ point signal, the \ac{SVD} of $\Hy$ took up roughly 80\% of the
complete run time, while the computation of the complex amplitudes took up
roughly 20\%. For a 8192 point signal, these percentages had changed to
$>\!\!99\%$ and $<\!\!1\%$, respectively.
For all values of $N$,

The \ac{MPM} was run a sixth time on each generated \ac{FID} in order to assess
the effect of $N$ on the space complexity.
The peak \ac{RAM} consumption is plotted in panel a2 of \cref{fig:mpm-profiling}.
A clear quadratic dependence on consumption is realised as function of $N$,
again in agreement with the expected space complexity of the
\ac{SVD}\footnoteref{fn:svd-complexity}.

A similar study was conducted in consideration of the \ac{MMEMPM}. A
series of \ac{2D} \acp{FID} were simulated, all of which all comprised $\None =
64$. The \acp{FID} possessed values of $\Ntwo \in \lbrace 32k
\hspace*{2pt} \vert \hspace*{2pt} \lbrace 1, \cdots, 16 \rbrace \rbrace$.
With the \ac{MPM}, since a
complete \ac{SVD} of the matrix $\Hy$ is computed, the model order is
irrelevant in dictating the run time (at least when $M \ll N$). This is not the
case for the \ac{MMEMPM}; because the \Python implementation used
(\cref{lst:mmempm}) employs a truncated \ac{SVD} in order to compute only the
first $M$ components of $\EY$, the elected model order will have an impact on
run time.  Therefore, \acp{FID} with different model orders were generated: $M
\in \lbrace 10, 20, 40, 80 \rbrace$.

The \ac{MMEMPM} was repeated 5 times for each \ac{FID}, and the mean run
times for each $\Ntwo$ and $M$ considered is plotted in panel b1 of
\cref{fig:mpm-profiling}.
Only the results for cases where the \ac{MMEMPM} was able to yeild a result in
agreement with the data are presented; for certain \acp{FID} with low
$\Ntwo$ and high $M$, appropriate estimation results could not be yielded as
the constituent signals were too poorly resolved.
While in the high-$N$ regime the \ac{MPM} has a cubic time dependence
on the number of points, the \ac{MMEMPM} can be seen to have an approximately
quadratic complexity regarding $\Ntwo$.
For all combinations of $\Ntwo$ and $M$,
the truncated \ac{SVD} was the most time consuming aspect of the routine,
however other steps have notable run times too. Most of the run time is due to
the following five steps (with the relevant lines in \cref{lst:mmempm} given):
\begin{itemize}
    \item Step 1: Construction of $\EY$. This involves building the Hankel matrices
        $\lbrace \symbf{H}_{\symbf{y},\none} : \none \in \lbrace 0, \cdots, 63
        \rbrace \rbrace$, assigning them to the
        correct locations in $\EY$, and finally converting  $\EY$ to a sparse
        matrix\footnote{
            Truncated \ac{SVD} is only available for sparse matrices in
            \textsc{NumPy}\cite{svds}. Some experimenting was done to determine
            the most efficient means of generating $\EY$ in sparse form, and
            subsequently compute its \ac{SVD}.
            It was determined that constructing $\EY$ using a
            standard \textsc{NumPy} array before converting it to
            \ac{CSR} format\cite{csr} was optimal.
        } (Lines \ref{ln:EY-start} to \ref{ln:sparse1}).
    \item Step 2: Truncated \ac{SVD} of $\EY$ to form $\symbf{U}_M$ (Line \ref{ln:sparse2}).
    \item Step 3: Determining $\bdzone$ and  $\symbf{W}^{(1)}$ by computing the
        \ac{EVD} of $\symbf{U}_{M1}^+ \symbf{U}_{M2}^{\vphantom{+}}$ (Lines
        \ref{ln:poles1-start} to \ref{ln:poles1-end}).
    \item Step 4: Generating the second set of signal poles $\bdztwo$, by
        multiplying $\symbf{U}_M$ by the permutation matrix, and extracting
        the diagonal from matrix $\symbf{G}$, computed using \cref{eq:G} (Lines
        \ref{ln:poles2-start} to \ref{ln:poles2-end}). N.B. The \acp{FID} were
        constructed such that the signal poles were unique on all occasions, so
        the additional treatment of repeated signal poles was not necessary.
    \item Step 5: Computation of the complex amplitudes using
        \cref{eq:complex-amplitudes-2d} (Lines \ref{ln:comp-amps-2d-start} to
        \ref{ln:comp-amps-2d-end}).
\end{itemize}
A comparison of the relative times to perform these steps for a few select
pairings of $M$ and $\Ntwo$ is provided by \cref{tab:mmempm-steps}. It can be
seen that as $M$ increases, the relative amount of time spent performing
\ac{SVD} increases, while the amount of time to generate $\EY$ decreases. This
reflects the greater number of iterations required by
the Rayleigh-Ritz method to produce the desired number of \ac{SVD} components,
while the run time to generate $\EY$ remains fixed.

Interestingly, the plots for a given value of $M$ in panel b1 do not exhibit
consistent quadratic behaviour throughout. After a certain value of $\Ntwo$, a
slight reduction in the gradient of the plots in panel b1 can be observed (cf
the square and circular points). This was found to be caused by the \ac{SVD}
computation, whose run times for the $M=40$ \acp{FID} are plotted in panel b2.
The square and circular points both display quadratic behaviour, though the
exact form of the function which
describes them are different ; both sets of points have been fit to curves of
the form of the form $a\Ntwo^2 + b$. After $\Ntwo$ becomes larger than
roughly $200$, the \ac{SVD} run time appears to enter a new regime in which
subsequent increases in  $\Ntwo$ cause the rate of \ac{SVD} to increase at a
slower rate than was the case previously. The exact reason for this probably
lies in the implementation of the Rayleigh-Ritz algorithm, and has not yet been
ascertained.

The peak \ac{RAM} usage for the $M=40$ \acp{FID} as a function of $\Ntwo$ is
plotted in panel b3, where a quadratic complexity is observed. The variation in
memory consumption barely changes as a function of the model order, since the
peak usage is dependent on the size of $\EY$.

It should be noted that the run time and peak \ac{RAM} consumption will also be
(roughly) quadratically dependent on $\None$, just as with $\Ntwo$, i.e.
increasing  $\None$ for  $64$ to  $128$ would cause all the run times in panel
b1 and peak memory usages in panel b3 to quadruple in value.

\begin{table}
    \begin{center}
        \begin{tabular}{ c c c c c c c }
            \toprule
            $\Ntwo$ &
            $M$ &
            Step 1 &
            Step 2 &
            Step 3 &
            Step 4 &
            Step 5 \\
            \midrule
            64 & 10 & 12.2\% & 60.6\% & 2.9\% & 9.7\% & 13.8\% \\
            64 & 40 & 4\% & 74.6\% & 1.9\% & 8.9\% & 9.6\% \\
            512 & 10 & 22.1\% & 67.2\% & 0.2\% & 3.1\% & 7.2\% \\
            512 & 40 & 7.4\% & 81.9\% & 0.1\% & 1.3\% & 9.4\% \\
            512 & 80 & 3.8\% & 85.8\% & 0.1\% & 0.8\% & 9.4\% \\
            \bottomrule
        \end{tabular}
    \end{center}
    \caption[
        A comparison of the relative times to perform the key steps in the \acs{MMEMPM}.
    ]{
        A comparison of the relative times to perform the key steps in the
        \acs{MMEMPM}, for selected pairings on $M$ and $\Ntwo$. See the main
        text for a description of what each step entails.
    }
    \label{tab:mmempm-steps}
\end{table}

\subsection{Computing the Hessian for \acs{NLP}}

\section{Frequency Filtration}
\label{sec:filtering}
The previous section provides motivation for finding ways to reduce the
number of points in the signal and also the number of oscillators that the
signal contains. This has led to work on a procedure for generating
frequency-filtered ``sub-FIDs'' from the original data. A detailed description
of the filtering procedure is presented in this section.

\subsection{The \acl{VE}}
\label{subsec:ve}
In brief, the filtering procedure  consists taking the \ac{FT} of the \ac{FID},
applying a band-pass filter on the spectral data to discard parts of not being
considered, and returning the spectrum back to the time-domain by an \ac{IFT}.
For a filtered \ac{FID} to be faithfully described by the model of a
summation of damped complex sinusoids, it is necessary that the
spectral peaks of interest lie effectively entirely within the filter
region\footnote{
    Lorentzian lineshapes tend to, but don't reach zero, as the distance from
    the maximum tends to $\infty$\cite{Tang1994}. However, as long as a
    sufficiently wide filtering is employed, the regions of the Lorentzian
    which do not pass through the filter can be assumed to be negligible.
}.
For absorptive Lorentzians, due to their characteristically narrow
linewidths, this is straightforward. However, for broader dispersive
Lorentzians, this is far more challenging. For this reason, generating
a spectrum in which only the real component is retained is desired.
Assuming the data has been phase-corrected, this will produce a
spectrum comprising only absorptive Lorentzians. The \ac{VE} has been employed
here, which has found application in the field of compressed sensing
NMR\cite{Mayzel2014,Golowicz2020,Luo2020}. This is a signal with double the
size as the original \ac{FID}, with the key characteristic that its \ac{FT} has
a real component which is equivalent to its counterpart derived from an
unaltered \ac{FID} (except it has double the points), and an imaginary
component of $0$s. The concept of a virtual echo can be applied to data of any
number of dimensions. However, only \ac{1D} virtual echoes are employed in the
work outlined in this thesis. An account of the \ac{2D} virtual echo is
provided in the Appendix (Section \ref{sec:multidim-ve}.

Assuming that a \ac{1D} \ac{FID} $\by \in \mathbb{C}^N$ is phased, such that
$\bdphi = \symbf{0} \in \mathbb{R}^M$, it can be described by
\begin{subequations}
    \begin{gather}
        y_n = \xi_n (c_n + \iu s_n) + w_n,\\
        \xi_n = \sum_m a_m \exp(-\eta_m n \Dt),\\
        c_n / s_n = \sum_m \cos / \sin(2 \pi f_m n \Dt).
    \end{gather}
\end{subequations}
The frequency-dependence has been decomposed into its real and imaginary
components. With this in mind, a conjugate pair of signals $\symbf{\psi}_{\pm}
\in \mathbb{C}^N$ are defined:
\begin{equation}
    \psi_{\pm,n} = \xi_n (c_n \pm \iu s_n) + w_n \equiv \Re(y_n) \pm \iu \Im(y_n)
\end{equation}
Two vectors $\lbrace \symbf{t}_{1}, \symbf{t}_2 \rbrace \in \mathbb{C}^{2N}$
are constructed using the conjugate pair.
$\symbf{t}_1$ is given by $\symbf{\psi}_+$ padded with zeros from below:
    \begin{equation}
        \symbf{t}_1 = \begin{bmatrix}
            \symbf{\psi}_+ \\ \symbf{0} \in \mathbb{C}^{N}
        \end{bmatrix}.
    \end{equation}
$\symbf{t}_2$ is given by $\symbf{\psi}_{-}$ with its elements in
    reversed order ($\cdot^{{\leftrightsquigarrow}}$), padded with zeros
    from above, and finally subjected to a right circular shift by one
    element ($\cdot^{{\circlearrowright}}$):
    \begin{equation}
        \symbf{t}_2 = \begin{bmatrix}
            \symbf{0} \in \mathbb{C}^{N} \\ \symbf{\psi}_-^{{\leftrightsquigarrow}}
    \end{bmatrix}^{{\circlearrowright}}.
   \end{equation}
The \ac{VE} $\by_{\text{ve}}$ is then given by $\symbf{t}_1 +
\symbf{t}_2$, with the first element divided by $2$, which is equivalent to
\begin{equation}
    \by_{\text{ve}} =
    \begin{bmatrix}
        \Re(y_0^{\vphantom{*}}) &
        y_1^{\vphantom{*}} &
        \cdots &
        y_{N-1}^{\vphantom{*}} &
        0 &
        y_{N-1}^* &
        \cdots &
        y_1^*
    \end{bmatrix}\T.
\end{equation}
As eluded to already, the \ac{FT} of $\by_{\text{ve}}$ produces a spectrum
$\symbf{s}_{\text{ve}}$ such that $\Im\left(\symbf{s}_{\text{ve}}\right) =
\symbf{0}$, with $\Re\left(\symbf{s}_{\text{ve}}\right)$ featuring absorption
Lorentzian peaks.

\subsection{The filtering process}
To filter the spectrum $\symbf{s}_{\text{ve}}$, it is subjected multiplication
with a function which acts as a band-pass filter. An example of a suitable
filter is a \emph{super-Gaussian} $\symbf{g} \in \mathbb{C}^{2N}$ defined by a
central index $c \in \lbrace 0, \cdots, 2N-1 \rbrace$ and a bandwidth  $b \in
\mathbb{N}: b < 2N$ in each dimension:
\begin{equation}
    g_n = \exp \left(-2^{p+1} \left(\frac{n - c}{b}\right)^p\right).
    \label{eq:super-Gaussian-onedim}
\end{equation}
The scalar $p \in \mathbb{R}_{>0}$ dictates the steepness
of the filter at the boundaries, with the function becoming more rectangular
as it increases. It is set to $40$ in this work. Application of the
super-Gaussian filter to $\symbf{s}_{\text{ve}}$
would lead to large sections of the filtered spectrum being $0$. This has an
undesired impact on the \ac{MDL}, as noise that has passed through filter (i.e.
the noise inside the region of interest) will now seem to resemble true signal,
as its amplitude is infinitely greater than the zeroed regions. A massive
over-estimation of model order result due to this. In order to obtain better
results from model order selection, an array of synthetic \ac{AWGN} is
added to the filtered spectrum. To achieve this, a region in
$\symbf{s}_{\text{ve}}$ is specified which contains no discernible signal peaks
(referred to as the \emph{noise region}). The variance of this region
$\sigma^2$ is determined, and used to construct a vector of values sampled from
a normal distribution with mean $0$ and variance $\sigma^2$,
$\symbf{w}_{\sigma^2} \in \mathbb{R}^{2N}$.
The filtered spectrum is then given by
\begin{equation}
    \widetilde{\symbf{s}}_{\text{ve}} = \symbf{s}_{\text{ve}} \odot \symbf{g} + \symbf{w}_{\sigma^2} \odot \left(\symbf{1} - \symbf{g} \right).
    \label{eq:Sve-tilde}
\end{equation}
Note that the noise array's magnitude at each point is attenuated by the value
of the super-Gaussian filter, as a means of ensuring the noise variance remains
consistent across the frequency space.

After filtering, $\widetilde{\symbf{s}}_{\text{ve}}$ is returned to the
time-domain by \ac{IFT}. The \ac{IFT} of a real-valued spectrum generates a
conjugate-symmetric signal, which is also a \ac{VE}. This is sliced so as to
retain the first half, which is the final filtered sub-FID $\widetilde{\by} \in
\mathbb{C}^{N}$:
\begin{subequations}
    \begin{gather}
        \widetilde{\symbf{y}}_{\text{ve}} = \IFT(\widetilde{\symbf{s}}_{\text{ve}}),\\
        \widetilde{\by} = \widetilde{\symbf{y}}_{\text{ve}}[0 : N].
    \end{gather}
    \label{eq:yve-tilde}
\end{subequations}
A summary of the filtering process is provided by Figure \ref{fig:filtering}.
\begin{figure}
     \centering
     \includegraphics{filtering/filtering.pdf}
     \caption[
         An illustration of the filtering procedure applied to a \acs{1D}
         \acs{FID}.
     ]{
         An illustration of the filtering procedure applied to a \ac{1D}
         \ac{FID}.
         \textbf{a.} A \ac{VE} $\by_{\text{ve}}$, with the first and last
         $N$ points coloured red and blue, respectively. The middle of the
         \ac{VE} is magnified to highlight its conjugate symmetry.
         \textbf{b.} The \ac{FT} of the \ac{VE}, $\symbf{s}_{\text{ve}}$.
         The region of interest (orange) and noise region (grey) are denoted.
         \textbf{c.} A super-Gaussian function used as a band-pass filter,
         $\symbf{g}$.
         \textbf{d.} \acs{AWGN} vector to be added to the filtered spectrum.
         The magnitude of the signal at each point is dependent on the
         corresponding super-Gaussian value.
         \textbf{e.} The filtered spectrum $\widetilde{\symbf{s}}_{\text{ve}}$,
         formed by applying the super-Gaussian filter, and adding the noise
         vector.
         \textbf{f.} The \ac{IFT} of the filtered spectrum,
         $\widetilde{\symbf{y}}_{\text{ve}}$, from which the final filtered
         signal $\widetilde{\symbf{y}}$ is obtained by extracting
         the first $N$ points.
     }
     \label{fig:filtering}
\end{figure}

The central index and bandwidth of the super-Gaussian filter function are given
by the following expressions:
\begin{subequations}
    \begin{gather}
        c = \tfrac{1}{2} \left(l_{\text{idx}} + r_{\text{idx}}\right), \\
        b = l_{\text{idx}} - r_{\text{idx}},
    \end{gather}
\end{subequations}
where $l_{\text{idx}}$ and $r_{\text{idx}}$ denote the desired
indices where the filter's left and right bounds are located, respectively.
Array indices can be obtained from the corresponding spectral frequencies
$f^{(d)}_{\unit{\hertz}}$ via
\begin{equation}
    \begin{gathered}
        f_{\text{idx}} =
            \left \lfloor
                \frac
                {
                    \left(2N - 1\right)
                    \left(\fsw + 2 \left(\foff - f_{\unit{\hertz}}\right) \right)
                }
                {2 \fsw}
            \right \rceil \\
        \forall f_{\unit{\hertz}} \in
            \left[\foff - \tfrac{1}{2} \fsw, \foff + \tfrac{1}{2} \fsw\right].
        \label{eq:fidx}
    \end{gathered}
\end{equation}
Conversion from \unit{\partspermillion} to array indices can be achieved by
replacing  $f_{\unit{\hertz}}$ in \eqref{eq:fidx} with
$f_{\unit{\partspermillion}} f_{\text{sfo}}$, where $f_{\text{sfo}}$ is the
transmitter frequency (\unit{\mega \hertz}) and $f_{\unit{\partspermillion}}$
is the frequency expressed as a chemical shift.

\subsubsection{Spectrum slicing}
Thus far, the method described is able to reduce the model order of a given
signal, however the signal still comprises the same number of points. However
it is clear that there are a large number of points outside the region of
interest in $\widetilde{\symbf{s}}_{\text{ve}}$ that do not possess any
meaningful information. Discarding such points will then lead to filtered
\ac{FID} with the same information about the signals of interest, but with
far fewer points. A slicing ratio is defined, $\chi \in \mathbb{R}: \chi>
1$,
which dictates the left and right indices at which the spectrum should be
sliced:
\begin{subequations}
    \begin{gather}
        l_{\text{slice}} =
        \begin{cases}
            c - \left \lfloor \frac{b \chi}{2} \right \rfloor &
            \text{if } \geq 0 \\
            0 & \text{otherwise}
        \end{cases} \\
        r_{\text{slice}} =
        \begin{cases}
            c + \left \lceil \frac{b \chi}{2} \right \rceil &
            \text{if } \leq 2N - 1 \\
            2N - 1 & \text{otherwise}
        \end{cases}
    \end{gather}
\end{subequations}
The filtered spectrum is then sliced accordingly:
\begin{equation}
    \widetilde{\symbf{s}}_{\text{ve,slice}} =
        \widetilde{\symbf{s}}_{\text{ve}} [
            l_{\text{slice}} :
            r_{\text{slice}} + 1
        ].
\end{equation}
Generation of the final sub-\ac{FID} is then achieved in a similar fashion to
before: by performing \ac{IFT}, and retaining the first half of the signal.
It is also necessary to scale the signal by the ratio of the number of points
in the sliced spectrum and it's unsliced counterpart, in order to ensure that
the amplitudes of each signal are unaffected.
\begin{subequations}
    \begin{gather}
        \widetilde{\by} =
            \frac{r_{\text{slice}} - l_{\text{slice}}}{2N}
            \IFT(\widetilde{\symbf{s}}_{\text{ve,slice}})
            [0 : N_{\text{slice}}],\\
            N_{\text{slice}} = \left \lfloor \frac{r_{\text{slice}} - l_{\text{slice}}}{2} \right \rfloor
    \end{gather}
\end{subequations}
The associated sweep width and transmitter offset of the \ac{FID} will have
been altered by this process, and in order to derive accurate frequencies and
damping factors for the sliced signal, it is necessary to determine these. The
corrected values can be computed using
\begin{subequations}
    \begin{gather}
        f_{\text{sw,slice}} = \frac{r_{\text{slice}} - l_{\text{slice}}}{2N - 1} \fsw\\
        f_{\text{off,slice}} = \foff + \frac{\fsw}{2} \left(
            1 - \frac{l_{\text{slice}} + r_{\text{slice}}}{2N - 1}
        \right)
    \end{gather}
\end{subequations}

\section{Summary}
The matrix pencil-based methods described above are well established as
effective procedures for parametric estimation of time-domain signals in a
number of disciplines, including radar detection\cite{Hua1994},
acoustics\cite{TODO}, and \ac{NMR}\cite{Lin1997}, with specific application to
relaxometry being a recently introduced application \cite{Fricke2020,
Wortge2023}. Due to considerable advances computational processing power since
the introduction of the technique, estimates can be acquired from \ac{NMR}
signals in reasonable times. One notable downside of the technique that has
been realised while assessing its effectiveness in parametrising \ac{NMR}
\ac{FID}s is its propensity to return oscillators with unexpected phase
behaviour, especially in scenarios involving resonances with close frequencies.
For this reason, estimating phase-corrected \acp{FID}, using the result of the
\ac{MPM} as an initial guess to feed into a phase-variance penalised \ac{NLP}
routine was proposed as a means of improving parameter estimates. The theory
underpinning the procedure has been explored in this chapter.

The computational burden of running the procedure is large and often
intractable for complete \ac{NMR} signals, which often comprise thousands of
samples, and at least hundreds of contributing resonances. This has been
illustrated through profiling both the \ac{CPU} time and the peak memory
requirements for the \ac{1D} and \ac{2D} methods. \note{Say more when
completed?} For this reason, a method to break \acp{FID} into
frequency-filtered sub-signals, by filtering spectra derived using virtual
echoes, is introduced.  This method enables to formation of signals with far
fewer resonances and samples as compared to the full signal.

Algorithm \ref{alg:1d-2d-summary} provides an overview of the principal steps
involved in the \ac{1D} and \ac{2D} estimation procedures. Detailed algorithms
for each step are presented elsewhere in this text.

\begin{algorithm}
    \begin{algorithmic}[1]
        \caption[
            An overview of the estimation procedure outlined in this work.
        ]{
            An overview of the estimation procedure outlined in this work, for
            the consideration of \ac{1D} and \ac{2D} \ac{NMR} signals.
        }
        \label{alg:1d-2d-summary}
        \Procedure{Estimate$1$D}{$
            \by \in \mathbb{C}^{\None},
            \symbf{r}_{\text{interest}} \in \mathbb{R}^2,
            \symbf{r}_{\text{noise}} \in \mathbb{R}^2,
            M \in \mathbb{N}_0
            $
        }
            \State $\widetilde{\by} \gets \textsc{Filter}1\textsc{D}\left(
                \by,
                \symbf{r}_{\text{interest}},
                \symbf{r}_{\text{noise}}
                \right)
            $;
            \Comment{Algorithm \ref{alg:filter-1d}}
            \If{$M=0$}
                \State $M \gets \textsc{MDL}\left(\widetilde{\symbf{y}}\right)$;
                \Comment{\note{TODO}}
            \EndIf
            \State $\bthzero \gets \textsc{MPM}\left(\widetilde{\by}, M\right)$;
            \Comment{Algorithm \ref{alg:mpm}}
            \State $\bthstar, \symbf{\epsilon}^{(*)} \gets \textsc{NLP}\left(\widetilde{\by}, \bthzero\right)$;
            \Comment{Algorithm \ref{alg:nlp}}
            \State \textbf{return} $\bthstar, \symbf{\epsilon}^{(*)}$;
        \EndProcedure
        \Statex
        \Procedure{Estimate$2$D}{$
            \bY_{\cos} \in \mathbb{C}^{\None \times \Ntwo},
            \bY_{\sin} \in \mathbb{C}^{\None \times \Ntwo},
            \symbf{R}_{\text{interest}} \in \mathbb{R}^{2 \times 2},
            \symbf{R}_{\text{noise}} \in \mathbb{R}^{2 \times 2},
            M \in \mathbb{N}_0
            $
        }
            \State $\widetilde{\bY} \gets \textsc{Filter}2\textsc{D}\left(
                \bY_{\cos},
                \bY_{\sin},
                \symbf{R}_{\text{interest}},
                \symbf{R}_{\text{noise}}
                \right)
            $;
            \Comment{Algorithm \ref{alg:filter-2d}}
            \If{$M=0$}
                \State $M \gets \textsc{MDL}\left(\widetilde{\bY}[:, 0]\right)$;
                \Comment{Run the MDL on the first direct-dimension slice.}
            \EndIf
            \State $\bthzero \gets \textsc{MMEMPM}\left(\widetilde{\bY}, M\right)$;
            \Comment{Algorithm \ref{alg:mmempm}}
            \State $\bthstar, \symbf{\epsilon}^{(*)} \gets \textsc{NLP}\left(\widetilde{\bY}, \bthzero\right)$;
            \Comment{Algorithm \ref{alg:nlp}}
            \State \textbf{return} $\bthstar, \symbf{\epsilon}^{(*)}$;
        \EndProcedure
    \end{algorithmic}
\end{algorithm}

Having established an estimation routine, the next chapter focusses on its
performance, as well as applications which are possible through parametric
estimation.

