\section{Information on simulated datasets}
\label{sec:simulated-datasets}

Table \ref{tab:shifts_and_couplings} provides a specification of the chemical
shifts and scalar couplings that made up the spin systems used to construct
simulated data with Spinach. In all cases, the dataset was generated using the
\texttt{new\_spinach} method associated with the relevant
\texttt{Estimator} object in \ac{EsPy}. The \texttt{new\_spianch} method takes the
chemical shifts and scalar couplings as arguments, as well as a specification
of the basic experiment parameters including the number of points sampled, the
sweep width, etc. It instantiates a
\href{https://www.mathworks.com/help/matlab/matlab-engine-for-python.html}{MATLAB\textregistered\
engine}, and runs a Spinach simulation, and finally stores the simulated
\ac{FID} within a new \texttt{Estimator} object.

For the \ac{2DJ} sucrose dataset, the chemical shifts and couplings were
extracted from a logfile of a Guassian computation which ships with Spinach at
the path
\path{SPINACHROOT/examples/standard_systems/sucrose.log}.
This file was loaded, and the chemical shift tensors and couplings for the 22 spins
were extracted. Isotropic chemical shifts were determined for each spin $i
\in \lbrace 1, \cdots, 22\rbrace$ via
\begin{equation}
    \sigma^{\text{iso}}_i = \frac{\Tr \symbf{\sigma}_i}{3},
\end{equation}
where $\symbf{\sigma}_i \in \mathbb{R}^{3 \times 3}$ is the chemical shift
tensor of spin $i$.


\begin{longtable}[h!]{c c c c c c}
\caption[
The isotropic chemical shifts, scalar couplings and relaxation times
associated with spin systems used in \textsc{Spinach} simulations.
]{
The isotropic chemical shifts ($\delta$), corresponding rotating frame
frequencies ($\omega_{\text{rot}}$), scalar couplings ($J$), and
relaxation times ($T_1$, $T_2$, if applicable)
associated with spin systems used in \textsc{Spinach} simulations.
}
\label{tab:shifts_and_couplings}\\
\hline
Spin & $\delta$ (\unit{\partspermillion}) & $\omega_{\text{rot}}$ (\unit{\hertz}) & $J$ (\unit{\hertz}) & $T_1$ (\unit{\second}) & $T_2$ (\unit{\second})\\
\hline
\endfirsthead
\hline
Spin & $\delta$ (\unit{\partspermillion}) & $\omega_{\text{rot}}$ (\unit{\hertz}) & $J$ (\unit{\hertz}) & $T_1$ (\unit{\second}) & $T_2$ (\unit{\second})\\
\hline
\endhead
\hline
\endlastfoot
\multicolumn{6}{r}{Continues on next page...}\\
\hline
\endfoot
\hline
\multicolumn{6}{c}{\textbf{Four Multiplets, Run 1}}\\
\hline
\textbf{A} & \num{-2.78e-02} & -13.93 & \textbf{E:} -9.627, \textbf{F:} -8.202, \textbf{G:} 6.742 & -- & --\\

\textbf{B} & \num{-7.11e-03} & -3.56 & \textbf{E:} -4.491, \textbf{F:} 5.333, \textbf{G:} 9.303 & -- & --\\

\textbf{C} & \num{-1.63e-03} & -0.81 & \textbf{E:} 3.953, \textbf{F:} 5.422, \textbf{G:} 5.914 & -- & --\\

\textbf{D} & \num{1.53e-02} & 7.66 & \textbf{E:} -7.902, \textbf{F:} -4.556, \textbf{G:} 6.217 & -- & --\\
\hline
\multicolumn{6}{c}{\textbf{Four Multiplets, Run 2}}\\
\hline
\textbf{A} & \num{-1.48e-02} & -7.38 & \textbf{E:} -7.492, \textbf{F:} 0.917, \textbf{G:} 2.933 & -- & --\\

\textbf{B} & \num{-1.18e-02} & -5.88 & \textbf{E:} -4.304, \textbf{F:} -1.815, \textbf{G:} 5.420 & -- & --\\

\textbf{C} & \num{-4.32e-03} & -2.16 & \textbf{E:} 4.832, \textbf{F:} 7.573, \textbf{G:} 8.268 & -- & --\\

\textbf{D} & \num{1.76e-02} & 8.80 & \textbf{E:} -9.244, \textbf{F:} -1.816, \textbf{G:} -0.478 & -- & --\\
\hline
\multicolumn{6}{c}{\textbf{Four Multiplets, Run 3}}\\
\hline
\textbf{A} & \num{-2.23e-02} & -11.17 & \textbf{E:} -5.347, \textbf{F:} -1.851, \textbf{G:} 1.407 & -- & --\\

\textbf{B} & \num{1.13e-02} & 5.66 & \textbf{E:} 6.425, \textbf{F:} 7.291, \textbf{G:} 9.806 & -- & --\\

\textbf{C} & \num{2.53e-02} & 12.64 & \textbf{E:} -8.640, \textbf{F:} 0.613, \textbf{G:} 6.998 & -- & --\\

\textbf{D} & \num{2.84e-02} & 14.21 & \textbf{E:} -8.613, \textbf{F:} 0.782, \textbf{G:} 3.830 & -- & --\\
\hline
\multicolumn{6}{c}{\textbf{Four Multiplets, Run 4}}\\
\hline
\textbf{A} & \num{-2.03e-02} & -10.16 & \textbf{E:} -8.646, \textbf{F:} 6.719, \textbf{G:} 7.921 & -- & --\\

\textbf{B} & \num{3.53e-03} & 1.77 & \textbf{E:} -8.857, \textbf{F:} 4.314, \textbf{G:} 9.197 & -- & --\\

\textbf{C} & \num{8.61e-03} & 4.30 & \textbf{E:} -0.620, \textbf{F:} 1.767, \textbf{G:} 6.567 & -- & --\\

\textbf{D} & \num{2.06e-02} & 10.30 & \textbf{E:} -9.060, \textbf{F:} 2.355, \textbf{G:} 9.810 & -- & --\\
\hline
\multicolumn{6}{c}{\textbf{Four Multiplets, Run 5}}\\
\hline
\textbf{A} & \num{-9.16e-03} & -4.58 & \textbf{E:} -8.281, \textbf{F:} 1.621, \textbf{G:} 3.229 & -- & --\\

\textbf{B} & \num{-2.79e-03} & -1.40 & \textbf{E:} 1.655, \textbf{F:} 4.219, \textbf{G:} 6.998 & -- & --\\

\textbf{C} & \num{3.00e-03} & 1.50 & \textbf{E:} -4.280, \textbf{F:} 1.045, \textbf{G:} 5.896 & -- & --\\

\textbf{D} & \num{2.74e-02} & 13.72 & \textbf{E:} -9.316, \textbf{F:} -8.322, \textbf{G:} -3.938 & -- & --\\

\hline
\multicolumn{6}{c}{\textbf{Sucrose}}\\
\hline
\textbf{A} & 6.005 & 1801.6 & \textbf{B}: $2.285$& --& -- \\
\textbf{B} & 3.510 & 1053.1 & \textbf{A}: $2.285$, \textbf{C}: $4.657$, \textbf{H}: $4.828$& --& -- \\
\textbf{C} & 3.934 & 1180.2 & \textbf{B}: $4.657$, \textbf{D}: $4.326$& --& -- \\
\textbf{D} & 3.423 & 1027.0 & \textbf{C}: $4.326$, \textbf{E}: $4.851$& --& -- \\
\textbf{E} & 4.554 & 1366.1 & \textbf{D}: $4.851$, \textbf{F}: $5.440$, \textbf{G}: $2.288$& --& -- \\
\textbf{F} & 3.891 & 1167.4 & \textbf{E}: $5.440$, \textbf{G}: $-6.210$& --& -- \\
\textbf{G} & 4.287 & 1286.2 & \textbf{E}: $2.288$, \textbf{F}: $-6.210$, \textbf{K}: $7.256$& --& -- \\
\textbf{H} & 3.332 & 999.5 & \textbf{B}: $4.828$& --& -- \\
\textbf{I} & 1.908 & 572.3 & --& --& -- \\
\textbf{J} & 1.555 & 466.6 & --& --& -- \\
\textbf{K} & 0.644 & 193.3 & \textbf{G}: $7.256$& --& -- \\
\textbf{L} & 4.042 & 1212.5 & \textbf{M}: $-4.005$, \textbf{S}: $1.460$& --& -- \\
\textbf{M} & 4.517 & 1355.0 & \textbf{L}: $-4.005$& --& -- \\
\textbf{N} & 3.889 & 1166.7 & \textbf{O}: $4.253$& --& -- \\
\textbf{O} & 4.635 & 1390.4 & \textbf{N}: $4.253$, \textbf{P}: $4.448$, \textbf{U}: $3.221$& --& -- \\
\textbf{P} & 4.160 & 1248.0 & \textbf{O}: $4.448$, \textbf{R}: $4.733$& --& -- \\
\textbf{Q} & 4.021 & 1206.2 & \textbf{R}: $-4.182$& --& -- \\
\textbf{R} & 4.408 & 1322.4 & \textbf{P}: $4.733$, \textbf{Q}: $-4.182$, \textbf{V}: $1.350$& --& -- \\
\textbf{S} & 0.311 & 93.3 & \textbf{L}: $1.460$& --& -- \\
\textbf{T} & 1.334 & 400.2 & --& --& -- \\
\textbf{U} & 0.893 & 267.9 & \textbf{O}: $3.221$& --& -- \\
\textbf{V} & 0.150 & 45.0 & \textbf{R}: $1.350$& --& -- \\

\hline
\multicolumn{6}{c}{\textbf{Five Multiplets, Run 1}}\\
\hline
\textbf{A} & \num{1.33e+00} & 665.99 & \textbf{F:} 10.965, \textbf{G:} 12.657, \textbf{H:} 17.070 & 2.178 & --\\

\textbf{B} & \num{1.47e+00} & 735.44 & \textbf{F:} 3.610, \textbf{G:} 2.543, \textbf{H:} 8.448 & 4.430 & --\\

\textbf{C} & \num{1.31e+00} & 653.87 & \textbf{F:} 10.630, \textbf{G:} 6.282, \textbf{H:} 3.012 & 3.319 & --\\

\textbf{D} & \num{1.41e+00} & 705.01 & \textbf{F:} 8.101, \textbf{G:} 4.589, \textbf{H:} 9.068 & 1.007 & --\\

\textbf{E} & \num{1.30e+00} & 650.23 & \textbf{F:} 3.014, \textbf{G:} 16.537, \textbf{H:} 15.587 & 4.992 & --\\
\hline
\multicolumn{6}{c}{\textbf{Five Multiplets, Run 2}}\\
\hline
\textbf{A} & \num{1.47e+00} & 733.07 & \textbf{F:} 19.488, \textbf{G:} 18.279, \textbf{H:} 3.147 & 3.600 & --\\

\textbf{B} & \num{1.36e+00} & 681.25 & \textbf{F:} 11.924, \textbf{G:} 8.400, \textbf{H:} 5.515 & 3.905 & --\\

\textbf{C} & \num{1.30e+00} & 651.61 & \textbf{F:} 13.672, \textbf{G:} 6.543, \textbf{H:} 16.275 & 4.291 & --\\

\textbf{D} & \num{1.43e+00} & 713.48 & \textbf{F:} 12.007, \textbf{G:} 5.141, \textbf{H:} 9.981 & 1.687 & --\\

\textbf{E} & \num{1.49e+00} & 744.53 & \textbf{F:} 8.715, \textbf{G:} 14.309, \textbf{H:} 9.805 & 3.214 & --\\
\hline
\multicolumn{6}{c}{\textbf{Five Multiplets, Run 3}}\\
\hline
\textbf{A} & \num{1.32e+00} & 658.87 & \textbf{F:} 4.984, \textbf{G:} 18.119, \textbf{H:} 10.642 & 3.846 & --\\

\textbf{B} & \num{1.44e+00} & 719.50 & \textbf{F:} 13.518, \textbf{G:} 14.381, \textbf{H:} 3.074 & 1.018 & --\\

\textbf{C} & \num{1.37e+00} & 683.34 & \textbf{F:} 8.758, \textbf{G:} 16.689, \textbf{H:} 12.956 & 4.766 & --\\

\textbf{D} & \num{1.35e+00} & 676.98 & \textbf{F:} 17.648, \textbf{G:} 7.514, \textbf{H:} 3.918 & 3.414 & --\\

\textbf{E} & \num{1.49e+00} & 746.76 & \textbf{F:} 16.396, \textbf{G:} 7.455, \textbf{H:} 11.352 & 1.827 & --\\

\hline
\multicolumn{6}{c}{\textbf{Strychinine}}\\
\hline
\textbf{A} & 7.167 & 2150.1 & \textbf{B}: $7.490$, \textbf{C}: $1.080$, \textbf{D}: $0.230$& 1.74& 1.26 \\
\textbf{B} & 7.098 & 2129.4 & \textbf{A}: $7.490$, \textbf{C}: $7.440$, \textbf{D}: $0.980$& 2.47& 1.78 \\
\textbf{C} & 7.255 & 2176.5 & \textbf{A}: $1.080$, \textbf{B}: $7.440$, \textbf{D}: $7.900$& 2.50& 1.80 \\
\textbf{D} & 8.092 & 2427.6 & \textbf{A}: $0.230$, \textbf{B}: $0.980$, \textbf{C}: $7.900$& 5.12& 3.69 \\
\textbf{E} & 3.860 & 1158.0 & \textbf{I}: $10.410$& 1.26& 0.91 \\
\textbf{F} & 3.132 & 939.6 & \textbf{G}: $-17.340$, \textbf{H}: $3.340$& 0.52& 0.37 \\
\textbf{G} & 2.670 & 801.0 & \textbf{F}: $-17.340$, \textbf{H}: $8.470$& 0.55& 0.39 \\
\textbf{H} & 4.288 & 1286.4 & \textbf{F}: $3.340$, \textbf{G}: $8.470$, \textbf{I}: $3.300$& 0.89& 0.64 \\
\textbf{I} & 1.276 & 382.8 & \textbf{E}: $10.410$, \textbf{H}: $3.300$, \textbf{J}: $3.290$& 0.97& 0.70 \\
\textbf{J} & 3.150 & 945.0 & \textbf{I}: $3.290$, \textbf{K}: $4.110$, \textbf{L}: $1.960$, \textbf{R}: $1.610$, \textbf{T}: $0.470$& 1.07& 0.77 \\
\textbf{K} & 2.360 & 708.0 & \textbf{J}: $4.110$, \textbf{L}: $-14.350$, \textbf{M}: $4.330$& 0.42& 0.30 \\
\textbf{L} & 1.462 & 438.6 & \textbf{J}: $1.960$, \textbf{K}: $-14.350$, \textbf{M}: $2.420$& 0.41& 0.30 \\
\textbf{M} & 3.963 & 1188.9 & \textbf{K}: $4.330$, \textbf{L}: $2.420$& 1.15& 0.83 \\
\textbf{N} & 1.890 & 567.0 & \textbf{O}: $-13.900$, \textbf{P}: $5.500$, \textbf{Q}: $7.200$& 0.48& 0.35 \\
\textbf{O} & 1.890 & 567.0 & \textbf{N}: $-13.900$, \textbf{P}: $3.200$, \textbf{Q}: $10.700$& 0.47& 0.34 \\
\textbf{P} & 3.219 & 965.7 & \textbf{N}: $5.500$, \textbf{O}: $3.200$, \textbf{Q}: $-13.900$& 0.49& 0.36 \\
\textbf{Q} & 2.878 & 863.4 & \textbf{N}: $7.200$, \textbf{O}: $10.700$, \textbf{P}: $-13.900$& 0.42& 0.30 \\
\textbf{R} & 3.716 & 1114.8 & \textbf{J}: $1.610$, \textbf{S}: $-14.800$, \textbf{T}: $1.790$& 0.47& 0.34 \\
\textbf{S} & 2.745 & 823.5 & \textbf{R}: $-14.800$& 0.45& 0.33 \\
\textbf{T} & 5.915 & 1774.5 & \textbf{J}: $0.470$, \textbf{R}: $1.790$, \textbf{U}: $7.000$, \textbf{V}: $6.100$& 1.48& 1.06 \\
\textbf{U} & 4.148 & 1244.4 & \textbf{T}: $7.000$, \textbf{V}: $-13.800$& 0.48& 0.34 \\
\textbf{V} & 4.066 & 1219.8 & \textbf{T}: $6.100$, \textbf{U}: $-13.800$& 0.55& 0.40 \\

\hline
\end{longtable}

