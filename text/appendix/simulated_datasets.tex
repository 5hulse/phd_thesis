\section{Information on simulated datasets}
\label{sec:simulated-datasets}

\subsection{\textsc{\textsc{Spinach}} Simulations}
Many of the simulated datasets presented in this work were generated using the
\textsc{Spinach} \textsc{Matlab}\textregistered\ package~\cite{Hogben2011}.
In each case, the dataset was generated via a call to the \texttt{new\_spinach}
method associated with the relevant \texttt{Estimator} object in \ac{EsPy}.
\texttt{new\_spinach} works by instantiating a \href{https://www.mathworks.com/help/matlab/matlab-engine-for-python.html}{\textsc{Matlab}
engine} from \textsc{Python}, which then runs a \textsc{\textsc{Spinach}} simulation of the relevant
experiment with the specifications provided. The \ac{FID} generated is then
stored within in a new \texttt{Estimator} object.
Table \ref{tab:shifts_and_couplings} provides a specification of the chemical
shifts and scalar couplings that made up the spin systems used to construct
simulated data with \textsc{Spinach}. Tables \ref{tab:spinach-jres-params} and
\ref{tab:spinach-invrec-params} specify key parameters used in each
\ac{2DJ} and inversion recovery simulation, respectively.

\subsubsection{Sucrose}
The chemical shifts and couplings for sucrose were extracted from a
\textsc{Gaussian}~\cite{Gaussian03} logfile which ships with \textsc{Spinach} at
the path \path{SPINACHROOT/examples/standard_systems/sucrose.log}.
Isotropic chemical shifts were determined for each spin $i$ via
\begin{equation}
    \delta_i = \frac{\Tr \left( \symbf{\sigma}_i \right)}{3},
\end{equation}
where $\symbf{\sigma}_i \in \mathbb{R}^{3 \times 3}$ is the computed chemical
shift tensor of spin $i$.

\subsubsection{Strychnine}
The strychnine spin system was derived from
the \textsc{Spinach} function \texttt{<SPINACHROOT>/etc/strychnine.m}, which returns a
spin system specification using chemical shifts and scalar couplings
from~\cite[Appendix 5]{Berger2004}, and atomic coordinates
from~\cite[Supplementary Material]{Butts2011}. To determine $T_1$ and  $T_2$
values for each spin, the relaxation superoperator $\symbf{R}$ was
constructed according to Redfield theory, under the assumption
that the molecule was undergoing spherical isotropic rotation with a rotational
correlation time of \qty{200}{\pico\second}, in a magnetic field of
\qty{700}{\mega\hertz}. Individual $T_1$s and $T_2$s were then extracted using
\begin{subequations}
    \begin{gather}
        T_{(1/2),i} = \frac{1}{R_{(1/2),i}}, \\
        R_{1,i} = - \Re \left(
            \symbf{l}_{z,i}^{\dagger} \symbf{R} \symbf{l}_{z,i}^{\vphantom{\dagger}}
            \right),\\
        R_{2,i} = - \Re \left(
            \symbf{l}_{+,i}^{\dagger} \symbf{R} \symbf{l}_{+,i}^{\vphantom{\dagger}}
            \right),
    \end{gather}
\end{subequations}
where $\symbf{l}_{z,i}^{\vphantom{\dagger}}$ is the state vector of the
Hilbert-space operator $\hat{L}_z$ for spin $i$, and
$\symbf{l}_{+,i}^{\vphantom{\dagger}}$ is the corresponding vector for the
$\hat{L}_+$ operator.


\begin{longtable}[h!]{c c c c c}
\caption[
The isotropic chemical shifts, scalar couplings and relaxation times
associated with spin systems used in \textsc{Spinach} simulations.
]{
The isotropic chemical shifts ($\delta$), corresponding rotating frame
frequencies ($\omega_0$), and scalar couplings ($J$)
associated with spin systems used in \textsc{Spinach} simulations.
For the ``Five multiplets'' spin systems, the associated $T_1$ times are provided too.
}
\label{tab:shifts_and_couplings}\\
\hline
Spin & $\delta$ (\unit{\partspermillion}) & $\omega_0$ (\unit{\hertz}) & $J$ (\unit{\hertz}) & $T_1$ (\unit{\second}) \\
\hline
\endfirsthead
\hline
Spin & $\delta$ (\unit{\partspermillion}) & $\omega_0$ (\unit{\hertz}) & $J$ (\unit{\hertz}) & $T_1$ (\unit{\second}) \\
\hline
\endhead
\hline
\endlastfoot
\multicolumn{5}{r}{Continues on next page...}\\
\hline
\endfoot
\hline
\multicolumn{5}{c}{\textbf{Five Multiplets, Run 1}}\\
\hline
\textbf{A} & \num{1.33e+00} & 665.99 & \textbf{F:} 10.965, \textbf{G:} 12.657, \textbf{H:} 17.070 & 2.178 \\

\textbf{B} & \num{1.47e+00} & 735.44 & \textbf{F:} 3.610, \textbf{G:} 2.543, \textbf{H:} 8.448 & 4.430 \\

\textbf{C} & \num{1.31e+00} & 653.87 & \textbf{F:} 10.630, \textbf{G:} 6.282, \textbf{H:} 3.012 & 3.319 \\

\textbf{D} & \num{1.41e+00} & 705.01 & \textbf{F:} 8.101, \textbf{G:} 4.589, \textbf{H:} 9.068 & 1.007 \\

\textbf{E} & \num{1.30e+00} & 650.23 & \textbf{F:} 3.014, \textbf{G:} 16.537, \textbf{H:} 15.587 & 4.992 \\
\hline
\multicolumn{5}{c}{\textbf{Five Multiplets, Run 2}}\\
\hline
\textbf{A} & \num{1.47e+00} & 733.07 & \textbf{F:} 19.488, \textbf{G:} 18.279, \textbf{H:} 3.147 & 3.600 \\

\textbf{B} & \num{1.36e+00} & 681.25 & \textbf{F:} 11.924, \textbf{G:} 8.400, \textbf{H:} 5.515 & 3.905 \\

\textbf{C} & \num{1.30e+00} & 651.61 & \textbf{F:} 13.672, \textbf{G:} 6.543, \textbf{H:} 16.275 & 4.291 \\

\textbf{D} & \num{1.43e+00} & 713.48 & \textbf{F:} 12.007, \textbf{G:} 5.141, \textbf{H:} 9.981 & 1.687 \\

\textbf{E} & \num{1.49e+00} & 744.53 & \textbf{F:} 8.715, \textbf{G:} 14.309, \textbf{H:} 9.805 & 3.214 \\
\hline
\multicolumn{5}{c}{\textbf{Five Multiplets, Run 3}}\\
\hline
\textbf{A} & \num{1.32e+00} & 658.87 & \textbf{F:} 4.984, \textbf{G:} 18.119, \textbf{H:} 10.642 & 3.846 \\

\textbf{B} & \num{1.44e+00} & 719.50 & \textbf{F:} 13.518, \textbf{G:} 14.381, \textbf{H:} 3.074 & 1.018 \\

\textbf{C} & \num{1.37e+00} & 683.34 & \textbf{F:} 8.758, \textbf{G:} 16.689, \textbf{H:} 12.956 & 4.766 \\

\textbf{D} & \num{1.35e+00} & 676.98 & \textbf{F:} 17.648, \textbf{G:} 7.514, \textbf{H:} 3.918 & 3.414 \\

\textbf{E} & \num{1.49e+00} & 746.76 & \textbf{F:} 16.396, \textbf{G:} 7.455, \textbf{H:} 11.352 & 1.827 \\

\hline
\multicolumn{5}{c}{\textbf{Four Multiplets, Run 1}}\\
\hline
\textbf{A} & \num{-2.78e-02} & -13.93 & \textbf{E:} -9.627, \textbf{F:} -8.202, \textbf{G:} 6.742 & -- \\

\textbf{B} & \num{-7.11e-03} & -3.56 & \textbf{E:} -4.491, \textbf{F:} 5.333, \textbf{G:} 9.303 & -- \\

\textbf{C} & \num{-1.63e-03} & -0.81 & \textbf{E:} 3.953, \textbf{F:} 5.422, \textbf{G:} 5.914 & -- \\

\textbf{D} & \num{1.53e-02} & 7.66 & \textbf{E:} -7.902, \textbf{F:} -4.556, \textbf{G:} 6.217 & -- \\
\hline
\multicolumn{5}{c}{\textbf{Four Multiplets, Run 2}}\\
\hline
\textbf{A} & \num{-1.48e-02} & -7.38 & \textbf{E:} -7.492, \textbf{F:} 0.917, \textbf{G:} 2.933 & -- \\

\textbf{B} & \num{-1.18e-02} & -5.88 & \textbf{E:} -4.304, \textbf{F:} -1.815, \textbf{G:} 5.420 & -- \\

\textbf{C} & \num{-4.32e-03} & -2.16 & \textbf{E:} 4.832, \textbf{F:} 7.573, \textbf{G:} 8.268 & -- \\

\textbf{D} & \num{1.76e-02} & 8.80 & \textbf{E:} -9.244, \textbf{F:} -1.816, \textbf{G:} -0.478 & -- \\
\hline
\multicolumn{5}{c}{\textbf{Four Multiplets, Run 3}}\\
\hline
\textbf{A} & \num{-2.23e-02} & -11.17 & \textbf{E:} -5.347, \textbf{F:} -1.851, \textbf{G:} 1.407 & -- \\

\textbf{B} & \num{1.13e-02} & 5.66 & \textbf{E:} 6.425, \textbf{F:} 7.291, \textbf{G:} 9.806 & -- \\

\textbf{C} & \num{2.53e-02} & 12.64 & \textbf{E:} -8.640, \textbf{F:} 0.613, \textbf{G:} 6.998 & -- \\

\textbf{D} & \num{2.84e-02} & 14.21 & \textbf{E:} -8.613, \textbf{F:} 0.782, \textbf{G:} 3.830 & -- \\
\hline
\multicolumn{5}{c}{\textbf{Four Multiplets, Run 4}}\\
\hline
\textbf{A} & \num{-2.03e-02} & -10.16 & \textbf{E:} -8.646, \textbf{F:} 6.719, \textbf{G:} 7.921 & -- \\

\textbf{B} & \num{3.53e-03} & 1.77 & \textbf{E:} -8.857, \textbf{F:} 4.314, \textbf{G:} 9.197 & -- \\

\textbf{C} & \num{8.61e-03} & 4.30 & \textbf{E:} -0.620, \textbf{F:} 1.767, \textbf{G:} 6.567 & -- \\

\textbf{D} & \num{2.06e-02} & 10.30 & \textbf{E:} -9.060, \textbf{F:} 2.355, \textbf{G:} 9.810 & -- \\
\hline
\multicolumn{5}{c}{\textbf{Four Multiplets, Run 5}}\\
\hline
\textbf{A} & \num{-9.16e-03} & -4.58 & \textbf{E:} -8.281, \textbf{F:} 1.621, \textbf{G:} 3.229 & -- \\

\textbf{B} & \num{-2.79e-03} & -1.40 & \textbf{E:} 1.655, \textbf{F:} 4.219, \textbf{G:} 6.998 & -- \\

\textbf{C} & \num{3.00e-03} & 1.50 & \textbf{E:} -4.280, \textbf{F:} 1.045, \textbf{G:} 5.896 & -- \\

\textbf{D} & \num{2.74e-02} & 13.72 & \textbf{E:} -9.316, \textbf{F:} -8.322, \textbf{G:} -3.938 & -- \\

\hline
\multicolumn{5}{c}{\textbf{Strychinine}}\\
\hline
\textbf{A} & 7.167 & 3583.5 & \textbf{B}: $7.490$, \textbf{C}: $1.080$, \textbf{D}: $0.230$& -- \\
\textbf{B} & 7.098 & 3549.0 & \textbf{A}: $7.490$, \textbf{C}: $7.440$, \textbf{D}: $0.980$& -- \\
\textbf{C} & 7.255 & 3627.5 & \textbf{A}: $1.080$, \textbf{B}: $7.440$, \textbf{D}: $7.900$& -- \\
\textbf{D} & 8.092 & 4046.0 & \textbf{A}: $0.230$, \textbf{B}: $0.980$, \textbf{C}: $7.900$& -- \\
\textbf{E} & 3.860 & 1930.0 & \textbf{I}: $10.410$& -- \\
\textbf{F} & 3.132 & 1566.0 & \textbf{G}: $-17.340$, \textbf{H}: $3.340$& -- \\
\textbf{G} & 2.670 & 1335.0 & \textbf{F}: $-17.340$, \textbf{H}: $8.470$& -- \\
\textbf{H} & 4.288 & 2144.0 & \textbf{F}: $3.340$, \textbf{G}: $8.470$, \textbf{I}: $3.300$& -- \\
\textbf{I} & 1.276 & 638.0 & \textbf{E}: $10.410$, \textbf{H}: $3.300$, \textbf{J}: $3.290$& -- \\
\textbf{J} & 3.150 & 1575.0 & \textbf{I}: $3.290$, \textbf{K}: $4.110$, \textbf{L}: $1.960$, \textbf{R}: $1.610$, \textbf{T}: $0.470$& -- \\
\textbf{K} & 2.360 & 1180.0 & \textbf{J}: $4.110$, \textbf{L}: $-14.350$, \textbf{M}: $4.330$& -- \\
\textbf{L} & 1.462 & 731.0 & \textbf{J}: $1.960$, \textbf{K}: $-14.350$, \textbf{M}: $2.420$& -- \\
\textbf{M} & 3.963 & 1981.5 & \textbf{K}: $4.330$, \textbf{L}: $2.420$& -- \\
\textbf{N} & 1.890 & 945.0 & \textbf{O}: $-13.900$, \textbf{P}: $5.500$, \textbf{Q}: $7.200$& -- \\
\textbf{O} & 1.890 & 945.0 & \textbf{N}: $-13.900$, \textbf{P}: $3.200$, \textbf{Q}: $10.700$& -- \\
\textbf{P} & 3.219 & 1609.5 & \textbf{N}: $5.500$, \textbf{O}: $3.200$, \textbf{Q}: $-13.900$& -- \\
\textbf{Q} & 2.878 & 1439.0 & \textbf{N}: $7.200$, \textbf{O}: $10.700$, \textbf{P}: $-13.900$& -- \\
\textbf{R} & 3.716 & 1858.0 & \textbf{J}: $1.610$, \textbf{S}: $-14.800$, \textbf{T}: $1.790$& -- \\
\textbf{S} & 2.745 & 1372.5 & \textbf{R}: $-14.800$& -- \\
\textbf{T} & 5.915 & 2957.5 & \textbf{J}: $0.470$, \textbf{R}: $1.790$, \textbf{U}: $7.000$, \textbf{V}: $6.100$& -- \\
\textbf{U} & 4.148 & 2074.0 & \textbf{T}: $7.000$, \textbf{V}: $-13.800$& -- \\
\textbf{V} & 4.066 & 2033.0 & \textbf{T}: $6.100$, \textbf{U}: $-13.800$& -- \\

\hline
\end{longtable}


\begin{table}[h!]
\centering
\begin{tabular}{ccc}
\hline
Parameter & Four Multiplets & Sucrose\\
\hline
$f_{\text{bf}}^{(1)} (\unit{\mega\hertz})$ & 500 & 300\\
$\fofftwo$ (\unit{\hertz}) & 0 & 1000\\
$\fofftwo$ (\unit{\partspermillion}) & 0 & 3.333\\
$\fswone$ (\unit{\hertz}) & 40 & 30\\
$\fswtwo$ (\unit{\hertz}) & 1000 & 2200\\
$\fswtwo$ (\unit{\partspermillion}) & 2 & 7.333\\
$\None$ & 128 & 64\\
$\Ntwo$ & 1024 & 4096\\
\hline
\end{tabular}
\caption[
    Experiment parameters for \ac{2DJ} simulations run using \textsc{Spinach}.
]{
    Experiment parameters for \ac{2DJ} simulations run using \textsc{Spinach}.
}
\label{tab:spinach-jres-params}
\end{table}


\begin{table}[h!]
\centering
\begin{tabular}{cc}
\hline
Parameter &Five Multiplets\\
\hline
$f_{\text{bf}}^{(1)} (\unit{\mega\hertz})$ & 500\\
$\foffone$ (\unit{\hertz}) & 2500\\
$\foffone$ (\unit{\partspermillion}) & 5\\
$\fswone$ (\unit{\hertz}) & 5000\\
$\fswone$ (\unit{\partspermillion}) & 10\\
$\None$ & 16384\\
$K$ & 21\\
$\tau_{\text{max}}$ (\unit{\second}) & 4\\
\hline
\end{tabular}
\caption[
    Experiment parameters for inversion recovery simulations run using \textsc{Spinach}.
]{
    Experiment parameters for inversion recovery simulations run using \textsc{Spinach}.
    $K$ specifies the number of increments run, and $\tau_{\text{max}}$
    specifies the largest delay time used. Delays were generated with linear spacings,
    with the first delay always being \qty{0}{\second}, such that the
    n\textsuperscript{th} delay was $\nicefrac{(n-1)\tau_{\text{max}}}{K -
    1}$.
}
\label{tab:spinach-invrec-params}
\end{table}

\note{Table of spinach experiment parameters}
\note{Description of 2DJ and Invrec simulations (i.e. relaxation model used, approximations to basis used etc.}
