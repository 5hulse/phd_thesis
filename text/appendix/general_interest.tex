\chapter{Miscellaneous Topics of Interest}
\label{chap:misc}

\subsection{The Nyquist Frequency}
\label{sec:nyquist}
The Nyquist frequency defines the highest frequency an oscillator can possess
such that it is sampled at least twice per period:
\begin{equation}
  f_{\mathrm{N}} = \frac{1}{2\Delta t}
\end{equation}
This quantity as at the heart of the Nyquist-Shannon Sampling Theorem, which -
in the words of Claude Shannon\cite{Shannon1949} - states:
\begin{quote}
  If a function $y(t)$ contains no frequencies
  higher than $f \si{\hertz}$, it is completely determined by giving
  its ordinates at a series of points spaced $\nicefrac{1}{2f}$ seconds
  apart.
\end{quote}
The implication of this is that the continuous FID $y(t)$ may be completely
described by its digitisation with sampling interval $\Delta t$ so long as the
frequencies $f_j$ that make it up satisfy:
\begin{equation}
  \label{permitted_freq}
  -\frac{1}{2\Delta t} \leq f_j \leq \frac{1}{2\Delta t}\ \forall j
\end{equation}
If any frequency $f_j$ does not satisfy \eqref{eq:permitted_freq}, the signal
will be insufficiently sampled, and will be spuriously represented as having a
frequency $f_a$ that satisfies
\begin{alignat}{2}
  &f_a = 2mf_{\mathrm{N}} - f_j \quad \quad &&(f_j > f_{\mathrm{N}})\\
  &f_a = 2mf_{\mathrm{N}} + f_j \quad \quad &&(f_j < -f_{\mathrm{N}}),
\end{alignat}
where $m \in \mathbb{Z}_+$. This phenomenon is called \textit{aliasing}.

\subsection{Hessian Validation}
\note{Talk about checking that my Hessian code is correct by using the finite difference method}

\subsection{Statistical Definitions}
\note{Definitions of standard error, likelihood function, ...}
