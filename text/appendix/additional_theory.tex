\chapter{Additional Theory}

\section{Mathematical definitions}

Descriptions of linear algebra and statics concepts that are referred to in
this thesis are provided here. More detail can be found in numerous relevant
texts\cite{Strang2018,Pawitan2001}.

\subsection{Linear algebra}

\note{TODO: rank, range}

\subsubsection{\Acl{SVD}}
\ac{SVD} is a generalisation of eigendecomposition for a matrix of any shape.
Given a matrix $\symbf{A} \in \mathbb{C}^{m \times n}$, the \ac{SVD} is a
factorisation given by
\begin{equation}
    \symbf{A} = \symbf{U} \symbf{\Sigma} \symbf{V}^{\dagger}.
\end{equation}
The matrices that make up the decomposition are as follows
\begin{itemize}[label={}]
    \item $\symbf{\Sigma} \in \mathbb{C}^{m \times n}$ is a rectangular
        diagonal matrix with diagonal elements comprising the \emph{singular
        values} in descending order of magnitude. The singular values are the
        square roots of the non-zero eigenvalues of both
        $\symbf{A}^{\dagger}\symbf{A}$ and
        $\symbf{A}\symbf{A}^{\dagger}$.
    \item $\symbf{U} \in \mathbb{C}^{m \times m}$ is a unitary matrix whose columns
        comprise the \emph{left singular vectors}. The left singular vectors
        are the eigenvectors of the matrix $\symbf{A}\symbf{A}^{\dagger}$.
    \item $\symbf{V} \in \mathbb{C}^{n \times n}$ is a unitary matrix whose columns
        comprise the \emph{right singular vectors}. The right singular vectors
        are the eigenvectors of the matrix $\symbf{A}^{\dagger}\symbf{A}$.
\end{itemize}
One fundamental property of the \ac{SVD} is that the number of non-zero
singular values is equivalent to the rank of the matrix. As the \ac{EYM}
theorem highlights, the \ac{SVD} is valuable in constructing low-rank
approximations of matrices, with applications in various fields such as signal
processing, (see Section \ref{subsec:mpm}) and data compression.

\subsection{Statistics and probability}

\note{TODO: Likelihood function, PDF, MLE}

\section{Additional algorithms}

\begin{algorithm}
    \caption{The \acs{MMEMPM}.}
    \label{alg:mmempm}
    \begin{algorithmic}[1]
        \Procedure {MMEMPM}{$\symbf{Y} \in \mathbb{C}^{\None \times \Ntwo}, M \in \mathbb{N}$}
        \State $\Lone, \Ltwo \gets \left\lfloor \nicefrac{\None}{2} \right\rfloor, \left\lfloor \nicefrac{\Ntwo}{2} \right\rfloor$;
        \For{$\none \gets \lbrace 0, \cdots, \None - 1 \rbrace$}
            \State  $\symbf{H}_{\symbf{Y},\none} \gets
                \def\arraystretch{1.4}
            \begin{bmatrix}
                \symbf{Y}\left[\none, 0\right] &
                \symbf{Y}\left[\none, 1\right] &
                \cdots &
                \symbf{Y}\left[\none, \Ntwo-L^{(2)}\right]\\
                \symbf{Y}\left[\none, 1\right] &
                \symbf{Y}\left[\none, 2\right] &
                \cdots &
                \symbf{Y}\left[\none, \Ntwo-L^{(2)}+1\right]\\
                \vdots & \vdots & \ddots & \vdots\\
                \symbf{Y}\left[\none, L^{(2)} - 1\right] &
                \symbf{Y}\left[\none, L^{(2)}\right] &
                \cdots &
                \symbf{Y}\left[\none, \Ntwo-1\right]\\
            \end{bmatrix}
        $
        \EndFor;
        \State $\symbf{E}_{\symbf{Y}} \gets
        \begin{bmatrix}
            \symbf{H}_{\symbf{Y},0} & \symbf{H}_{\symbf{Y},1} & \cdots & \symbf{H}_{\symbf{Y},\None - L^{(1)}}\\
            \symbf{H}_{\symbf{Y},1} & \symbf{H}_{\symbf{Y},2} & \cdots & \symbf{H}_{\symbf{Y},\None - L^{(1)} + 1}\\
            \vdots & \vdots & \ddots & \vdots\\
            \symbf{H}_{\symbf{Y},L^{(1)} - 1} & \symbf{H}_{\symbf{Y},L^{(1)}} & \cdots & \symbf{H}_{\symbf{Y},\None - 1}
        \end{bmatrix}
        $;
        \State $\symbf{U}_M^{\vphantom{\dagger}},
            \symbf{\Sigma}_M^{\vphantom{\dagger}},
            \symbf{V}_M^{\dagger} \gets
            \textsc{TruncatedSVD}\left(\EY, M\right)$;
        \State $\symbf{P} \gets \symbf{0} \in \mathbb{C}^{\Lone \Ltwo \times \Lone \Ltwo}$;
        \State $r \gets 0$
        \For{$i = 0, \cdots, \Ltwo - 1$}
            \For{$j = 0, \cdots, \Lone - 1$}
                \State $c \gets i + j \Ltwo$;
                \State $\symbf{P}\left[r, c\right] \gets 1$;
                \State $r = r + 1$;
            \EndFor
        \EndFor
        \State $\symbf{U}_{M1}, \symbf{U}_{M2} \gets \symbf{U}_M\left[ : L^{(1)}(L^{(2)}-1)\right], \symbf{U}_M\left[L^{(2)}:\right]$;
        \Comment{Last/First $L^{(2)}$ rows deleted}
        \State $\symbf{z}^{(1)}, \symbf{W}^{(1)} \gets \textsc{Eigendecomposition}\left( \symbf{U}_{M1}^+ \symbf{U}_{M2}^{\vphantom{+}} \right)$;
        \State $
            \symbf{f}^{(1)},
            \symbf{\eta}^{(1)} \gets
            \left(
                \nicefrac{f_{\text{sw}}^{(1)}}{2 \pi}
            \right)
            \Im \left( \ln \symbf{z}^{(1)} \right) + \foffone,
            -\fswone \Re \left( \ln \symbf{z}^{(1)} \right)
        $;
        \State $\symbf{U}_{M\text{P}} \gets \symbf{P} \symbf{U}_M$;
        \State $
            \symbf{U}_{M\text{P}1},
            \symbf{U}_{M \text{P} 2} \gets
            \symbf{U}_{M \text{P}}\left[ : (L^{(1)}-1)L^{(2)}\right],
            \symbf{U}_{M \text{P}}\left[L^{(1)}:\right]
        $;
        \Comment{Last/First $L^{(1)}$ rows deleted}
        \State $
            \symbf{z}^{(2)} \gets \operatorname{diag} \left(
                \left[\symbf{W}^{(1)}\right]^{-1}
                \symbf{U}_{M\text{P}1}^{+}
                \symbf{U}_{M\text{P}2}^{\vphantom{+}}
                \symbf{W}^{(1)}
            \right)$;
        \State $
            \bdftwo,
            \bdetatwo \gets
            \left(
                \nicefrac{\fswtwo}{2 \pi}
            \right)
            \Im \left( \ln \symbf{z}^{(2)} \right) + \fofftwo,
            -\fswtwo \Re \left( \ln \symbf{z}^{(2)} \right)
        $;
        \State $
        \symbf{Z}^{(2)}_{\text{L}} =
        \begin{bmatrix}
            \symbf{1} &
            \bdztwo &
            {\bdztwo}^2 &
            \cdots &
            {\bdztwo}^{\Ltwo-1}
        \end{bmatrix}\T
        $;
        \State $
            \symbf{Z}^{(2)}_{\text{R}} \gets
            \begin{bmatrix}
                \symbf{1} & \bdztwo & {\bdztwo}^2 & \cdots & {\bdztwo}^{\Ntwo - \Ltwo}
            \end{bmatrix}
        $;
        \State $\symbf{Z}^{(1)}_{\text{D}} \gets \diag\left(\bdzone\right)$;
       \State $
            \symbf{E}_{\text{L}} \gets
            \begin{bmatrix}
                \symbf{Z}^{(2)}_{\text{L}} \\
                \symbf{Z}^{(2)}_{\text{L}} \symbf{Z}^{(1)}_{\text{D}} \\
                \vdots\\
                \symbf{Z}^{(2)}_{\text{L}} \left[\symbf{Z}^{(1)}_{\text{D}}\right]^{\Lone - 1}
            \end{bmatrix}
        $;
        \State $
            \symbf{E}_{\text{R}} \gets
            \begin{bmatrix}
                \symbf{Z}^{(2)}_{\text{R}} &
                \symbf{Z}^{(1)}_{\text{D}} \symbf{Z}^{(2)}_{\text{R}} &
                \cdots &
                \left[\symbf{Z}^{(1)}_{\text{D}}\right]^{\None - \Lone} \symbf{Z}^{(2)}_{\text{R}}
            \end{bmatrix}
        $;
        \State $
           \bdalpha \gets \diag
           \left(
               \symbf{E}_{\text{L}}^+
               \symbf{E}_{\symbf{Y}}^{\vphantom{+}}
               \symbf{E}_{\text{R}}^+
           \right)
           $;
        \State $
            \bda, \bdphi \gets
            \left\lvert \bdalpha \right\rvert,
            \arctan \left( \frac{\Im\left(\bdalpha\right)}{\Re\left(\bdalpha\right)} \right)
        $;
        \State $\symbf{\theta}^{(0)} \gets
        \begin{bmatrix}
            \bda\T &
            \bdphi\T &
            \left[\symbf{f}^{(1)}\right]\T &
            \left[\symbf{f}^{(2)}\right]\T &
            \left[\symbf{\eta}^{(1)}\right]\T &
            \left[\symbf{\eta}^{(2)}\right]\T
        \end{bmatrix}
        ^{\mathrm{T}}$;
        \State \textbf{return} $\symbf{\theta}^{(0)}$
    \EndProcedure
    \end{algorithmic}
\end{algorithm}



\begin{algorithm}
    \caption[
        Steihaug-Toint method for determining an update for nonlinear
        programming.
    ]{
        Steihaug-Toint method for determining an update for nonlinear
        programming. This is equivalent to Algorithm 7.2 in \cite{Nocedal2006}.
    }
    \label{alg:steihaug-toint}
    \begin{algorithmic}[1]
        \Procedure{SteihaugToint}{
            $\bY \in \mathbb{C}^{\None \times \cdots \times \ND},
            \bthk \in \mathbb{R}^{2(1 + D)M},
            \trustradius{k} \in \mathbb{R}_{>0}
            $
        }
            \State $\symbf{g} \gets \nabla \FphithkY$;
            \Comment{Grad vector: \eqref{eq:grad}}
            \State $\symbf{H} \gets \nabla^2 \FphithkY$;
            \Comment{Hessian matrix, either exact: \eqref{eq:hess} or approximate: \eqref{eq:hess-approx}}
            \State $\epsilon^{(k)} \gets \min\left(
                    \nicefrac{1}{2},
                    \sqrt{\left\lVert \symbf{g} \right\rVert}
                \right)
                \left\lVert \symbf{g} \right\rVert
                $;
            \State $\symbf{z}^{(0)} \gets \symbf{0} \in \mathbb{R}^{6M}$;
            \State $\symbf{r}^{(0)} \gets \symbf{g}$;
            \State $\symbf{d}^{(0)} \gets -\symbf{r}^{(0)}$;
            \If {$\left \lVert \symbf{r}^{(0)} \right \rVert < \epsilon^{(k)}$}
                \State \textbf{return} $\symbf{z}^{(0)}$;
            \EndIf
            \For {$j = \lbrace 0, 1, \cdots \rbrace$}
                \If {
                    ${\symbf{d}^{(j)}}^{\mathrm{T}}
                    \symbf{H}
                    \symbf{d}^{(j)}
                    \leq 0$
                }
                \State Find $\tau$ such that $\symbf{p}^{(k)} = \symbf{z}^{(j)} + \tau \symbf{d}^{(j)}$
                    minimises $\FphiQthkpk$, subject to
                    $\left \lVert \symbf{p}^{(k)} \right \rVert = \trustradius{k}$;
                    \State \textbf{return} $\symbf{p}^{(k)}$;
                \EndIf
                \State $\alpha^{(j)} \gets \dfrac
                    {{\symbf{r}^{(j)}}^{\mathrm{T}} \symbf{r}^{(j)}}
                    {
                        {\symbf{d}^{(j)}}^{\mathrm{T}}
                        \symbf{H}
                        \symbf{d}^{(j)}
                    }$;
                \State $\symbf{z}^{(j+1)} \gets \symbf{z}^{(j)} + \alpha^{(j)} \symbf{d}^{(j)}$;
                \If {$\left \lVert \symbf{z}^{(j+1)} \right \rVert < \epsilon^{(k)}$}
                    \State Find $\tau \in \mathbb{R}_{>0}$ such that
                        $\symbf{p}^{(k)} = \symbf{z}^{(j)} + \tau \symbf{d}^{(j)}$
                        satisfies
                        $\left \lVert \symbf{p}^{(k)} \right \rVert = \trustradius{k}$;
                    \State \textbf{return} $\symbf{p}^{(k)}$;
                \EndIf
                \State $\symbf{r}^{(j+1)} \gets
                    \symbf{r}^{(j)} +
                    \alpha^{(j)}
                    \symbf{H}
                    \symbf{d}^{(j)}$;
                    \If{$\left\lVert \symbf{r}^{(j+1)} \right \rVert < \epsilon^{(k)}$}
                    \State \textbf{return} $\symbf{z}^{(j+1)}$;
                \EndIf
                \State $\beta^{(j+1)} \gets
                    \dfrac{{\symbf{r}^{(j+1)}}^{\mathrm{T}} \symbf{r}^{(j+1)}}{{\symbf{r}^{(j)}}^{\mathrm{T}} \symbf{r}^{(j)}}$;
                \State $\symbf{d}^{(j+1)} \gets -\symbf{r}^{(j+1)} + \beta^{(j+1)} \symbf{d}^{(j)}$;
            \EndFor
        \EndProcedure
    \end{algorithmic}
\end{algorithm}



\begin{algorithm}[h!]
    \begin{algorithmic}[1]
        \caption[
            Filtering procedure for 1D data.
        ]
        {
            Filtering procedure for 1D data.
            $\symbf{r}_{\text{interest}}$ is a vector of length 2 containing
            the indices of the left and right bounds of the region of interest.
            These would typically be provided in units of \unit{\hertz} or
            \unit{\partspermillion} by a user. Conversion to array indices can
            be carried out using \eqref{eq:fidx}.
            $\symbf{r}_{\text{noise}}$ contains the left and right bounds of
            the region used to estimate the noise variance.
            \textsc{RandomSample} indicates taking a random sample from the
            given distribution.
        }
        \label{alg:filter-1d}
        \Procedure{Filter$1$D}{
            $\by \in \mathbb{C}^{\None},
            \symbf{r}_{\text{interest}} \in \mathbb{N}_0^2,
            \symbf{r}_{\text{noise}} \in \mathbb{N}_0^2
            $}
            \State $\by_{\text{ve}} \gets \textsc{VirtualEcho$1$D}\left(\by\right)$;
            \State $\symbf{s}_{\text{ve}} \gets \FT\left(\by_{\text{ve}}\right)$;
            \State $l^{(1)}_{\text{idx}}, r^{(1)}_{\text{idx}} \gets \symbf{r}_{\text{interest}}[0], \symbf{r}_{\text{interest}}[1]$;
            \State $l^{(1)}_{\text{idx,noise}}, r^{(1)}_{\text{idx,noise}} \gets \symbf{r}_{\text{noise}}[0], \symbf{r}_{\text{noise}}[1]$;
            \State $c_{\text{idx}}^{(1)} \gets \nicefrac{\left(l_{\text{idx}}^{(1)} + r_{\text{idx}}^{(1)}\right)}{2}$;
            \State $b_{\text{idx}}^{(1)} \gets r_{\text{idx}}^{(1)} - l_{\text{idx}}^{(1)}$;
            \State $\symbf{g} \gets \textsc{SuperGaussian$1$D}\left(\None_{\vphantom{\text{idx}}}, c_{\text{idx}}^{(1)}, b_{\text{idx}}^{(1)}\right)$;
            \State $\symbf{s}_{\text{noise}} \gets \symbf{s}_{\text{ve}} \left[
                l^{(1)}_{\text{idx,noise}} : r^{(1)}_{\text{idx,noise}} + 1
            \right]
            $;
            \State $\sigma^2 \gets \Var\left(\symbf{s}_{\text{noise}}\right)$;
            \State $\symbf{w}_{\sigma^2} \gets \symbf{0} \in \mathbb{R}^{2\None}$;
            \For {$\none = 0, \cdots, \None - 1$}
                \State $\symbf{w}_{\sigma^2}\left[ \none \right] \gets \textsc{RandomSample}\left(\mathcal{N}\left(0, \sigma^2\right)\right)$;
            \EndFor
            \State $\widetilde{\symbf{s}}_{\text{ve}} \gets \symbf{s}_{\text{ve}} \odot \symbf{g} + \symbf{w}_{\sigma^2} \odot \left(\symbf{1} - \symbf{g}\right)$;
            \State $\widetilde{\symbf{y}}_{\text{ve}} \gets \IFT \left( \widetilde{\symbf{s}}_{\text{ve}} \right)$;
            \State $\widetilde{\symbf{y}} \gets \widetilde{\symbf{y}}_{\text{ve}}\left[ : \None \right]$;
            \State \textbf{return} $\widetilde{\symbf{y}}$;
        \EndProcedure
        \Statex
        \Procedure{VirtualEcho$1$D}{$\by \in \mathbb{C}^{\None}$}
            \State $\symbf{t}_1 \gets
            \begin{bmatrix}
                \by \\ \symbf{0} \in \mathbb{C}^{\None}
            \end{bmatrix}
            $;
            \State $\symbf{t}_2 \gets
            \begin{bmatrix}
            \symbf{0} \in \mathbb{C}^{\None} \\ {\by^*}^{\leftrightsquigarrow (1)}
            \end{bmatrix}^{\circlearrowright (1)}
            $;
            \State $\by_{\text{ve}} \gets \symbf{t}_1 + \symbf{t}_2$;
            \State $\by_{\text{ve}}[0] \gets \nicefrac{\by_{\text{ve}}[0]}{2}$;
            \State \textbf{return} $\by_{\text{ve}}$;
        \EndProcedure
        \Statex
        \Procedure{SuperGaussian$1$D}{$N \in \mathbb{N}, c_{\text{idx}} \in \mathbb{R}_{>0}, b_{\text{idx}} \in \mathbb{N}$}
            \State $\symbf{g} \gets \symbf{0} \in \mathbb{R}^{N}$;
            \For {$n = 0, \cdots, N - 1$}
                \State $\symbf{g}\left[ n \right] \gets \exp\left(
                    -2^{41} \left(
                        \frac{n - c_{\text{idx}}}{b_{\text{idx}}}
                    \right)^{40}
                    \right)
                $;
                \Comment{$p$ in \eqref{eq:super-Gaussian-onedim} has been set to 40.}
            \EndFor
            \State \textbf{return} $\symbf{g}$
        \EndProcedure
    \end{algorithmic}
\end{algorithm}

\begin{algorithm}[h!]
    \begin{algorithmic}[1]
        \caption{Filtering procedure for 2D data.}
        \label{alg:filter-2d}
        \Procedure{Filter$2$D}{$\bY_{\cos} \in \mathbb{C}^{\None \times \Ntwo}, \bY_{\sin} \in \mathbb{C}^{\None \times \Ntwo}, \symbf{R}_{\text{interest}} \in \mathbb{N}_0^{2 \times 2}, \symbf{R}_{\text{noise}} \in \mathbb{N}_0^{2 \times 2}$}
            \State $\bY_{\text{ve}} \gets \textsc{VirtualEcho$2$D}\left(\bY_{\cos}, \bY_{\sin}\right)$;
            \State $\symbf{S}_{\text{ve}} \gets \FT\left(\bY_{\text{ve}}\right)$;
            \State $
                l^{(1)}_{\text{idx}},
                r^{(1)}_{\text{idx}},
                l^{(2)}_{\text{idx}},
                r^{(2)}_{\text{idx}}
                \gets
                \symbf{R}_{\text{interest}}[0,0],
                \symbf{R}_{\text{interest}}[0,1],
                \symbf{R}_{\text{interest}}[1,0],
                \symbf{R}_{\text{interest}}[1,1]
                $;
            \For{$d = 1, 2$}
                \State $c_{\text{idx}}^{(d)} \gets \nicefrac{\left(l_{\text{idx}}^{(d)} + r_{\text{idx}}^{(d)}\right)}{2}$;
                \State $b_{\text{idx}}^{(d)} \gets r_{\text{idx}}^{(d)} - l_{\text{idx}}^{(d)}$;
                \State $\symbf{g}^{(d)} \gets \textsc{SuperGaussian$1$D}\left( 2\Nd_{\vphantom{\text{idx}}}, c_{\text{idx}}^{(d)}, b_{\text{idx}}^{(d)} \right)$;
                \State $\symbf{G} \gets  \symbf{g}^{(1)} \otimes \symbf{g}^{(2)}$;
            \EndFor
            \State $
                l^{(1)}_{\text{idx,noise}},
                r^{(1)}_{\text{idx,noise}},
                l^{(2)}_{\text{idx,noise}},
                r^{(2)}_{\text{idx,noise}}
                \gets
                \symbf{R}_{\text{noise}}[0,0],
                \symbf{R}_{\text{noise}}[0,1],
                \symbf{R}_{\text{noise}}[1,0],
                \symbf{R}_{\text{noise}}[1,1]
                $;
            \State $\symbf{S}_{\text{noise}} \gets \symbf{S}_{\text{ve}} \left[
                    l^{(1)}_{\text{idx,noise}} :
                    r^{(1)}_{\text{idx,noise}} + 1,
                    l^{(2)}_{\text{idx,noise}} :
                    r^{(2)}_{\text{idx,noise}} + 1
                \right] $
            \State $\sigma^2 \gets \Var\left(\symbf{S}_{\text{noise}}\right)$;
            \State $\symbf{W}_{\sigma^2} \gets \symbf{0} \in \mathbb{R}^{2\None \times 2\Ntwo}$;
            \For {$\none = 0, \cdots, 2\None - 1$}
                \For {$\ntwo = 0, \cdots, 2\Ntwo - 1$}
                    \State $\symbf{W}_{\sigma^2}\left[ \none, \ntwo \right] \gets \textsc{RandomSample}\left(\mathcal{N}\left(0, \sigma^2\right)\right)$;
                \EndFor
            \EndFor
            \State $\widetilde{\symbf{S}}_{\text{ve}} \gets \symbf{S}_{\text{ve}} \odot \symbf{G} + \symbf{W}_{\sigma^2} \odot \left(\symbf{1} - \symbf{G}\right)$;
            \State $\widetilde{\symbf{Y}}_{\text{ve}} \gets \IFT \left( \widetilde{\symbf{S}}_{\text{ve}} \right)$;
            \State $\widetilde{\symbf{Y}} \gets \widetilde{\symbf{Y}}_{\text{ve}}\left[ :\None, :\Ntwo \right]$;
            \State \textbf{return} $\widetilde{\symbf{Y}}$;
        \EndProcedure
        \Statex
        \Procedure{VirtualEcho$2$D}{$\bY_{\cos} \in \mathbb{C}^{\None \times \Ntwo}, \bY_{\sin} \in \mathbb{C}^{\None \times \Ntwo} $}
            \State $\symbf{\Psi}_{++} \gets
                \Re\left(\bY_{\cos}\right) -
                \Im\left(\bY_{\sin}\right) + \iu \left(
                \Im\left(\bY_{\cos}\right) +
                \Re\left(\bY_{\sin}\right)
            \right)
            $;
            \State $\symbf{\Psi}_{+-} \gets
                \Re\left(\bY_{\cos}\right) +
                \Im\left(\bY_{\sin}\right) + \iu \left(
                \Re\left(\bY_{\sin}\right) -
                \Im\left(\bY_{\cos}\right)
            \right)
            $;
            \State $\symbf{\Psi}_{-+} \gets
                \Re\left(\bY_{\cos}\right) +
                \Im\left(\bY_{\sin}\right) + \iu \left(
                \Im\left(\bY_{\cos}\right) -
                \Re\left(\bY_{\sin}\right)
            \right)
            $;
            \State $\symbf{\Psi}_{--} \gets
                \Re\left(\bY_{\cos}\right) -
                \Im\left(\bY_{\sin}\right) - \iu \left(
                \Im\left(\bY_{\cos}\right) +
                \Re\left(\bY_{\sin}\right)
            \right)
            $;
            \State $\symbf{Z} \gets \symbf{0} \in \mathbb{C}^{\None \times \Ntwo}$
            \State $\symbf{T}_1 \gets
                \begin{bmatrix}
                    \symbf{\Psi}_{++} & \symbf{Z} \\
                    \symbf{Z} & \symbf{Z}
                \end{bmatrix}
            $;
            \State $\symbf{T}_2 \gets
                \begin{bmatrix}
                    \symbf{Z} & \symbf{\Psi}_{+-}^{\leftrightsquigarrow (2)} \\
                    \symbf{Z} & \symbf{Z}
                \end{bmatrix}^{\circlearrowright (2)}
            $;
            \State $\symbf{T}_3 \gets
                \begin{bmatrix}
                    \symbf{Z} & \symbf{Z}\\
                    \symbf{\Psi}_{-+}^{\leftrightsquigarrow (1)} & \symbf{Z}
                \end{bmatrix}^{\circlearrowright (1)}
            $;
            \State $\symbf{T}_4 \gets
                \begin{bmatrix}
                    \symbf{Z} & \symbf{Z}\\
                    \symbf{Z} & \symbf{\Psi}_{--}^{\leftrightsquigarrow (1,2)}
                \end{bmatrix}^{\circlearrowright (1,2)}
            $;
            \State $\bY_{\text{ve}} \gets \symbf{T}_1 + \symbf{T}_2 + \symbf{T}_3 + \symbf{T}_4$;
            \For{$\none = 0, \cdots, 2 \None - 1$}
                \State $\bY_{\text{ve}}\left[\none, 0\right] \gets
                    \nicefrac{\bY_{\text{ve}}\left[\none, 0\right]}{2}
                $;
            \EndFor
            \For{$\ntwo = 0, \cdots, 2 \Ntwo - 1$}
                \State $\bY_{\text{ve}}\left[0, \ntwo \right] \gets
                    \nicefrac{\bY_{\text{ve}}\left[0,\hspace*{1pt} \ntwo \right]}{2}
                $;
            \EndFor
            \State \textbf{return} $\bY_{\text{ve}}$;
        \EndProcedure
    \end{algorithmic}
\end{algorithm}

\begin{algorithm}[h!]
    \begin{algorithmic}[1]
        \caption{Filtering procedure for 2DJ data.}
        \label{alg:filter-2dj}
        \Procedure{Filter$2$DJ}{
            $\bY \in \mathbb{C}^{\None \times \Ntwo},
            \symbf{r}_{\text{interest}} \in \mathbb{N}_0^2,
            \symbf{r}_{\text{noise}} \in \mathbb{N}_0^2$
        }
            \State $\bY_{\text{ve}} \gets \symbf{0} \in \mathbb{C}^{\None \times 2 \Ntwo}$;
            \For{$\none = 0, \cdots, \None - 1$}
                \State $\bY_{\text{ve}}\left[\none, :\right] \gets \textsc{VirtualEcho$1$D}\left(\bY\left[\none, :\right]\right)$;
            \EndFor
            \State $\symbf{S}_{\text{ve}} \gets \FT^{(2)}\left(\bY_{\text{ve}}\right)$;
            \State $l^{(2)}_{\text{idx}}, r^{(2)}_{\text{idx}} \gets \symbf{r}_{\text{interest}}[0], \symbf{r}_{\text{interest}}[1]$;
            \State $l^{(2)}_{\text{idx,noise}}, r^{(2)}_{\text{idx,noise}} \gets \symbf{r}_{\text{noise}}[0], \symbf{r}_{\text{noise}}[1]$;
            \State $c_{\text{idx}}^{(2)} \gets \nicefrac{\left(l_{\text{idx}}^{(2)} + r_{\text{idx}}^{(2)}\right)}{2}$;
            \State $b_{\text{idx}}^{(2)} \gets r_{\text{idx}}^{(2)} - l_{\text{idx}}^{(2)}$;
            \State $\symbf{g}^{(1)} \gets \symbf{1} \in \mathbb{R}^{\None}$;
            \State $\symbf{g}^{(2)} \gets \textsc{SuperGaussian$1$D}\left(\Ntwo_{\vphantom{\text{idx}}}, c_{\text{idx}}^{(2)}, b_{\text{idx}}^{(2)}\right)$;
            \State $\symbf{G} \gets \symbf{g}^{(1)} \otimes \symbf{g}^{(2)}$;
            \State $\symbf{S}_{\text{noise}} \gets \symbf{S}_{\text{ve}} \left[
                :, l^{(2)}_{\text{idx,noise}} : r^{(2)}_{\text{idx,noise}} + 1
            \right]
            $;
            \State $\sigma^2 \gets \Var\left(\symbf{S}_{\text{noise}}\right)$;
            \State $\symbf{W}_{\sigma^2} \gets \symbf{0} \in \mathbb{R}^{\None \times 2 \Ntwo}$;
            \For {$\none = 0, \cdots, \None - 1$}
                \For {$\ntwo = 0, \cdots, 2\Ntwo - 1$}
                    \State $\symbf{W}_{\sigma^2}\left[ \none, \ntwo \right] \gets \textsc{RandomSample}\left(\mathcal{N}\left(0, \sigma^2\right)\right)$;
                \EndFor
            \EndFor
            \State $\widetilde{\symbf{S}}_{\text{ve}} \gets \symbf{S}_{\text{ve}} \odot \symbf{G} + \symbf{W}_{\sigma^2} \odot \left(\symbf{1} - \symbf{G}\right)$;
            \State $\widetilde{\symbf{Y}}_{\text{ve}} \gets \IFT^{(2)} \left( \widetilde{\symbf{S}}_{\text{ve}} \right)$;
            \State $\widetilde{\symbf{Y}} \gets \widetilde{\symbf{Y}}_{\text{ve}}\left[ :, :\Ntwo \right]$;
            \State \textbf{return} $\widetilde{\symbf{Y}}$;
        \EndProcedure
    \end{algorithmic}
\end{algorithm}

\begin{algorithm}
    \caption{An algorithm for multiplet assignment of a \ac{2DJ} estimation result.}
    \label{alg:mp-assign}
    \begin{algorithmic}[1]
        \Procedure {MultipletAssign}{$\bth \in \mathbb{R}^{6M}, \epsilon \in \mathbb{R}_{>0}$}
        \State $\bdfone, \bdftwo \gets \bth\left[2M : 3M\right], \bth\left[3M : 4M\right]$;
        \State $\textsc{Map} \gets \textsc{HashMap}\left[\mathbb{R}, \textsc{Vector}\left[\mathbb{N}_0\right]\right]$;
        \For {$m = \lbrace 0, \cdots, M-1 \rbrace$}
            \State $f_{\text{c}} = \bdftwom - \bdfonem$;
            \State $\textsc{Assigned} \gets \textsc{False}$;
            \For {$f_{\text{mp}}, \symbf{i}$ \textbf{in} \textsc{Map}}
                \If{$\left \lvert f_{\text{c}} - f_{\text{mp}} \right \rvert < \epsilon$}
                    \State $\symbf{i} \gets \left[ \symbf{i}\T \hspace{5pt} m \right]\T$;
                    \State $\textsc{Assigned} \gets \textsc{True}$;
                    \State \textbf{break};
                \EndIf
            \EndFor
            \If{$\textsc{Assigned} = \textsc{False}$}
                \State \textsc{Map}.insert $\left(f_{\text{c}}, \left[m\right]\right)$;
            \EndIf
        \EndFor
        \State \textbf{return} \textsc{Map};
        \EndProcedure
    \end{algorithmic}
\end{algorithm}

\clearpage
