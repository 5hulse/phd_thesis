\chapter{Code Listings}

\note{NEED TO ADD TEXT TO DESCRIBE THINGS}
The following is all the required imports:
\mylisting{python3}{code_listings/imports.py}{%
Required imports for the subsequent \Python listings.
}{
    Required imports for the subsequent listings in this chapter.
    The \mintinline{python3}{product} function is employed in the \acs{MMEMPM}
    routine (Listing \ref{lst:mmempm}).
    The imports from the \mintinline{python3}{typing} module are used to
    annotate what the expected types are for each argument, and what the return
    type is.
    The \mintinline{python3}{numpy} and \mintinline{python3}{scipy} modules are
    ubiquitous in the listings, providing access to efficient routines for
    numerical computations.
}{lst:imports}

\section{\Aclp{MPM}}

\subsection{\acs{MDL}}
\mylisting{python3}{code_listings/mdl.py}{%
\Python implementation of the \acs{MDL} for estimation of the model order
of a \acs{1D} \acs{FID}.
}{%
The \acs{MDL} for estimation of the model order of a \acs{1D} \acs{FID}.
The first relative minimum in \mintinline{python}{mdl_vec} is determined to be
the estimate of $M$, rather than the global minimum (line \ref{ln:argmin}),
since the presence of very small singular values when \mintinline{python3}{k}
is large can lead to errors involving floating-point arithmetic.
}{lst:mdl}

\subsection{\acs{MPM}}
\mylisting{python3}{code_listings/mpm.py}{
    \Python implementation of the \acs{MPM} for estimation of a \acs{1D} \acs{FID}.
}{
    The \acs{MPM} for estimation of a \acs{1D} \acs{FID}, with the option of
    estimating the model order using the \acs{MDL}.
}{lst:mpm}

\subsection{\acs{MMEMPM}}
\mylisting{python3}{code_listings/mmempm.py}{
\Python implementation of the \acs{MMEMPM} for estimation of a \acs{2D}
hypercomplex \acs{FID}.
}{
The \acs{MMEMPM} for estimation of a \acs{2D} hypercomplex \acs{FID}. Due
to the very large size of the Hankel matrix $\symbf{E}_{\bY}$, a truncated
\acs{SVD} routine is employed, which determines only the first $M$
components of the decomposition. This is only available to arrays stored in
sparse form in \textsc{SciPy} (lines \ref{ln:sparse1}--\ref{ln:sparse2}).
}{lst:mmempm}

\section{\ac{NLP}}

\subsection{Trust Region Algorithm}

\mylisting{python3}{code_listings/trust_region.py}{
\Python implementation of the Steihaug-Toint trust region algorithm.
}{
Steihaug-Toint trust region algorithm. Included is a check for oscillators with
negative amplitudes, which causes the routine to terminate, in order for said
oscillators to be purged (lines \ref{ln:negamp1}--\ref{ln:negamp2}).
}{lst:tr}

\subsection{Computing
    \texorpdfstring{$\mathcal{F}_{\phi}$}{F},
    \texorpdfstring{$\nabla \mathcal{F}_{\phi}$}{grad}, and
    \texorpdfstring{$\nabla^2 \mathcal{F}_{\phi}$}{hess}
}

\mylisting{python3}{code_listings/obj_grad_hess.py}{
\Python implementation for the generation of the fidelity for \ac{NLP}
applied to \ac{1D} estimation, as well as the gradient and (approximated)
Hessian
}{
Code for the generation of the fidelity for \ac{NLP}
applied to \ac{1D} estimation, as well as the gradient and (approximated)
Hessian. The \mintinline{python3}{FunctionFactory} object accepts a parameter
set (\mintinline{python3}{theta}) and function (\mintinline{python3}{fun}),
which computes the objective, gradient and Hessian. The first time a quantity
is requested from the factory, the function is run, and the objective and its
derivatives are cached (memoised), such that the next time a quantity is
requested, the cached result is used, rather than the expensive function being
re-computed.
\mintinline{python3}{FunctionFactoryGaussNewton1D} (lines
\ref{ln:ff1}--\ref{ln:ff2}) inherits from the base class, for specific use in
\ac{1D} estimation, with an approximated Hessian.}{lst:obj-grad-hess}

\subsection{The main routine}

\mylisting{python3}{code_listings/nlp_routine.py}{
    \Python implementation for running the \acs{NLP} routine for \acs{FID}
    estimation.
}{
    Code for running the \acs{NLP} routine for \acs{FID} estimation. The
    routine consists of running the \ac{ST} algorithm (Listing \ref{lst:tr})
    until it returns a parameter array without negative amplitudes. If negative
    amplitudes are present, the corresponding oscillators are removed, and the
    \ac{ST} algorithm is re-run.
}{lst:nlp}

\section{\acs{CUPID}}

\subsection{Assigning multiplet structures}

\mylisting{python3}{code_listings/mp_assign.py}{
    \Python implementation for performing multiplet assignment as part of
    \acs{CUPID}.
}{
    Code for performing multiplet assignment as part of \acs{CUPID}.
}{lst:mp-assign}
