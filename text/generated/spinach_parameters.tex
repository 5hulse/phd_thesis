
\begin{table}[h!]
\centering
\begin{tabular}{ccc}
\hline
Parameter & Four Multiplets & Sucrose\\
\hline
$f_{\text{bf}}^{(1)} (\unit{\mega\hertz})$ & 500 & 300\\
$\fofftwo$ (\unit{\hertz}) & 0 & 1000\\
$\fofftwo$ (\unit{\partspermillion}) & 0 & 3.333\\
$\fswone$ (\unit{\hertz}) & 40 & 30\\
$\fswtwo$ (\unit{\hertz}) & 1000 & 2200\\
$\fswtwo$ (\unit{\partspermillion}) & 2 & 7.333\\
$\None$ & 128 & 64\\
$\Ntwo$ & 1024 & 4096\\
\hline
\end{tabular}
\caption[
    Experiment parameters for \ac{2DJ} simulations run using \textsc{Spinach}.
]{
    Experiment parameters for \ac{2DJ} simulations run using \textsc{Spinach}.
}
\label{tab:spinach-jres-params}
\end{table}


\begin{table}[h!]
\centering
\begin{tabular}{cc}
\hline
Parameter &Five Multiplets\\
\hline
$f_{\text{bf}}^{(1)} (\unit{\mega\hertz})$ & 500\\
$\foffone$ (\unit{\hertz}) & 2500\\
$\foffone$ (\unit{\partspermillion}) & 5\\
$\fswone$ (\unit{\hertz}) & 5000\\
$\fswone$ (\unit{\partspermillion}) & 10\\
$\None$ & 16384\\
$K$ & 21\\
$\tau_{\text{max}}$ (\unit{\second}) & 4\\
\hline
\end{tabular}
\caption[
    Experiment parameters for inversion recovery simulations run using \textsc{Spinach}.
]{
    Experiment parameters for inversion recovery simulations run using \textsc{Spinach}.
    $K$ specifies the number of increments run, and $\tau_{\text{max}}$
    specifies the largest delay time used. Delays were generated with linear spacings,
    with the first delay always being \qty{0}{\second}, such that the
    n\textsuperscript{th} delay was $\nicefrac{(n-1)\tau_{\text{max}}}{K -
    1}$.
}
\label{tab:spinach-invrec-params}
\end{table}
