\documentclass[12pt]{article}

\usepackage{xr}
\externaldocument{../thesis}
\usepackage{datetime}
\dmyyyydate
\usepackage{xcolor}
%%% font configuration
\usepackage{fontspec}
\usepackage[%
    math-style=ISO,%
    bold-style=ISO,%
    nabla=upright,%
]{unicode-math}
\setmainfont{EBGaramond}[%
    Path           = ../fonts/EBGaramond/,%
    Extension      = .otf,%
    UprightFont    = *-Regular,%
    BoldFont       = *-SemiBold,%
    ItalicFont     = *-Italic,%
    BoldItalicFont = *-BoldItalic,%
    FontFace       = {sb}{n}{*-SemiBold},%
]
\setmathfont{Garamond-Math}[%
    Path           = ../fonts/,%
    Extension    = .otf,%
    StylisticSet = {2,8,9,10},%
]
\setmonofont{FiraMono}[%
    Scale       = MatchLowercase,%
    Path        = ../fonts/FiraMono/,%
    Extension   = .ttf,%
    UprightFont = *-Regular,%
    BoldFont    = *-Bold,%
]

\usepackage{acronym}
\usepackage{varioref}
\usepackage[capitalize,noabbrev]{cleveref}
\crefformat{equation}{Equation~#2#1#3}
\Crefformat{equation}{Equation~#2#1#3}
\crefrangeformat{equation}{Equations~(#3#1#4) to~(#5#2#6)}
\Crefrangeformat{equation}{Equations~(#3#1#4) to~(#5#2#6)}
\crefmultiformat{equation}{Equations~#2#1#3}%
{ and~#2#1#3}{, #2#1#3}{ and~#2#1#3}
\Crefmultiformat{equation}{Equations~#2#1#3}%
{ and~#2#1#3}{, #2#1#3}{ and~#2#1#3}

\usepackage{enumitem}
\setlist[itemize]{label=\null,leftmargin=*}

\author{Simon Hulse}
\title{Notes on Thesis Corrections}
\date{\today}

\newcommand{\figref}[1]{\cref{#1} on \cpageref{#1}}

\begin{document}
    \maketitle

    \subsubsection*{General}
    \begin{itemize}
        \item I have made some stylistic changes to the thesis (specifically, the
            formatting of the title page and chapter titles). Ralph asked about Oxford
            formatting guidelines. I am unaware of Oxford having specific thesis format
            guidelines, and haven't been able to find any guidance on this, so I am
            assuming that my thesis format is acceptable.
        \item Spaces were added to citations.
        \item In the corrected thesis, corrections which aren't simply typo fixes are in red.
    \end{itemize}
    \subsubsection*{Corrections by page}
    \begin{itemize}
        \item \cpageref{corr:prot-neut} -- Specified that only nuclei with odd
            numbers of protons and/or neutrons possess spin.
        \item \cpageref{corr:mol-rot} -- Mentioned molecular rotation as
            an additional source of angular momentum.
        \item \cpageref{eq:I-squared} -- \cref{eq:I-squared}: Squared the
            reduced Planck constant.
        \item \cpageref{corr:with-spin} -- Replaced ``with non-zero spin'' with
            ``with spin''.
        \item \cpageref{tab:nuclei} -- \cref{tab:nuclei}: Updated caption to
            include source of gyromagnetic ratios.
        \item \cpageref{corr:no-spin} -- Replaced ``which are spin-0'' with ``which
            do not possess spin''.
        \item \cpageref{fig:energy_levels} --  \cref{fig:energy_levels}:
            Mentioned the sign of $\gamma$ for each nucleus in the caption.
        \item \cpageref{corr:high-temp} --  Mentioned that the high temperature
            approximation is applied in arriving at \cref{eq:M0}. See also
            \cref{fn:corr-high-temp}.
        \item \cpageref{corr:rf-pulse} --  Re-worded description of RF pulse.
            See also \cref{fn:resonant-pulse}.
        \item \cpageref{eq:rot-frame} -- \cref{eq:rot-frame}: Corrected the
            expressions for $\tilde{\symbf{i}}$ and $\tilde{\symbf{j}}$.
        \item \cpageref{corr:relaxation} -- Included a more detailed
            qualitative description of relaxation.
        \item \cpageref{corr:vacuum} -- Explicitly mentioned the presence of a
            vacuum chamber.
        \item \cpageref{corr:solid-state} -- Included solid-state NMR as an
            area that requires very high field strengths.
        \item \cpageref{corr:small-inhomog} -- ``inhomogeneities''
            $\rightarrow$ ``small inhomogeneities''.
        \item \cpageref{corr:probe} -- A few extra details about the probe.
        \item \cpageref{corr:travel} -- ``is sent to'' $\rightarrow$ ``travels to''.
        \item \cpageref{fn:ref-decon} -- Peter suggested that I mention how
            data could be treated if there were lineshape distortions
            (annotation on Section 3.1.2 heading). \Cref{fn:ref-decon} mentions
            reference deconvolution as a means of correcting these in order to
            yield Lorentzian lineshapes.
        \item \cpageref{corr:sw} -- Replace ``sweep width'' with ``spectral
            width''. This has been done in numerous places in the thesis.
        \item \cpageref{eq:x-1d} -- \cref{eq:x-1d}: Correct equation label.
        \item \cpageref{corr:exp-apod} -- Mentioned that exponential broadening
            is not the optimal window function for sensitivity enhancement. See
            also \cref{fn:dolph-cheb}.
        \item \cpageref{corr:gaussian} -- Improved comparison of Gaussian vs
            Lorentzian lineshapes.
        \item \cpageref{corr:trunc} -- Reworded paragraph on truncation artefacts.
        \item \cpageref{corr:kram-kron} -- Elaborated on Kramers-Kronig
            relations, and included a citation.
        \item \cpageref{fn:lock-disp} -- Added \cref{fn:lock-disp} to mention
            the lock's use of dispersion lineshapes for monitoring field
            drifts.
        \item \cpageref{corr:delete-T2} -- Removed footnote discussing
            consideration of linewidth for $T_2$ measurement, as this is not
            reliable.
        \item \cpageref{corr:varpro-amares} -- Ralph commented that iterative methods
            are employed routinely in \textsuperscript{13}C NMR for
            metabolomics fingerprinting. I am unaware of this; from what I am
            aware, the typical method of performing metabolomics fingerprinting
            is to break up spectrum into small regions (bins), integrate these
            bins, and then input the integrals into some routine for multivariate
            analysis, such as PCA. I have added the phrase ``like VARPRO and
            AMARES'' to clarify what I mean by an ``iterative method''.
        \item \cpageref{corr:1d-vs-2d} -- Improved wording of why holistically
            analysing a 2D dataset can be better than sequentially analysing 1D
            increments.
        \item \cpageref{corr:prev-work} -- Added citations to make the fact
            that the routine makes use of previous theory more explicit.
        \item \cpageref{eq:complex-normal} -- Peter mentioned that a square
            root was absent in the probability density (what was Equation 2.3
            in my pre-viva draft). The square root wasn't present as the
            expression was the product of the PDFs of the real and imaginary
            components of a particular datapoint. I have rewritten this to make
            the origin of the scaling factor more clear.
        \item \cpageref{fn:matrix-pencil} -- Added \cref{fn:matrix-pencil} to
            give the definition of a matrix pencil.
        \item \cpageref{corr:rect-filter} -- Mentioned that the filtering
            process could likely be replaced by the simpler method of using a
            rectangular filter and slicing at the filter boundaries.
        \item \cpageref{fig:mpm_vs_nlp} -- \cref{fig:mpm_vs_nlp}: Replaced
            landscape figure with portrait version. Included a description of the
            different peak colours in the caption.
        \item \cpageref{fig:andro-onedim} -- \cref{fig:andro-onedim}: Replaced
            landscape figure with portrait version. Added structure of andrographolide.
        \item \cpageref{fig:cyclosporin} -- \cref{fig:cyclosporin}: Added structure of
            cyclosporin, and edited the caption accordingly.
        \item \cpageref{corr:spin-molecule} -- Replaced ``the spin tumbles''
            with ``the molecule that the spin is associated with tumbles''.
        \item \cpageref{corr:t1-t2} -- $T_1 \ll T_2 \rightarrow T_1 \gg T_2$.
        \item \cpageref{corr:c-unit} -- Fixed unit of constant $c$ in
            Stejskal-Tanner.
        \item \cpageref{fig:diffusion_sequences} -- \cref{fig:diffusion_sequences}:
            Tweaked the relative widths of pulses and gradients.
        \item \cpageref{corr:seq-errors} -- Mentioned that the errors for $D,
            T_1, T_2$ measurements can be obtained using the Hessian at
            convergence in the main text. An explicit derivation of the
            expression for errors is provided in the appendix.
        \item \cpageref{fig:five-multiplets-invrec} -- \cref{fig:five-multiplets-invrec}:
            Edited panel d; changed the aspect ratio and viewing angle.
            Hopefully the lineshapes look better with the altered view. If you
            still think it looks odd, I could simply remove the panel
            altogether; it is not the most crucial aspect of the figure.
        \item \cpageref{fig:andrographolide-dosy} -- \cref{fig:andrographolide-dosy}:
            Replaced landscape figure with portrait version. Added structure of
            andrographolide.
        \item \cpageref{fig:gluc_val_thre} -- \cref{fig:gluc_val_thre}: Added
            structures of valine, threonine and major anomeric forms of glucose.
        \item \cpageref{fig:strychnine-cupid} -- \cref{fig:strychnine-cupid}: Added
            structure of strychnine.
        \item \cpageref{fig:quinine-cupid} -- \cref{fig:quinine-cupid}: Added structure
            of quinine.
        \item \cpageref{fig:camphor-cupid} -- \cref{fig:camphor-cupid}: Added structure
            of camphor.
        \item \cpageref{fig:dexamethasone-cupid} -- \cref{fig:dexamethasone-cupid}:
            Replaced landscape figure with portrait version. Added structure of
            dexamethaone.
        \item \cpageref{fig:estradiol-cupid} -- \cref{fig:estradiol-cupid}: Added
            structure of estradiol.
    \end{itemize}
\end{document}
