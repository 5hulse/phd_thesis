\usepackage{amsmath}
\allowdisplaybreaks[1]
\DeclareMathOperator{\diag}{diag}
\DeclareMathOperator{\IFT}{IFT}
\DeclareMathOperator{\range}{range}
\DeclareMathOperator{\rank}{rank}
\DeclareMathOperator{\Var}{Var}
\DeclareMathOperator{\SNR}{SNR}
\DeclareMathOperator{\SVD}{SVD}
\DeclareMathOperator{\EIGENVALUES}{EIGENVALUES}

\usepackage{amsthm}
\newtheorem{remark}{Remark}
\newtheorem{definition}{Definition}
\newtheorem{theorem}{Theorem}

\usepackage{datetime}
\dmyyyydate

\usepackage{nicefrac}
\usepackage{upgreek} % For upright capital delta
\usepackage{textgreek} % For Greek symbols in text mode
\usepackage{MnSymbol} % For conditional independence symbol
\usepackage{chemformula}
\usepackage{pdfpages} % to insert NMR-EsPy walkthrough
\usepackage{lscape} % for CUPID metrics table
\usepackage{rotating} % for sidewaysfigure environment
\usepackage{subfig} % for GUI figure: spans 2 pages
\usepackage{booktabs,longtable}  % for tables
\usepackage{graphicx}
\graphicspath{{figures/}}

%%% Refer to same footnote in multiple places
% https://tex.stackexchange.com/questions/35043/reference-different-places-to-the-same-footnote
\makeatletter
\newcommand\footnoteref[1]{\protected@xdef\@thefnmark{\ref{#1}}\@footnotemark}
\makeatother
% prevent footnote breaking
\interfootnotelinepenalty=10000

%%% Nomenclature: see text/front/nomenclature.tex
\usepackage{nomencl}
\renewcommand{\nompreamble}{\pagestyle{plain}}
\makenomenclature
\usepackage{ifthen}

%%% Unit configuration
\usepackage[
     print-unity-mantissa=false,%
]{siunitx}
\DeclareSIUnit \gauss {G}
\DeclareSIUnit \molar {M}
\DeclareSIUnit \partspermillion {ppm}
\sisetup{detect-all,input-digits=0123456789\mitpi}

\usepackage[printonlyused]{acronym}  % See text/front/acrons.tex


%%% Circled numbers
\usepackage{tikz}
\newcommand*\circled[1]{{\scriptsize\tikz[baseline=(char.base)]{
            \node[shape=circle,draw,inner sep=1pt] (char) {#1};}}}

\usepackage{bibentry}  % To insert references in main text
